\documentclass{beamer}
%\documentclass[10pt,sansserif,blue,dvips,ignorenonframetext]{beamer}

%: text for the paper version is ignored

\mode<presentation>
{
  %1st CHOICE:
  \usetheme{Darmstadt}
  %2nd CHOICE:
  %\usetheme{Dresden}
  %3rd CHOICE:
  %\usetheme{Warsaw}
  %\setbeamercovered{transparent}
}
\mode<article>
{
  \usepackage{fullpage}
  \usepackage{pgf}
  \usepackage{hyperref}
  \setjobnamebeamerversion{beamerexample2.beamer}
}

\usepackage[english]{babel}
\usepackage[latin1]{inputenc}

\usepackage{times}
\usepackage{amsfonts}
\usepackage{amssymb}
\usepackage{amsmath}
\usepackage{euscript}

\usepackage{color}
\usepackage{fancybox}
\usepackage{graphicx}
\usepackage{boxedminipage}
\usepackage{lscape}
%\usepackage{algpseudocode}
\usepackage{epsf, epsfig,psfrag,graphicx,verbatim,url}

\title{The Public Sector in EURACE}
\subtitle[EURACE WS '09]{EURACE Winter School 2009}
\author{Sander van der Hoog}
\date[Genoa '09]{Genoa, 19 Nov 2009}

\begin{document}
\frame{\titlepage}

\section[Overview]{}
\begin{frame}{}
\frametitle{Overview}
Topics of this presentation:
\tableofcontents
\end{frame}

\section{The public sector module}
\begin{frame}{}
\frametitle{Main aim}
Major policy issue in the EU: Lisbon Strategy
\begin{quote}
`EU should become the most competitive and dynamic knowledge based economy'
\end{quote}

\bigskip
Objectives of the "Lisbon Strategy for Growth and Jobs" (2000):
\begin{itemize}
\item Fostering sustainable economic growth and job creation
\item Invest in R\&D activities: total (public and private) investment of $3\%$ of Europe's GDP in research and development by 2010
\item Invest in human capital: an employment rate of $70\%$ by the same date.
\end{itemize}
\end{frame}

\begin{frame}{}
\frametitle{Research questions}
\begin{itemize}
\item How do the actions of the Government affect the income of specific
groups of people (workers with different skill levels, households in
different regions)?

\item How do they affect the behavior of the different actors mentioned
above?

\item How do they affect the factor productivity and the (regional or
economy-wide) growth?
\end{itemize}
\end{frame}

\begin{frame}{}
\frametitle{Features}
Empirical calibration according to European data as much as possible:
\begin{itemize}
\item unemployment benefit schemes differ across European counties, e.g. the net replacement rate (the percentage of the last
earned wage paid out as unemployment benefit in the first month of unemployment)
\item size of Government expenditures as a fraction of GDP
\item distribution of the budget between different kind of expenditures
\end{itemize}
\end{frame}

\section{The Government module}
\begin{frame}{}
\frametitle{Stages of model development}
Stage 1: Government only has redistributive functions
\begin{itemize}
\item Simple taxation: only income and corporate profit tax (these are social security contributions)
\item Expenditures: unemployment benefits
\item A provisional Government budget is thus given by the total subsidies and total unemployment benefits.
\item Deficit financing: bond issuing to households, ECB, or fiat money creation by ECB
\item Any surplus is deposited at the central bank.
\end{itemize}
\end{frame}

\begin{frame}{}
\frametitle{Budget forecasting}
\begin{itemize}
\item The forecast for next year's GDP is a naive expectation that follows from extrapolating the current GDP growth rate:
\begin{equation}
Y^e_{t+1} = (Y_t/Y_{t-1}) Y_t.
\end{equation}
\item The Government budgeting starts with an estimate of next year's Government income which is expected to grow at the same rate as the GDP growth rate:
\begin{equation}
Y^g_{t+1} = (Y_t/Y_{t-1})Y^g_{t}.
\end{equation}
\item Projected Government expenditures:
	\begin{equation}
		G^e_{t+1} = (Y_t/Y_{t-1})G_t
	\end{equation}
	\item Projected Government budget balance (surplus/deficit):
	\begin{equation}
		B^e_{t+1} = Y^g_{t+1} - G^e_{t+1}
	\end{equation}
\end{itemize}
\end{frame}


\begin{frame}{}
\frametitle{Deficit financing}
When does the Government execute the actual financing of the deficit?
\begin{itemize}
\item The Government computes its budget deficit once per month, but enters the bond market on a daily basis.
\item The Government runs the budget accounting function each month to determine the monthly budget deficit.
\item The Government has a standing facility at the Central Bank that functions as a buffer account to finance ongoing payments.
\end{itemize}
\end{frame}

\begin{frame}{}
\frametitle{Deficit financing (cont.)}
Financing options:
\begin{itemize}
\item Bond financing: The budget deficit can be $100\%$ financed by selling bonds on the bond market to households
\item Money financing: Bonds can be sold to ECB, which creates fiat money to buy the bonds.
\item Quantitative easing: no bonds are sold, but the ECB directly creates fiat money for the government.
\end{itemize}
\end{frame}

\begin{frame}{}
\frametitle{Government balance sheet}
\begin{table}[ht!]
\begin{boxedminipage}{10cm}
\centering\leavevmode
\begin{tabular}{ll}
\underline{Positive cash flows} & \underline{Negative cash flows} \\
%\emph{Cash flow from public sector activities:} & \\
Tax revenues    & Investments\\
                & Consumption\\
                & Unemployment benefit payments\\
                & Subsidy payments\\
%\emph{Cash flow from financing activities:} & \\
New bond issues  & Bond interest payments\\
\line(1,0){75} & \line(1,0){85} \\
Total income    & Total expenses \\
\end{tabular}%
\end{boxedminipage}
\end{table}

\begin{table}[ht!]
\begin{boxedminipage}{10cm}
\centering\leavevmode
\begin{tabular}{ll}
\underline{Assets} & \underline{Liabilities} \\
Gov. cash holdings \hspace{1cm}  & Outstanding bonds \\
\end{tabular}%
\end{boxedminipage}
\end{table}
\end{frame}

\begin{frame}{Stage 2}
\frametitle{Stage 2: Subsidies}
Stage 2: Government distributes subsidies
\begin{itemize}
\item Expenditures: unemployment benefits, household subsidies
\item Subsidies are conditional on a specific purpose under which they
are granted:
	\begin{itemize}
	\item a firm can receive a subsidy for internal training of employees that
	will raise the general skills.
	\item a firm can receive an investment subsidy to buy investment goods
	\item a household can receive a consumption subsidy to buy consumption goods
	\end{itemize}
\item The subsidies are non-discriminatory: they are available to all firms
or households. 
\end{itemize}
\end{frame}

\begin{frame}{Stage 2}
\frametitle{Stage 2: Subsidies (cont.)}
\begin{itemize}
\item Allows the possibility of regional policies:
some subsidies are available in one region, not in the other.
\item Firms and households decide whether to apply for a subsidy.
The corresponding behavioral rules need to be modelled.
\item Firms or households who apply for a subsidy are certain to receive it, and
they spend it as required by the Government.
\end{itemize}
\end{frame}


\begin{frame}{Stage 3}
\frametitle{Stage 3: Government consumption and investment}
(Stage 3 is not yet implemented)
\begin{itemize}
\item Government consumption: Government purchases with CGP (interaction
with the malls).
\item Government investments: Government purchases with IGP (general
expenditure for any capital formation, e.g. infrastructure projects).
\end{itemize}

The investment and/or consumption can have diverse impacts, such as:
\begin{itemize}
\item lowering transportation costs (commuting and/or distribution costs)
\item augmenting the productivity of the firms
\item augmenting the productivity of labour
\end{itemize}
\end{frame}

\begin{frame}{Stage 4}
\frametitle{Stage 4: Government employment}
Government buys investment and consumption goods, hires labour, and uses it to produce a public good.
(Stage 4 is not yet implemented)

\begin{itemize}
\item The Government hires workers with specific characteristics (government officers).
\item Hiring can be local or interregional.
\item The public good can have diverse local or inter-regional impacts
	\begin{itemize}
	\item augmenting the productivity of the firms (technological infrastructure)
	\item augmenting the productivity of labour (general skill level)
	\end{itemize}
\end{itemize}
\end{frame}

\begin{frame}{Stage 5}
\frametitle{Stage 5: Government finance}
Advanced Government fiscal and monetary policy.
(Stage 5 is not yet implemented)
\begin{itemize}
\item As in Stage 1, but in addition the Government has a more refined financial policy; smoothing expenditures in response to macroeconomic variables
\item It manages its debt and/or surpluses on the money and bond market according
to standard fiscal and monetary  policy rules.
\end{itemize}
\end{frame}

\begin{frame}{}
\frametitle{Implementation details}
\begin{itemize}
\item At the start of the year: the Government announces new policies by
sending a general \url{policy_announcement_message}:
\begin{tabular}{l}
\url{gov_id}\\
\url{tax_rate_corporate}\\
\url{tax_rate_hh_labour}\\
\url{tax_rate_hh_capital}\\
\url{tax_rate_vat}\\
\url{unemployment_benefit_pct}\\
\url{hh_subsidy_pct}
\url{firm_subsidy_pct}
\end{tabular}
\end{itemize}
\end{frame}

\begin{frame}{}
\frametitle{}
\begin{itemize}
\item At the start of the year: all agents read the Government policy
announcements (only from their own Government), and store that information
in memory variables for later use. When an agent applies for any payment
(benefits, subsidies, or transfers) it sends a notification message to its
Government.
\end{itemize}
\end{frame}

\begin{frame}{}
\frametitle{}
\begin{itemize}
\item List of notification messages:
\begin{tabular}{l}
\hspace{-0.5cm}\small\url{tax_payment_message(gov_id, tax_payment)} \\
\hspace{-0.5cm}\small\url{unemployment_notification_message(gov_id, unemployment_benefit)} \\
\hspace{-0.5cm}\small\url{hh_subsidy_notification_message(gov_id, subsidy_payment)} \\
\hspace{-0.5cm}\small\url{firm_subsidy_notification_message(gov_id, subsidy_payment)} \\
\hspace{-0.5cm}\small\url{hh_transfer_notification_message(gov_id, transfer_payment)} \\
\hspace{-0.5cm}\small\url{firm_transfer_notification_message(gov_id, transfer_payment)}%
\end{tabular}

\item Government computes total payments by looping over these notification
messages each day and also computes monthly and yearly sums.
\end{itemize}
\end{frame}

\begin{frame}{}
\frametitle{Other messages in the Government module}
\begin{itemize}
\item When households become unemployed \emph{during} the month, they send
an \url{unemployment_notification} message to their Government immediately.
\item At the end of the subjective month all agents send their \url{tax_payment}
message, read by the Government daily and added to the Government's payment
account.
\item At the end of the year Eurostat sends multiple %
\url{data_for_Government} messages (one for each region) that contains the
data for that region.
\end{itemize}
\end{frame}

\section{The Eurostat module}
\begin{frame}{}
\LARGE Eurostat module
\end{frame}

\begin{frame}{}
\frametitle{Aims of the Eurostat module}
Data collection:
\begin{itemize}
\item Micro to macro: construction of macro variables following statistical procedures of real Eurostat.
\item Macro to micro: feedback to micro-level (downward causation). The macrovariables influence the behavior of Governments, Households, Firms, Banks.
\end{itemize}
\end{frame}

\begin{frame}{}
\frametitle{Aims of the Eurostat module}
Spatial levels:
\begin{itemize}
\item Regional data for a single region
\item National data for a set of regions belonging to a Government (\url{region_list})
\item Supra-national data for the economy as a whole
\end{itemize}

EU-27:
\begin{itemize}
\item 268 regions (the number of NUTS level 2 regions)
\item 27 states
\item 1 supranational entity
\end{itemize}
\end{frame}

\begin{frame}{}
\frametitle{Contents of Eurostat memory}
\begin{itemize}
    \item Storing histories of simple statistics: output, GDP, employment, ...
    \item Storing month-on-month growth rates and `monthly based' annual growth rates
    \item Storing quarter-on-quarter growth rates and `quarterly based' annual growth rates
    \item Composite indices: CPI
    \item Measuring imports and exports
    \item Firm demographics: firm birth and death rates, firm survival rates
    \item Indicators for the start/end of recessions, duration of recessions
\end{itemize}    
\end{frame}
\end{document}

\begin{frame}{}
\frametitle{History storage}
\begin{itemize}
\item The memory of the Eurostat agent contains a running history of monthly and quarterly data.
\item The ADT for storing an item in the history is available as:
\end{itemize}
\begin{verbatim}
      struct history_item
      {
            double cpi
            double gdp
            double output
            employment
            unemployment_rate
            unemployment_rate_skill_1
            unemployment_rate_skill_2
            unemployment_rate_skill_3
            unemployment_rate_skill_4
            unemployment_rate_skill_5
            double average_wage   
            double no_firms
            double no_firm_births
            double no_firm_deaths
       }
\end{verbatim}
\end{frame}

\begin{frame}{}
\frametitle{}
\begin{itemize}
\item 
\end{itemize}
\end{frame}

\begin{frame}{}
\frametitle{}
\begin{itemize}
\item 
\end{itemize}
\end{frame}

\end{document}