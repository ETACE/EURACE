\documentclass{beamer}
%\documentclass[10pt,sansserif,blue,dvips,ignorenonframetext]{beamer}

%: text for the paper version is ignored

\mode<presentation>
{
  %1st CHOICE:
  %\usetheme{Darmstadt}
  %2nd CHOICE:
  %\usetheme{Dresden}
  %3rd CHOICE:
  \usetheme{Warsaw}
  %\setbeamercovered{transparent}
}
\mode<article>
{
  \usepackage{fullpage}
  \usepackage{pgf}
  \usepackage{hyperref}
  \setjobnamebeamerversion{beamerexample2.beamer}
}

\usepackage[english]{babel}
\usepackage[latin1]{inputenc}

\usepackage{times}
\usepackage{amsfonts}
\usepackage{amssymb}
\usepackage{amsmath}
\usepackage{euscript}

\usepackage{color}
\usepackage{fancybox}
\usepackage{graphicx}
\usepackage{boxedminipage}
\usepackage{lscape}
%\usepackage{algpseudocode}
\usepackage{epsf, epsfig,psfrag,graphicx,verbatim,url}

\title{The Public Sector in EURACE}
\subtitle[EURACE WS '09]{EURACE Winter School 2009}
\author{Sander van der Hoog}
\institute{GREQAM, Universit\'{e} de la M\'{e}diterran\'{e}e Aix-Marseille II}
\date{18 November 2009}

\setcounter{tocdepth}{1}

\begin{document}
\begin{frame}{}
\thispagestyle{empty}
\centering
\includegraphics[scale=0.1]{EURACE_Flag.png}
\titlepage
\end{frame}

\section{}
\subsection{Overview}
\begin{frame}{}
\frametitle{Overview}
Topics of this presentation:
\tableofcontents
\end{frame}

\begin{frame}{}
\centering
\includegraphics[scale=0.2]{EURACE-Markets.png}
\end{frame}

\section{Main aim of the public sector module}
\subsection{Main aim}
\begin{frame}{}
\frametitle{Main aim}
Major policy issue in the EU: Lisbon Strategy (Lisbon, 2000)
\begin{quote}
`The EU should become the most competitive and dynamic knowledge-based economy in the world capable of sustainable economic growth with more and better jobs and greater social cohesion.'
\end{quote}

\smallskip
Objectives of the "Lisbon Strategy for Growth and Jobs" (2005):
\begin{itemize}
\item Invest in R\&D activities: total (public and private) investment of $3\%$ of Europe's GDP in research and development by 2010.
\item Invest in human capital: an employment rate of $70\%$ by the same date.
\end{itemize}
\end{frame}

\subsection{Research questions}
\begin{frame}{}
\frametitle{Research questions}
\begin{itemize}
\item How do the actions of the Government affect the income of specific
groups of people (workers with different skill levels, households in
different regions)?

\item How do they affect the behavior of the different actors mentioned
above?

\item How do they affect the factor productivity and the (regional or
economy-wide) growth?
\end{itemize}
\end{frame}

\section{The Government module}
\subsection{Features}
\begin{frame}{}
\frametitle{The Government module}
Features:
\begin{itemize}
\item Taxation: income tax, corporate tax (capital gains tax, VAT)
\item Unemployment benefits
\item Subsidies and transfers
\item (Government consumption and investment)
\item (Government employment)
\end{itemize}
\end{frame}

\subsection{Stages}
\begin{frame}{}
\frametitle{Stages of model development}
Stage 1: Government only has redistributive functions
\begin{itemize}
\item Simple taxation: only income and corporate profit tax (only unemployment insurance fees).
\item Expenditures: unemployment benefits, bond interest payments.
\item Deficit financing: bond issuing to households, ECB, or fiat money creation by ECB.
\item Any surplus is deposited in an account at the central bank.
\end{itemize}
\end{frame}

\subsection{Budgeting}
\begin{frame}{}
\frametitle{Budget projection}
\small
\begin{itemize}
\item The forecast for next year's GDP is a naive expectation that follows from extrapolating the current GDP growth rate:
\begin{equation}
Y^e_{t+1} = (Y_t/Y_{t-1}) Y_t.
\end{equation}
\item Next year's Government income is expected to grow at the same rate as the GDP growth rate:
\begin{equation}
I^g_{t+1} = (Y_t/Y_{t-1})I^g_{t}.
\end{equation}
\item Projected Government expenditures:
    \begin{equation}
        G^e_{t+1} = (Y_t/Y_{t-1})G_t
    \end{equation}
    \item Projected Government budget balance (surplus/deficit):
    \begin{equation}
        B^e_{t+1} = I^g_{t+1} - G^e_{t+1}
    \end{equation}
\end{itemize}
\end{frame}

\subsection{Deficit financing}
\begin{frame}{}
\frametitle{Deficit financing}
When does the Government execute the actual financing of the deficit?
\begin{itemize}
\item The Government computes its budget deficit once per month, but enters the bond market on a daily basis.
\item The Government runs the budget accounting function each month to determine the monthly budget deficit.
\item The Government has a standing facility at the Central Bank that functions as a buffer account to finance ongoing payments.
\end{itemize}
\end{frame}

\begin{frame}{}
\frametitle{Deficit financing (cont.)}
Financing options:
\begin{itemize}
\item Bond financing: The budget deficit can be $100\%$ financed by selling bonds on the bond market to households.
\item Quantitative easing: In case of bond market rationing, the unsold bonds can be sold to the ECB, which creates fiat money to buy the bonds.
\item Money financing: no bonds are sold, but the ECB directly creates fiat money for the government.
\end{itemize}
\end{frame}

\subsection{Balance sheet}
\begin{frame}{}
\frametitle{Government balance sheet}
\begin{table}[ht!]
\begin{boxedminipage}{10cm}
\centering\leavevmode
\begin{tabular}{ll}
\underline{Positive cash flows} & \underline{Negative cash flows} \\
%\emph{Cash flow from public sector activities:} & \\
Tax revenues    & Investments\\
                & Consumption\\
                & Unemployment benefit payments\\
                & Subsidy payments\\
%\emph{Cash flow from financing activities:} & \\
New bond issues  & Bond interest payments\\
\line(1,0){75} & \line(1,0){85} \\
Total income    & Total expenses \\
\end{tabular}%
\end{boxedminipage}
\end{table}

\begin{table}[ht!]
\begin{boxedminipage}{10cm}
\centering\leavevmode
\begin{tabular}{ll}
\underline{Assets} & \underline{Liabilities} \\
Gov. cash holdings \hspace{1cm}  & Outstanding bonds \\
\end{tabular}%
\end{boxedminipage}
\end{table}
\end{frame}

\subsection{Subsidies}
\begin{frame}{Stage 2}
\frametitle{Stage 2: Subsidies}
Stage 2: Government distributes subsidies
\begin{itemize}
\item Subsidies are conditional on a specific purpose under which they
are granted:
    \begin{itemize}
    \item a firm can receive a subsidy for internal training of employees that
    will raise the general skills.
    \item a firm can receive an investment subsidy to buy investment goods.
    \item a household can receive a consumption subsidy to buy consumption goods.
    \end{itemize}
\item The subsidies are non-discriminatory: they are available to all firms or households. 
\end{itemize}
\end{frame}

\begin{frame}{}
\frametitle{Stage 2: Subsidies (cont.)}
\begin{itemize}
\item Allows the possibility of regional policies:
some subsidies are available in one region, not in the other.
\item Firms and households decide whether to apply for a subsidy.
The corresponding behavioral rules need to be modelled.
\item Firms or households who apply for a subsidy are certain to receive it, and
they spend it as required by the Government.
\end{itemize}
\end{frame}

%\subsection{Stage 3: Government consumption and investment}
\begin{frame}{}
\frametitle{Stage 3: Government consumption and investment}
(Stage 3 is not yet implemented)
\begin{itemize}
\item Government consumption: Government purchases with CGP (interaction
with the malls).
\item Government investments: Government purchases with IGP (general
expenditure for any capital formation, e.g. infrastructure projects).
\end{itemize}

The investment and/or consumption can have diverse impacts, such as:
\begin{itemize}
\item lowering transportation costs (commuting and/or distribution costs).
\item augmenting the productivity of the firms.
\item augmenting the productivity of labour.
\end{itemize}
\end{frame}

%\subsection{Stage 4: Government employment}
\begin{frame}{Stage 4}
\frametitle{Stage 4: Government employment}
Government buys investment and consumption goods, hires labour, and uses it to produce a public good.
(Stage 4 is not yet implemented)

\begin{itemize}
\item The Government hires workers with specific characteristics (government officers).
\item Hiring can be local or interregional.
\item The public good can have diverse local or inter-regional impacts:
    \begin{itemize}
    \item augmenting the productivity of the firms (technological infrastructure).
    \item augmenting the productivity of labour (general skill level).
    \end{itemize}
\end{itemize}
\end{frame}

%\subsection{Stage 5: Government finance}
\begin{frame}{Stage 5}
\frametitle{Stage 5: Government finance}
Advanced Government fiscal and monetary policy.
(Stage 5 is not yet implemented)
\begin{itemize}
\item As in Stage 1, but in addition the Government has a more refined financial policy; smoothing expenditures in response to macroeconomic variables.
\item It manages its debt and/or surpluses on the money and bond market according
to standard fiscal and monetary  policy rules.
\end{itemize}
\end{frame}

\subsection{Implementation details}
\begin{frame}{}
\frametitle{Implementation details}
\begin{itemize}
\item At the start of the year: the Government announces new policies by
sending a general \url{policy_announcement_message}:
\begin{tabular}{l}
\url{gov_id}\\
\url{tax_rate_corporate}\\
\url{tax_rate_hh_labour}\\
\url{tax_rate_hh_capital}\\
\url{tax_rate_vat}\\
\url{unemployment_benefit_pct}\\
\url{hh_subsidy_pct}\\
\url{firm_subsidy_pct}
\end{tabular}
\end{itemize}
\end{frame}

\begin{frame}{}
\frametitle{}
\begin{itemize}
\item At the start of the year: all agents read the Government policy
announcements (only from their own Government).
\item Store that information in memory variables for later use.
\item When an agent applies for any payment (benefits, subsidies, or transfers) it sends a notification message to its
Government.
\end{itemize}
\end{frame}

\begin{frame}{}
\frametitle{}
\begin{itemize}
\item List of notification messages:

\begin{tabular}{l}
\hspace{-1.7cm}\small\url{tax_payment_message(gov_id, tax_payment)} \\
\hspace{-1.7cm}\small\url{unemployment_notification_message(gov_id, unemployment_benefit)} \\
\hspace{-1.7cm}\small\url{hh_subsidy_notification_message(gov_id, subsidy_payment)} \\
\hspace{-1.7cm}\small\url{firm_subsidy_notification_message(gov_id, subsidy_payment)} \\
\hspace{-1.7cm}\small\url{hh_transfer_notification_message(gov_id, transfer_payment)} \\
\hspace{-1.7cm}\small\url{firm_transfer_notification_message(gov_id, transfer_payment)}%
\end{tabular}

\item Government computes total payments by looping over these notification
messages each day and also computes monthly and yearly sums.
\end{itemize}
\end{frame}

\begin{frame}{}
\frametitle{Other messages in the Government module}
\begin{itemize}
\item When households become unemployed \emph{during} the month, they send
an \url{unemployment_notification} message to their Government immediately.
\item At the end of the subjective month all agents send their \url{tax_payment}
message, read by the Government daily and added to the Government's payment
account.
\item At the end of the year Eurostat sends multiple %
\url{data_for_Government} messages (one for each region) that contains the
data for that region.
\end{itemize}
\end{frame}

\section{The Eurostat module}
\begin{frame}{}
\LARGE Eurostat module
\end{frame}

\subsection{Aims}
\begin{frame}{}
\frametitle{Aims of the Eurostat module}
Data collection:
\begin{itemize}
\item Micro to macro: construction of macro variables following statistical procedures of real Eurostat.
\item Macro to micro: feedback to micro-level (downward causation). The macrovariables influence the behavior of Governments, Households, Firms, Banks.
\end{itemize}
\end{frame}

\begin{frame}{}
\frametitle{Aims of the Eurostat module}
Spatial levels:
\begin{itemize}
\item Regional data for a single region
\item National data for a set of regions belonging to a Government (\url{region_list})
\item Supra-national data for the economy as a whole
\end{itemize}

EU-27:
\begin{itemize}
\item 268 regions (the number of NUTS level 2 regions)
\item 27 states
\item 1 supranational entity
\end{itemize}
\end{frame}

\subsection{Contents of Eurostat memory}
\begin{frame}{}
\frametitle{Contents of Eurostat memory}
\begin{itemize}
    \item Constructed data: GDP, CPI, inflation rate, total output, unemployment rates, ...
    \item Stores histories of monthly and quarterly statistics
    \item Stores month-on-month growth rates and `monthly based' annual growth rates.
    \item Stores quarter-on-quarter growth rates and `quarterly based' annual growth rates.
    \item Measuring imports and exports between regions.
    \item Firm demographics: firm birth and death rates, firm survival rates.
    \item Indicators for the start and end of recessions, duration of recessions.
\end{itemize}    
\end{frame}

\subsection{History storage}
\begin{frame}{}
\frametitle{History storage}
The memory of the Eurostat agent contains a moving history of monthly and quarterly data:
\begin{itemize}
\item cpi
\item gdp
\item output
\item employment
\item unemployment rate averaged
\item unemployment rate per skill level
\item wage averaged
\item wage per skill level
\item number of active firms
\item number of firm `births' (reactivations)
\item number of firm `deaths' (bankruptcies)
\end{itemize}
\end{frame}


\subsection{GDP computation}
\begin{frame}{}
\frametitle{GDP computation}
GDP can measured using three different approaches:
\begin{enumerate}
\item The `expenditure' approach: measures total expenditure on finished or final goods and services produced in the domestic economy.

\item The `production' approach: measures the contribution of each economic unit by estimating the value of an output (goods or services) less the value of inputs used in that output's production process.

\item The `income' approach: measures the incomes earned by individuals (e.g. wages) and corporations (e.g. profits) in the production of outputs (goods or services).
\end{enumerate}

$\rightarrow$ Here: we use the expenditure approach.
\end{frame}

\begin{frame}{GDP computation (cont.)}
\frametitle{}
\[ GDP = \sum_{h\in H} C + \sum_{f\in F}I + \sum_{g\in I_g}G \]
\begin{itemize}
\item Household consumption expenditure: aggregated by the mall on a transaction basis
\item Firm investments: aggregated in the IGFirm as revenues from selling investment goods
\item Government consumption: aggregated by Eurostat
\end{itemize}
\end{frame}

\subsection{Imports and exports}
\begin{frame}{}
\frametitle{Imports and exports}
From the perspective of region 1:
\begin{figure}[ht!]
\centering\leavevmode
\resizebox{10cm}{!}{\includegraphics{ImportExport.png}}
\end{figure}
\end{frame}

\subsection{Start/end of recessions}
\begin{frame}{}
\frametitle{Indicators for the start/end of recessions}
\small
The definition for the start of a recession:
\begin{definition}
A recession begins at the beginning of the month at the start of two successive quarters with negative GDP growth: $X_{t+2}/X_{t+1}<1$ and $X_{t+1}/X_{t}<1$, where $X_{t}$ is quarterly GDP.
\end{definition}

The definition for the end of a recession:
\begin{definition}
A recession ends at the beginning of the month following one quarter with positive GDP growth after the start of a recession: $X_{t+k+1}/X_{t+k}\geq 1$. The duration of the recession is $k$ quarters.
\end{definition}
\end{frame}

\begin{frame}{Macro-feedback}
\frametitle{}
\begin{itemize}
\item Regional data: Eurostat knows all, so sends messages back
\item National data: each Government has a list of regions
\item Supra-national data: stored in Eurostat for analysis
\end{itemize}
\end{frame}

\end{document} 