In this report we have described the parallel implementation of the FLAME framework and its assessment together with some benchmarking results using the EURACE Model. We have also demonstrated FLAMEs use in a number of EURACE related simulations in addition to complete EURACE model on populations ranging from a few hundereds of agents, through tens of thousands to, in one case, a million agents.  In some of these simulations the parallel implementation of FLAME has shown reasonable scalability and parallel efficiency but in other the results have been disappointing.

An important goal of the project has been to perform, in parallel, a large simulation using the EURACE Model. The project has achieve this to a degree: the model has been defined, important parameters have been values, a method of generating agent populations implemented and a parallel implementation of the EURACE model can be generated by FLAME. Using these steps serial and parallel simulations of the EURACE Model have been perform. In this process a detail assessment of the FLAME generated code, the serial and parallel implementations and the EURACE Model have been performed. Message counts, function times and sychronisation times are a few of the measures that have been used together with a detail static analysis of the model to identify the performance defficiencies in both the FLAME framework and the EURACE model.

All this analysis has lead to improvements in FLAME and the EURACE Model which in general have improved its computational performance. However the presence of substanial serial components in any model has resulted in very poor parallel scalability. It is well known that parallel speedup is limited by the serial faction of a code - this is Amdah's Law. The analyses performed on the EURACE Model have shown that the singleton agents - in particularly the Clearing House - have a significant impact of the parallel performance of the model.
These types of potential problem were understood - fine grained tasks - at the start of the project and the modeller took steps to avoid them. The Clearing House was thought necessary to the architecture of the EURACE Model and although different strategies were tested to reduce its effect there was little that could be achieved. The Clearing House and any other serial bottleneck will compromise the parallel performance of the application.

Although at the end of EURACE we have not achieved the \textsl{optimum} solution to these problems we have at least advanced the current state of the art in the parallel implementation of agent-based simulations in the context of the FLAME Framework.
