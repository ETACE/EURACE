\subsection{Introduction to Cloning}
\label{sec:cloning}

As the EURACE model became more and more complicated so did the relationships between agents. These realtionships are both inter- and intra- regional and provide a complex set of constraints for the initial population of agents. As modellers and computer scientists demanded larger populations it was found that the Tubitak population GUI could not provide populations quickly enough and ran into memory problems even on very well specified machines. For this reason a simpler method of creating large populations was designed, that of \textit{cloning} one population many times replacing agent ids as required. The popGUI still has a roll, that of providing an easy way of specifying the population characteristics for one region, including statistical distributions for agent memory variables and inter-relationships between them. This data is stored in a \texttt{pop} file that is used to initiate the cloning process.

A variety of implementations were tried, \texttt{bash} scripts using \texttt{sed/awk}, python code and finally and most successfully C code that could run in serial or in parallel. The seed population for cloning has one region and so there are intially no inter-regional interactions but it is possible for these to occur in the transient stage of the simulation if the model is set up correctly. The lack of these interactions means that cloning is a perfectly parallel algorithm.

Cloning from a single region also has the benefit that it is possible to provide pre-partitioned input data for the parallel implementation of the EURACE model. For example, clone one region to give 16 regions and input files for 1, 2, 4, 8 and 16 processes can be easily produced.

The cloning process relies on a special \texttt{xml} file in which any agent ids are marked with \texttt{REPLACE\_ID\_n} where \texttt{n} is the id of the agent in the region being cloned. This \texttt{xml} file is produced from a \texttt{pop} file by the \texttt{instantiate.py} script written by Tubitak and using code from the population GUI. The script is invoked as
\begin {quote}
\tt
python instantiate.py -r 0\_bench\_oct\_12.pop 0\_markers\_oct\_12.xml
\end{quote}
Actually cloning the region is illustrated by the following example:

\begin{quote}
\tt
./clone.sh 0\_markers\_oct\_12.xml 8 tmp -r 2
\end{quote}

This clones \texttt{0\_markers\_oct\_12.xml} creating 2 input files each containing 8 regions. The output files are put into a directory called \texttt{16R\_2P} indicating that this is input for a 16 region run on 2 processes.

It is possible to control which agents are cloned by listing their names in the file \texttt{agent\_list.txt}.

\subsection{Implementation of Cloning}

The cloning application runs from the commandline using a \texttt{bash} script to control the cloning and organisation of output files, \texttt{sed} to perform some ad-hoc text manipulation of output files and a C application to do the computationaly intensive work of reading the input file and writing an output file. Both serial and parallel (MPI) versions of the C code are available.

The serial code works by repeatedly calling it with arguments that are the increment to make in id numbers between regions and a number (0 based) of this particular clone. The id increment is set to the number of agents in the original region since there can be no more agents in any cloned region. Running the code in paralle requires only the increment value to be given as each process can get the clone number as its node id by calling \texttt{MPI\_Comm\_rank}.

Source code is available from the EURACE Subversion repository:
\begin{quote}
http://ccpforge.cse.rl.ac.uk/svn/eurace/models/utils/cloning
\end{quote}
Using these tools we are able to generate large data sets from a small seed population.