%\section{Parallel Computing Systems Used In EURACE}
\label{app:parallel-systems}
A number of the partners in the EURACE project have their own parallel systems. The FLAME Framework and the EURACE Model have been ported to these systems.
\begin{table}[ht]
\renewcommand{\arraystretch}{1.2}
	\centering
	{\scriptsize
		\begin{tabular}{|l|l|l|l|}\hline
	\bf	Unit &	\bf GREQAM &\bf	 TUBITAK &\bf	 UNIBI\\\hline
\bf Processor Type 	& Intel Xeon 5140 (Dual Core)&	Intel Xeon E5355 (Quad Core) &	Intel Xeon 5160 (Dual Core)\\\hline
\bf Total Cores 	&4 (2 x 2) &	4 (1 x 4) &	4 (2 x 2)\\\hline
\bf Total Memory &	4GB (4 x 1GB) &	16GB (4 x 4GB) &	2GB\\\hline
\bf Memory per core &	1GB &	4GB &	512MB\\\hline
\bf Total Storage &	146GB (2 x 73GB) &	292GB (2 x 146GB) &	219GB (3 x 73GB)\\\hline
\bf Usable Storage &	~73GB (RAID 1) &	 ? &	 ?\\\hline
\bf Operating System &	Windows XP Pro x64 &	 ? &	Red Hat Enterprise Linux 4\\\hline
\bf MPI Library$^1$ 	& MPI 1 &	 MPI 1 &	 MPI 1 \\\hline
			
		\end{tabular}
		}
\end{table}
STFC has a large number of different parallel computing systems which it has made available to the project and used in testing the EURACE Model. Each of these machine has a different hardware architecture and software infrastructure.
\begin{description}
\item[HPCx]: The HPCx platform is currently number 43 in the 28th Top 500 Supercomputer list (Nov 2006). It is based on the IBM pSeries 575 system, and has a total of 2560 processors. The HPCx system uses IBM eServer 575 nodes for the compute and IBM eServer 575 nodes for login and disk I/O. Each eServer node contains 16 processors. At present there are two service nodes. The main HPCx service provides 160 nodes for compute jobs for users, giving a total of 2560 processors. There is a separate partition of 12 nodes reserved for certain projects. The peak computational power of the HPCx system is 15.3 Tflops peak.
\item[SCARF]: SCARF is a compute cluster run by the STFC's e-Science Centre (HPC services group). It uses the RedHat Enterprise Linux 4.4 Operating System, and provides the LSF scheduler and SCALI-MPI. SCARF has the following hardware specification:360 2.2GHz AMD Opteron processor cores, 1.296TB total memory, Gigabit networking and Myrinet M3F-PCIXD-2 low latency interconnect.
\item[HAPU]: HAPU is an HP Cluster Platform 4000 based on Redhat Enterprise Linux 4. It has 128 x 2.4GHz Opteron cores, with 2Gb memory per core, and a Voltaire InfiniBand interconnect.
\item[NW-GRID]: The NW-GRID Cluster comprises three compute racks, with each rack containing 32 SUN x4100 nodes. Each node contains 2 Dual Core 2.4Ghz Opterons with 8GB of memory. That brings the total processor count to 192 Dual-Core Opterons (384 processor cores). 
\item[MANO]: MANO is an IBM Blue Gene/L machine. It comprises 1024 nodes of dual-core 700MHz PowerPC chips with the second cpu usually dedicated i/o and communications. The frontend (or login) node is a p5-520Q with 4x1.5GHz processors, 16GB RAM and running SuSE Linux Enterprise Server 9 and this is supplemented with an identical service node for system control. GPFS is provided through two p5-505 servers each with 2x1.5GHz processors and 4GB RAM.
\item[bglogin2]: is a single frame of a IBM Blue Gene/P machine. A standard Blue Gene/P configuration will house 4,096 processors per rack. Four 850 MHz PowerPC 450 processors are integrated on each Blue Gene/P chip. It is at least seven times more energy efficient than any other supercomputer, accomplished by using many small, low-power chips connected through five specialized networks.
\item[Hector]: is a Cray XT4 scalar supercomputer. The XT4 comprises 1416 compute blades, each of which has 4 quad-core processor sockets. This amounts to a total of 22,656 cores, each of which acts as a single CPU. The processor is an AMD 2.3 GHz Opteron. Each quad-core socket shares 8 GB of memory, giving a total of 45.3 TB over the whole XT4 system. The theoretical peak performance of the system is 208 Tflops. There are also 24 service blades, each with 2 dual-core processor sockets. They act as login nodes and controllers for I/O and for the network. In addition there is a Cray vector Blackwidow part of the system which includes 28 vector compute nodes; each node has 4 Cray vector processors, making 112 processors in all. Each processor is capable of 25.6 Gflops, giving a peak performance of 2.87 Tflops. Each 4-processor node shares 32 GB of memory.
\end{description}
