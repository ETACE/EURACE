\documentclass[a4paper,11pt]{article}
\usepackage[pdftex]{graphicx}
\usepackage{lscape}

%\usepackage{draftwatermark}
%\SetWatermarkScale{5.0}


\vfuzz2pt % Don't report over-full v-boxes if over-edge is small
\hfuzz2pt % Don't report over-full h-boxes if over-edge is small

\setlength{\oddsidemargin}{0cm}
\setlength{\evensidemargin}{0cm}
\setlength{\textwidth}{16cm}
\setlength{\textheight}{24cm}
\setlength{\topmargin}{-1cm}

\begin{document}
\thispagestyle{empty}

% Common EURACE Title Page
% EURACE and FW6 logos
\vspace{\baselineskip}
\includegraphics[width=45mm]{EURACE-logo.png}		
\hfill
\includegraphics[width=45mm]{FW6-logo.png}

% Title
\begin{center}
Project no.\\
035086\\
Project acronym\\
{\bf EURACE}\\
Project title\\
{\bf An Agent-Based software platform for European economic policy design with heterogeneous interacting agents: new insights from a bottom up approach to economic modelling and simulation}\\
\end{center}

\vspace*{\baselineskip}\noindent
Instrument: STREP\\[\baselineskip]
Thematic Priority: IST FET PROACTIVE INITIATIVE ``SIMULATING EMERGENT PROPERTIES IN COMPLEX SYSTEMS\\

% Deliverable information: title & number
\vspace*{2\baselineskip}
\begin{center}
{\bf
Deliverable reference number and title\\
%D1.4: Porting of agent models to parallel computers\\
D8.4: Porting of the software platform to parallel computers
}
Due date of deliverable:\\
%31/08/2008\\
31/08/2009\\
Actual submission date:\\
%30/09/2008\\
30/09/2009\\
\end{center}


% Project Info
\vspace*{\baselineskip}\noindent
{Start date of project: September 1$^{st}$} 2006 \hfill {Duration: 36 months}\\

%Deliverable Info - partner
\vspace{2\baselineskip}\noindent
Organisation name of lead contractor for this deliverable\\
{\bf STFC Rutherford Appleton Laboratory - STFC}
\begin{flushright} 
Revision 2
\end{flushright}

\vspace{\baselineskip}
\begin{table}[hb]
\begin{tabular}{||c|l|l||} \hline\hline
\multicolumn{3}{||l||}{\small Project co-funded by the European Commission within the Sixth Framework Programme (2002-2006)}\\ \hline
\multicolumn{3}{||l||}{\bf Dissemination Level}\\ \hline
\bf PU &\small Public\hfill~& \bf X \\ \hline
\bf PP &\small Restricted to other programme participants (including the Commission Services)& \\ \hline 
\bf RE &\small Restricted to a group specified by the consortium (including the Commission Services)&  \\ \hline
\bf CO &\small Confidential, only for members of the consortium (including the Commission Services)&  \\ \hline \hline
\end{tabular}
\end{table}
\pagebreak
% End of EURACE Title Page

\pagenumbering{roman}
% Table of Contents
\tableofcontents
\pagebreak

% Abstract/SUmmary
\begin{abstract}\noindent
Making use of high performance computers in agent-based simulation is a complex and difficult task. This report describes the approach being explored within the EURACE project to exploit parallel computing technology in large-scale agent-based simulations involving millions of agents. The underlying software is the FLAME framework and the report will give an overview of the features and use of FLAME and discuss the implementation of techniques that attempt to exploit large parallel computing systems.  

This report updates the ealier deliverable on porting FLAME to parallel systems - Deliverable D1.4 - and presents the developments of FLAME and the porting of the Full Integrated EURACE Model during the final year of the project.

Some of the initial results from the initial benchmarking activities have been included in the Appendix so as to provide a complete record of the FLAME developments.

In the first report the results presented demonstrated that the parallel implementation of FLAME worked and is portable between a number of systems the efficiency is quite poor as the full implementation of message filtering is not yet complete and filtering is not yet fully exploited within the EURACE models. During the
final year significant developments have been made to FLAME and the underlying Message Board Library that significantly improved both its serial and parallel performance. The filtering available within FLAME has been enriched and its utilisation within the framework improved. 

The report presents the developments of FLAME and its associated analysis tools during the final period which including a model consistency check, a initial validation tool and various static and dynamic analysis tools to help the assessment of model performance.

The final section of the report describes the results achieved in benchmarking the EURACE Model at the time of writing. Although largist populations can be simulated - in the order of 30,000 agents - we have not been able to achieve the 100,000s that was the potential goal of the project.
\end{abstract}
\pagebreak
\pagenumbering{arabic}

% Start of document
%
% Fix page number

%
% General structure
%   1. Introduction
%   2. Characterisatics of Agent Based Simulators
%   3. General Parallel Implementation of FLAME.
%   4. Detailed Desciption of Parallelisation
%   5. Detail on Dynamic Load Balancing
%   6. Testing: Functional and Portability
%   7. Benchmark Problems
%   8. Future Development and Optimisation
%   9. Conclusion\section{Introduction}
\section{Introduction}	In this deliverable we present the internal logical consistency of the fully integrated EURACE model.
Since the focus of WP8 is on the development, integration and validation of the EURACE model, we found it appropriate to restrict D8.5 to these topics. This deliverable therefore does not contain any scenarios or policy experiments, which are collected in WP9.

Chapter 1 describes the construction of a system of national accounts for EURACE, with all the interlinkages between the balance sheets of the agents. We provide a set of accounting rules that should be satisfied for the model to be stock-flow consistent. We have verified that all these rules a true for the fully integrated model, thereby validating the internal logical consistency of all monetary and physical flows.

Chapter 2 on robustness checks shows that the model is robust against changes in critical model parameters and to scaling of the population size. We also show the effects of synchronous versus asynchronous timing of the agents.

Finally, in Chapter 3 we show benchmark results for the Integrated Model that provide the background for the more elaborate policy experiments presented in D9.1-D9.3. 
\section{General Parallel Implementation of FLAME} The FLAME architecture has some inherently good characteristics that lend itself to parallelisation. Unfortunately, it also has a number of bad characteristics. Because FLAME is an application generator
it does not have a full understanding of the application it is generating. Agent-based applications could be characterised as a set of communicating tasks. Although the agent population and their interactions can be specified \textit{a priori} the computational load of each agent and the number of communications they perform are very difficult to determine without running the code. 

We will not address the dynamic re-configuring of an agent population at run time. Initial experiments have shown that this is a very complex problem where there are many trade offs to be considered. In this effort to have an automatically generated parallel implementation of a FLAME application we focus on the most basic characteristic of FLAME and its agents: that of communications - agent to agent. This communication between agents is implemented within FLAME as a set of \textit{message boards} on which agents post messages (information) and from which agents can read the messages (information). There is one message board per message type and FLAME manages all the users interactions with the message boards through a Message Board API.

In a fully connected and communicating agent population, interaction may not be local but long range leading to many-to-many, inter-node communication which can drastically impact the scalability and simulation time. However, in many applications of FLAME, we have seen that there is sufficient locality (that can be taken advantage of) to consider parallelisation taking into account the general population sizes. 

The use of simple read/write, single-type message boards allows the framework implementer to divide the agent population and their associated communications areas. This division could be based on any number of parameters or separators but the simplest to appreciate is position or locality. If, as in EURACE, agents are people or companies for example, they will have locality defined either as location or by some group topology. It is also reasonable to assume that the dominant communications in both scenarios will be with neighbouring agents.

As explained above, FLAME uses a collection of message boards to facilitate inter-agent communication. As the majority of large high performance computing systems currently use a distributed memory model a Single Program Multiple Data (SPMD) paradigm is considered most appropriate for the FLAME architecture. The parallelisation of FLAME utilises partitioned agent populations and distributed message boards linked through MPI communication. Figure~\ref{fig:Figure2} shows the difference between the serial and parallel implementation.

\begin{figure}[h]
	\centering
		\includegraphics[scale=0.25]{flame.jpg}
	\caption{Serial and Parallel Message Boards}
	\label{fig:Figure2}
\end{figure}

The most significant operation in the parallel implementation is providing the message information required by agents on one node of the processor array but stored on a remote node of the processor. The FLAME Message Board Library manages these data requests by using a set of predefined message filters to limit the message movement. This process could be considered a synchronisation of the local message boards within an iteration of the simulation. This synchronisation essentially ensures that local agents have the message information they need as the simulation progresses.

An additional advantage of implementing parallelism in FLAME through the Message Board Library is that development of the FLAME framework and the message board algorithms can continue independently to a great extend as the Message Board API defines the interface between the two elements of the code. This should enable the message board routines to be developed and optimised without major re-engineering of the framework.

The two main areas of algorithmic and technical development needed to achieve an effective parallel implementation are load balancing and communications strategy. 

Initial load balancing is not too difficult: we have a population of agents, of various complexities, to which we can assign relative weights and so in the most general case the agents can be distributed over the available processors using the weights. This may well give an initial load balance but makes no reference to the possible communication patterns of the agent population. As the simulation develops the numbers of agents in the population may change and adversely affect the load balance of the processors. It is a very interesting and difficult problem to gauge whether the additional work (computation and communication) involved in remedying a load imbalance  is worth the gain. Given that the goal of any dynamic re-organising of the agents is to reduce the elapsed time of the overall simulation, determining whether a process of dynamically re-balancing the population will contribute to this is very problematic. It may well be that a slight load in-balance will have no significant effect of the wall clock time of the simulation. These problems are under investigation.

The patterns and volumes of communication for the population will have a considerable impact on the performance and parallel efficiency of the simulation. In general, agents are rather light-weight in terms of computational load. Where all agents can and do communicate with all others the communications load within and across processors will be great. Fortunately communications within a processor are generally efficient. However across processors this communication can dominate the application. Within FLAME communication between agents is managed by the Message Board Library, which uses MPI to communicate between processors. The Message Board Library implementation attempts to minimise this communication overhead by overlapping the computational load of the agents with the communication. 

Where the agents have some form of locality the initial distribution of agents makes use of this information in placing agents on processing nodes. During the simulation agents can be dynamically re-distributed to maintain computational load balance. However given the light-weight computational nature of many agent types the effect of dynamically re-distributing agents on the grounds of their communications load may well turn out to be more important than considering computational load.

Within the EURACE Project a parallel version of FLAME has been developed using these ideas and the sections below discuss some of the results in performing parallel simulations.


\section{Detailed Description of Parallelisation} \subsection{Overview}

Within a FLAME simulation, every agent only interacts with its environment via the reading and writing of messages to a collection of Message Boards. This makes the Message Board component the ideal candidate for enabling parallelism. With a distributable Message Boards, agents can be farmed out across multiple processing nodes and simulated in parallel, while a coherent simulation can be maintained through the unified view of the distributed Boards.

\begin{figure}[h]
 \centering
  \includegraphics[scale=0.50]{mboard_flame.jpg}
 \caption{Parallelisation of FLAME using distributed Message Boards}
 \label{fig:mb_flame}
\end{figure}

In the recent code release, the Message Board was decoupled from the FLAME framework and implemented as a separate library. This will provides us with the flexibility to experiment with different parallelisation strategies while minimising the impact on current users of the FLAME framework.

The Message Board Library (\textit{libmboard}) was designed as a static library that can be linked to the simulation binaries and accessed via the libmboard Application Program Interface (API). 

\textit{libmboard} uses MPI to communicate between processors, and POSIX threads (pthreads) to fork a separate thread for handling data management and inter-process communication. The use of threads enables the memory intensive operations of managing the Message Boards to be performed concurrently with the agent simulations. 

Apart from potentially making better use of multi-core processors, delegating \textit{libmboard} operations to a separate thread also allows us to minimise the overheads by overlapping the Board synchronisation time with useful computation.
\section{Initial Data Partitioning} As described above in general terms parallelisation in FLAME has been introduced through distributed message boards and distributed agents populations. Hence at the start of any simulation the agent population must be distributed over the available processors.

As achieving some form of load balance - each processor performing a similar work load - is important in reducing the elapsed time of a simulation, the initial distribution of the population should attempt to achieve this. However such an initial distribution can only be based on the information provided in the XMML models files and the associated user provided C code. In the current version of FLAME there is little useful information provided.

Although achieving a load balance over the processors is important in reducing elapsed time reducing inter-processor communication is equally if not more important in agent-based applications. Deriving information on the communications load of an agent population can only be achieved whilst the application is executing although some information can be derived from the XML and C code.

Two basic methods of static partitioning have been developed: partitioning based on a separator and \textsl{round robin} partitioning.
\section{Initial Data Generation} \subsection{Introduction on Cloning}

As the EURACE model became more and more complicated so did the relationships between agents. These realtionships are both inter- and instra- regional and provide a complex set of constraints for the initial population of agents. As modellers and computer scientists demanded larger populations it was found that the Tubitak population GUI could not provide populations quickly enough and ran into memory problems even on very well specified machines. For this reason a simpler method of creating large populations was designed, that of \textit{cloning} one population many times replacing agent ids as required.

A variety of implementations were tried, \texttt{bash} scripts using \texttt{sed/awk}, python code and finally and most successfully C code that could run in serial or in parallel. The seed population for cloning has one region and so there are intially no inter-regional interactions but it is possible for these to occur in the transient stage of the simulation if the model is set up correctly. The lack of these interactions means that cloning is a perfectly parallel algorithm.

Cloning from a single region also has the benefit that it is possible to provide pre-partitioned input data for the parallel implementation of the EURACE model. For example, clone one region to give 16 regions and input files for 1, 2, 4, 8 and 16 processes can be easily produced.

The cloning process relies on a special \texttt{xml} file in which any agent ids are marked with \texttt{REPLACE\_ID\_n} where \texttt{n} is the id of the agent in the region being cloned. This \texttt{xml} file is produced from a \texttt{pop} file by the \texttt{instantiate.py} script written by Tubitak and using code from the population GUI. The script is invoked as

\begin{verbatim}
python instantiate.py -r 0_bench_oct_12.pop 0_markers_oct_12.xml
\end{verbatim}

Actually cloning the region is illustrated by the following example:

\begin{verbatim}
./clone.sh 0_markers_oct_12.xml 8 tmp -r 2
\end{verbatim}

This clones \texttt{0\_markers\_oct\_12.xml} creating 2 input files each containing 8 regions. The output files are put into a directory called \texttt{16R\_2P} indicating that this is input for a 16 region run on 2 processes.

It is possible to control which agents are cloned by listing their names in the file \texttt{agent\_list.txt}.

\subsection{Implementation of Cloning}

The cloning application runs from the commandline using a \texttt{bash} script to control the cloning and organisation of output files, \texttt{sed} to perform some ad-hoc text manipulation of output files and a C application to do the computationaly intensive work of reading the input file and writing an output file. Both serial and parallel (MPI) versions of the C code are available.

The serial code works by repeatedly calling it with arguments that are the increment to make in id numbers between regions and a number (0 based) of this particular clone. The id increment is set to the number of agents in the original region since there can be no more agents in any cloned region. Running the code in paralle requires only the increment value to be given as each process can get the clone number as its node id by calling \texttt{MPI\_Comm\_rank}.

{\raggedright Source code is available from the EURACE Subversion repository \break \verb+http://ccpforge.cse.rl.ac.uk/svn/eurace/models/utils/cloning+.}

\section{Integration with ExGUI} The experiment manager module is highly coupled with policy analysis module. Policy experiments require comparison of outcomes of alternative simulations, where the compared cases correspond to co-variation of several parameters among their corresponding ranges. ExpGUI facilitates creation of a series of simulation tasks via a GUI, thus reducing errors in creation of policy experiments.

\begin{figure}[h]
  \centering
\screenshot{expgui}
  \caption{Experiment Manager}
  \label{figure:expgui}
\end{figure}

ExpGUI inherits an initial population file from PopGUI. It principally facilitates designer to run the designed model for desired number of iterations within a policy experimentation. Policy maker is able to test the designed model on following conditions:
\begin{itemize}
\item sensitivity to policy parameters
\item sensitivity to initial population
\item sensitivity to stochastic process through out iterations.
\end{itemize}

Figure~\ref{figure:expgui} is a snapshot of the experimet manager under Windows.
 



\section{Tools for FLAME and Model Assessment} \subsection{Static Analysis of Models}


EURACE has developed a very complex model in which there are many agents and many communications. The nested model directories contain 11 subdirectories and $\sim$50 \textit{xml} and C-code files. Checking the consistency of the model is a very difficult task. Although FLAME's parser will check the validity of $xml$ within the contex of the DDT of FLAME tags, checking that messages are used in a consistent way is difficult.


A number of static analysis tools have been developed to perform these types of tasks.


SVN: http://ccpforge.cse.rl.ac.uk/svn/eurace/models/utils/model_analysis


    * analyse_model.py - Python script to do a static analysis of a model.

    * check_message_consistency.py - Python script to check consistency between XMML code and C code. 


[edit] Features


A set of model consistency scripts to do a static analysis of a model.


    * Outputs the communication graph between agent types.

    * Check consistency of messages defined in XMML and in C code 


\subsection{Dynamic Message Monitoring}


text about sql data base implementation which can be use to verify the model.


\subsection{Dynamic Performance Analysis}


The timer package described in Appendix \ref{timer} has been used to measure elapsed CPU time for functions and message board synchronisations. Knowing which functions take the longest time has helped to narrow the application of more detailed profiling tools such as \texttt{gprof} allowing for quicker identification of problems and possible solutions. Analysis of message board synchronisation times has shown that the message board implementation has provided excellent overlap of communication and computation.


\subsection {FLAME Verification}


\subsection{Model Validation}
\section{Performance of the FLAME Framework} The parallel implementation of FLAME seeks to use an SPMD paradyn - each node of the parallel system is running the essentially same $program$. However given the nature of agent-bsaed modelling and the FLAME implementation each nodal program could perform a very different sequence of instructions and function activations as it traverses its part of the state space. 

The two fundamental design features of FLAME is that all communications between agents takes place through a specified message board - a message respository - and that these message boards are distributed over the processing nodes of the parallel system. These message boards can be considered the data load and the agents themselves the computational load. In FLAME both these are distributed over the computational nodes.

However, although there may be some imbalance in computational load, with a reasonable initial data - agents - distribution, any imbalance should be small. The crucial element in the parallel implementation of FLAME is the distribution of the message boards over the system and their syncronisation.

Use Times Library to measure the wait time for each message board to complete syncronisation.


 
%\section{Remote Job Submission} %job submission
\subsection{Introduction}

We have designed and implemented a job submission system for FLAME jobs so that it is easy to get runs going on remote (parallel) machines. The sequence of steps for job submission was drawn up in discussion with TUBITEK and is as follows:

\begin{enumerate}
	\item \textbf{Check authentication}. Does the user provided information allow log in to the target machine? Return code for success/failure.
    \item \textbf{Check FLAME version}. Is the required version of FLAME available on the target machine? If not copy files onto target and install. Return code for success/failure of installation or success if installed. Could be some output text to say when FLAME has to be installed.
    \item \textbf{Create a project}. Send the model XMML file and C code. Parse and compile the model. Return code for parse failure/compilation failure/success. Success return code is the project id.
    \item \textbf{Submit job}. Send the 0.xml file(s) and project id. Submit the job according to data in $<$machine$>$.conf. Return code code for success/failure. Success return code is the job id.
    \item \textbf{Query job status}. Send project and job id. Return code pending/done/running/failed.
    \item \textbf{Query status of all jobs in project}. Send project id. Not sure about the return code. Return text could be {job id, status} for each job.
    \item \textbf{Query status of all jobs in all projects}. Again not sure about return code. Return text could be {job id, status} for each job.
    \item \textbf{Get results}. Send project and job id. Copy results back and gather if parallel. Return code for success/failure. 
\end{enumerate}

The details for each of these steps are given later.

Connection to remote machines will be via \texttt{ssh} a standard secure connection mechanism which encrypts data between machines or \texttt{gsissh} a grid-enabled version of ssh (part of Globus \verb+http://www.globus.org+) which requires the user to have a grid (X.509) certificate. The scripts will work best if the user arranges for login authentication without a password. For ssh this means generating a public/private key pair (see the Authentication section of \texttt{ssh} manual and the \texttt{ssh-keygen} manual for details). The public key should be copied to the remote machine and then, using \texttt{ssh-agent} as shown below the operations can be carried out without further authentication.

\begin{verbatim}
# Get the environment variables for ssh-agent
ssh-agent > file
# Set the variables
. ./file
# Add the private key to this session. Will require pass phrase for ssh key
ssh-add
# Run the job submission you want. As an example I have put in a simple ssh
ssh user@remote.machine.ac.uk
# Kill the ssh-agent session
ssh-agent -k	
\end{verbatim}


For gsissh the process is different. The local Grid computing community will have details on obtaining and using a Grid certificate and possibly be able to give advice on installing enough of Globus to use gsissh. It is beyond the scope of this document to go further

\subsection{Authentication}

This will check whether the user and remote machine data given in the configuration file allow a log in to the remote machine. Comparing the hostname of the remote machine with that in the configuration file will indicate whether the log in was successful or not.

\subsection{Check FLAME}

Check for the xparser executable on the remote machine, assuming that its presence means that necessary libraries (such as the message board library) are therefore present. First look in the PATH environment variable and if not found then look in a known directory where a previous check may have installed the parser. If the parser is found then check the version against that version required by the user. If the version is correct the script finishes. 

If the parser is not found or the version is incorrect then the script copies the source for the parser and associated libraries to the remote machine and builds and installs them in a known directory.

\subsection{Create Project}

A project comprises the XMML file and C code for a model and then jobs are added to projects by giving the initial data, number of iterations and number of partitions for a FLAME run. The project is created by giving a directory where the XMML and C code files can be found and they are copied to the remote machine. The xparser is run and the resulting C code compiled. Errors from the parsing or compilation are reported where necessary and when the project has been successfully created it is given a project id that is returned to the user. 

\subsection{Submit Job}

The initial data file is copied to the remote machine and the run initiated for the user defined number of iterations and number of partitions. The job should be assigned to a particular project so the system knows what model is to be run.  Typically large parallel machines use some form of job scheduler to ensure users get a fair share of the machine and details of how to submit jobs to the scheduler should be provided by the user. These details go in the configuration file. When the job is scheduled on the remote machine the script returns a job id for the user to use later in queries.

It is possible to run jobs interactively, that is the script starts the job and waits until it is complete before returning.



\subsection{Query Job Status}

\subsection{Get Results}



%\section{Testing: Functional and Portability} \subsection{Unit testing of the message board API}
As mentioned above the message boards (are accessed by the FLAME framework via a Application Program Interface - the Message Board Library (the libmboard API). This provides the FLAME developer with a uniform interface to the functionality of the libmboard.
\subsection{Testing serial and parallel implementations}
It is important to ensure that application generated by the FLAME framework execute \textsl{correctly} in both their serial and parallel modes. Because of the stocastic nature of the agent-based approach to modelling it is unrealistic to expect complex simulations to following exactly the solution path although general trends should be similar. However for some simple applications we can expect to serial and parallel implementations to produce exactly the results throughout the simulation. Such example applications can be used to verify the correctness of both the serial and parallel implementations.

The \textsl{Circles Model} is one such application. The \textsl{Circles} agent is very simple. It has a position in two-dimensional space and a radius of influence. Each agent will react to its neighbours within its interaction radius repulsively. So given a sufficient simulation time the initial distribution of agents will tend to a field of uniformly spaced agents. Each agent has $x$, $y$, $fx$, $fy$ and $radius$ in its memory and has three states: outputdata, inputdata and move. The agents communicate via a single message board, $location$, which holds the agent $id$ and position. Given the simplicity of the agent it is possible to determine the final result of a number of ideal models.

A set of simple test models and problems have been developed based on the \textsl{Circles} agent. Each test has a \textsl{model.xmml} files and a set of initial data (\textsl{0.xml}).
\begin{description}
	\item [Test 1]: Model: single \textsl{Circles} agent type; Initial population of no agents. Expected result:
	\item [Test 2]: Model: single \textsl{Circles} agent type; Initial population of one agent at (0,0).
	\item [Test 3]: Model: Two \textsl{Circles} agent type; Initial population of agents at (-1,0) and (+,0).
	\item [Test 4]: Model: Four \textsl{Circles} agent type; Initial population of one agent at ($\pm$1,$\pm$1).
	\item [Test 5]: Model: Four \textsl{Circles} agent type; Initial population of one agent at (0,$\pm$1) and ($\pm$1,0).
	\item [Test 6]: Model: Four \textsl{Circles} agent type; Initial population of one agent at random positions.
	\end{description}
In each of these models the expected results can be specified and therefore they provide a very simple check of the implementation.

The \textsl{Circles} agent also provides a good mechanism to check the parallel implementation against the serial. Such is the nature of the model the positions on the agents at each iteration of the simulation is independent on the order of calculation. As the order of calculation can not be easily prescript in the parallel simulation we can use this characteristic to test the validity of the parallel implementation against the serial. We would expect to get the identical positions for each agent at very iteration of the simulation.


\section{Performance of EURACE Model} \begin{comment}
\documentclass{article}
\usepackage{epsfig,graphicx,verbatim, boxedminipage, url}
\begin{document}
\end{comment}

%\begin{comment}
%\subsubsection*{Growth rates}
\begin{figure}[H!]
\centering\leavevmode
\begin{minipage}{17cm}
\centering\leavevmode
\includegraphics[width=8cm]{./benchmark_plots/Eurostat-annual_growth_rates_monthly-gdp.png}
\includegraphics[width=8cm]{./benchmark_plots/Eurostat-annual_growth_rates_monthly-output.png}\\
\includegraphics[width=8cm]{./benchmark_plots/Eurostat-annual_growth_rates_monthly-unemployment_rate.png}
\includegraphics[width=8cm]{./benchmark_plots/Eurostat-annual_growth_rates_monthly-average_wage.png}
\end{minipage}
\caption{Annual growth rates (with respect to the same month the previous year) of GDP, total output, unemployment rate and average wage.}
\label{Figure: Eurostat macrodata growth rates}
\end{figure}
\clearpage
%\end{comment}

%\pagebreak
%\subsubsection*{Government}
\begin{figure}[H!]
\centering\leavevmode
\begin{minipage}{17cm}
\centering\leavevmode
\includegraphics[width=8cm]{./benchmark_plots/Government-monthly_tax_revenues.png}
\includegraphics[width=8cm]{./benchmark_plots/Government-monthly_benefit_payment.png}\\
%\includegraphics[width=8cm]{./benchmark_plots/Government-monthly_subsidy_payment.png}
%\includegraphics[width=8cm]{./benchmark_plots/Government-cumulated_deficit.png}
\includegraphics[width=8cm]{./benchmark_plots/Government-monthly_budget_balance.png}
\includegraphics[width=8cm]{./benchmark_plots/Government-total_bond_financing.png}
%\includegraphics[width=8cm]{./benchmark_plots/Government-total_money_financing.png}
\end{minipage}
\caption{Government finances.}
\label{Figure: Government}
\end{figure}
\clearpage

%\pagebreak
%\subsubsection*{Firms}
\begin{figure}[H!]
\centering\leavevmode
\begin{minipage}{17cm}
\centering\leavevmode
\includegraphics[width=8cm]{./benchmark_plots/Firm-output.png}
\includegraphics[width=8cm]{./benchmark_plots/Firm-cum_revenue.png}\\
\includegraphics[width=8cm]{./benchmark_plots/Firm-earnings.png}
\includegraphics[width=8cm]{./benchmark_plots/Firm-payment_account.png}\\
\includegraphics[width=8cm]{./benchmark_plots/Firm-no_employees.png}
\includegraphics[width=8cm]{./benchmark_plots/Firm-actual_cap_price.png}
%\includegraphics[width=8cm]{./benchmark_plots/Firm-bankruptcy_state.png}
\end{minipage}
\caption{Firm production data.}
\label{Figure: Firm Production}
\end{figure}

\begin{figure}[H!]
\centering\leavevmode
\begin{minipage}{17cm}
\centering\leavevmode
\includegraphics[width=8cm]{./benchmark_plots/Firm-total_units_capital_stock.png}
\includegraphics[width=8cm]{./benchmark_plots/Firm-capital_costs.png}
\end{minipage}
\caption{Firm capital stock. Left: units of capital stock, right: investments.}
\label{Figure: Firm capital stock}
\end{figure}

\begin{figure}[H!]
\centering\leavevmode
\begin{minipage}{17cm}
\centering\leavevmode
\includegraphics[width=8cm]{./benchmark_plots/Firm-cum_revenue-batches.png}
\includegraphics[width=8cm]{./benchmark_plots/Firm-earnings-batches.png}
\end{minipage}
\caption{Firm production data, all batch runs.}
\label{Figure: Firm Production batch}
\end{figure}
\clearpage


%\subsubsection*{Financial Data}
\begin{figure}[H!]
\centering\leavevmode
\begin{minipage}{17cm}
\centering\leavevmode
\includegraphics[width=8cm]{./benchmark_plots/Eurostat-total_assets.png}
\includegraphics[width=8cm]{./benchmark_plots/Eurostat-total_debt.png}\\
\includegraphics[width=8cm]{./benchmark_plots/Eurostat-total_equity.png}
\includegraphics[width=8cm]{./benchmark_plots/Eurostat-total_earnings.png}\\
\includegraphics[width=8cm]{./benchmark_plots/Eurostat-average_debt_earnings_ratio.png}
\includegraphics[width=8cm]{./benchmark_plots/Eurostat-average_debt_equity_ratio.png}
\end{minipage}
\caption{Firm financial data.}
\label{Figure: Firm Financial Data}
\end{figure}


%\pagebreak
%\subsubsection*{Labour Market}
\begin{figure}[H!]
\centering\leavevmode
\begin{minipage}{17cm}
\centering\leavevmode
\includegraphics[width=8cm]{./benchmark_plots/Eurostat-unemployment_rate.png}
\includegraphics[width=8cm]{./benchmark_plots/Eurostat-average_wage.png}\\
\includegraphics[width=8cm]{./benchmark_plots/Eurostat-unemployment_rate_skill_1.png}
\includegraphics[width=8cm]{./benchmark_plots/Eurostat-average_wage_skill_1.png}\\
\includegraphics[width=8cm]{./benchmark_plots/Eurostat-unemployment_rate_skill_5.png}
\includegraphics[width=8cm]{./benchmark_plots/Eurostat-average_wage_skill_5.png}
\end{minipage}
\caption{Labour market data.}
\label{Figure: Labour Market}
\end{figure}


\begin{figure}[H!]
\centering\leavevmode
\begin{minipage}{17cm}
\centering\leavevmode
\includegraphics[width=8cm]{./benchmark_plots/Eurostat-no_vacancies.png}
\includegraphics[width=8cm]{./benchmark_plots/Eurostat-labour_share_ratio.png}
%\includegraphics[width=8cm]{./benchmark_plots/Eurostat-average_s_skill.png}
\end{minipage}
\caption{Labour market data (cont).}
\label{Figure: Labour Market 2}
\end{figure}

\clearpage


%\subsubsection*{Consumption Market}
\begin{figure}[H!]
\centering\leavevmode
\begin{minipage}{17cm}
\centering\leavevmode
\includegraphics[width=8cm]{./benchmark_plots/Eurostat-monthly_output.png}
\includegraphics[width=8cm]{./benchmark_plots/Eurostat-monthly_planned_output.png}\\
\includegraphics[width=8cm]{./benchmark_plots/Eurostat-monthly_sold_quantity.png}
\includegraphics[width=8cm]{./benchmark_plots/Eurostat-monthly_revenue.png}\\
\includegraphics[width=8cm]{./benchmark_plots/Eurostat-firm_average_productivity.png}
\includegraphics[width=8cm]{./benchmark_plots/Eurostat-firm_average_productivity_progress.png}
\end{minipage}
\caption{Consumption goods market.}
\label{Figure: Consumption Market}
\end{figure}
\clearpage


%\pagebreak
%\subsubsection*{Credit market}
\begin{figure}[H!]
\centering\leavevmode
\begin{minipage}{17cm}
\centering\leavevmode
\includegraphics[width=8cm]{./benchmark_plots/Bank-cash.png}
\includegraphics[width=8cm]{./benchmark_plots/Bank-deposits.png}\\
\includegraphics[width=8cm]{./benchmark_plots/Bank-total_credit.png}
\includegraphics[width=8cm]{./benchmark_plots/Bank-equity.png}\\
\includegraphics[width=8cm]{./benchmark_plots/Bank-ecb_debt.png}
\includegraphics[width=8cm]{./benchmark_plots/Bank-total_dividends.png}
\end{minipage}
\caption{Credit market data.}
\label{Figure: Credit Market}
\end{figure}
%\clearpage

%\end{document}

\section{Conclusion} In this report we have described the parallel implementation of the FLAME framework and its assessment together with some benchmarking results using the EURACE Model. We have also demonstrated FLAMEs use in a number of EURACE related simulations in addition to complete EURACE model on populations ranging from a few hundereds of agents, through tens of thousands to, in one case, a million agents.  In some of these simulations the parallel implementation of FLAME has shown reasonable scalability and parallel efficiency but in other the results have been disappointing.

An important goal of the project has been to perform, in parallel, a large simulation using the EURACE Model. The project has achieve this to a degree: the model has been defined, important parameters have been values, a method of generating agent populations implemented and a parallel implementation of the EURACE model can be generated by FLAME. Using these steps serial and parallel simulations of the EURACE Model have been perform. In this process a detail assessment of the FLAME generated code, the serial and parallel implementations and the EURACE Model have been performed. Message counts, function times and sychronisation times are a few of the measures that have been used together with a detail static analysis of the model to identify the performance defficiencies in both the FLAME framework and the EURACE model.

All this analysis has lead to improvements in FLAME and the EURACE Model which in general have improved its computational performance. However the presence of substanial serial components in any model has resulted in very poor parallel scalability. It is well known that parallel speedup is limited by the serial faction of a code - this is Amdah's Law. The analyses performed on the EURACE Model have shown that the singleton agents - in particularly the Clearing House - have a significant impact of the parallel performance of the model.
These types of potential problem were understood - fine grained tasks - at the start of the project and the modeller took steps to avoid them. The Clearing House was thought necessary to the architecture of the EURACE Model and although different strategies were tested to reduce its effect there was little that could be achieved. The Clearing House and any other serial bottleneck will compromise the parallel performance of the application.

Although at the end of EURACE we have not achieved the \textsl{optimum} solution to these problems we have at least advanced the current state of the art in the parallel implementation of agent-based simulations in the context of the FLAME Framework.


\begin{thebibliography}{99}
\bibitem{Tesfatsion} Tesfatison (2006) ''Agent-based computational economics: a constructive approach to economic theory'' in Handbook for Computational Economics, Vol 2, Noth-Holland
\bibitem{Finin} Finin et al (1994) ''KQML as an Agent Communication Language'', The Proceedings of  the Third International Conference on Information and Knowledge Management
\bibitem{Gregory} Gregory et al (2001)  ''Computing Microbial Interactions and Communications in Real Life'', 4th International Conference on Information Processing in Cells and Tissues
\bibitem{Noble} Noble (2002)  ''Modeling the heart-from genes to cells to the whole organ'', Science
\bibitem{Coakley} Coakley  (2005) ''Formal Software Architecture for Agent-Based Modelling in Biology'', PhD Thesis, University of Sheffield
\bibitem{Walker-a} Walker et al (2004) ''Agent-based computational modeling of wounded epithelial cell monolayers'', IEEE Transactions in NanoBioscience
\bibitem{Walker-b} Walker et al (2004) ''The Epitheliome: Agent-Based Modelling Of The Social Behaviour Of Cells'', Biosystems
\bibitem{Pogson} Pogson  et al (2006) ''Formal Agent-Based Modelling of Intracellular Chemical Reactions'', to appear in Biosystems
\bibitem{Qwarnstrom} Qwarnstrom et al (2006) ''Predictive agent-based NFkB modelling - involvement of the actin cytoskeleton in pathway control'', Submitted
\bibitem{Jackson} Jackson et al (2004) ''Trial geometry gives polarity to ant foraging networks'', Nature
\bibitem{EURACE} EURACE (2006)  ''Agent-based software platform for European economic policy design with heterogeneous interacting agents'', EU IST Sixth Framework Programme.
\bibitem{Holcombe} Holcombe (1998) ''X-machines a basis for dynamic system specification'', Software Engineering Journal
\bibitem{Kefalas-a} Kefalas et al (2003) ''Communicating X-machines: From Theory to Practice'', Lecture Notes in Computer Science
\bibitem{Kefalas-b} Kefalas et al (2003) ''Simulation and verification of P systems through communicating X-machines'', Biosystems
\bibitem{Eleftherakis} Eleftherakis et al (2003) ''An agile formal development methodology'', Proceedings of the First South-East European Workshop on Formal Methods
\bibitem{Delli Gatti} Delli Gatti et al (2006), ''Emergent Macroeconomics An Agent-based Approach to Business Fluctuations'', submitted.
\end{thebibliography}
\newpage
\appendix \appendix
\section{Initial Benchmarks Results}
\section{}

% End of document
\end{document} 
