\subsection{Introduction}

As described above in general terms, parallelisation in FLAME has been introduced through distributed message boards and distributed agent populations. Hence at the start of any simulation the agent population must be distributed over the available processors.

As achieving some form of load balance - each processor performing a similar work load - is important in reducing the elapsed time of a simulation, the initial distribution of the population should attempt to achieve this. However such an initial distribution can only be based on the information provided in the XMML models files and the associated user provided C code. In the current version of FLAME there is little useful information provided.

Although achieving a load balance over the processors is important in reducing elapsed time, reducing inter-processor communication is equally if not more important in agent-based applications. Deriving information on the communications load of an agent population can only be achieved whilst the application is executing although some information can be derived from the XML and C code.

Two basic methods of static partitioning have been developed: partitioning based on a separator and \textsl{round robin} partitioning.

\subsection{Separator Partitioning}

Separator partitioning distributes the agents amongst the partitions based on one or more memory variables of the agent. Every agent must have these variables for this to work and the variables can be either discrete or continuous numerical values. The most obvious example is distributing agents using their position ($x,y, z$ co-ordinates) which gives a good initial distribution in cases where communication is between near neighbours. This is already implemented in FLAME due to the framework's initial field of application, biological systems, and is referred to as \textit{geometric partitioning}. 

Other examples of separators could be region or country id, but the overall aim is to get those agents that will generate a lot of communication with each other on to the same partition - something that is not always feasible.

\subsection{Round Robin Partitioning}

This is the simplest form of partitioning in which agents are distributed, one at a time to each partition in turn. No account is taken of behaviour during the simulation but this may be the only way of partitioning if the agents have no common memory variables. This type of partitioning is implemented in FLAME.

It is possible to extend this method by using the agent type as a discriminator. Agents of a particular type would be allocated to one partition (or set of partitions) on a round robin basis if it were known (or envisaged) that agents communicated with other agents of their own type more than any other type. (This could also be seen as a special form of separator partitioning.)

\subsection{Other Benefits}

As well as hoping to improve the load balance, partitioning of the initial data will mean that each node in the compute system will have to read only its data. In the current version of FLAME one node has to read all the data, decide on the partitioning and then distribute the data with which agents can be partitioned. Then all the nodes have to read all the initial data picking out only those agents that fall within their partition.


