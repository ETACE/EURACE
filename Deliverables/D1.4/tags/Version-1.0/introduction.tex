In this report we describe the parallel implementation of the FLAME Framework \cite{Coakley}. We cover some of the reasons for parallelisation and also give a detailed description of the approach being adopted. We also present some initial results from a set of benchmarks which hope to capture the computation and communications loads inherent in the EURACE application.

There are many agent-based modelling systems. A detailed survey of such programs, systems and frameworks is given by Mangina \cite{Mangina}. Many of these systems are based on Java as their implementation language. Although a good language for web-based and some communications applications it is not one often used in the area of high performance computing. Similarly there are relatively few agent systems that address the problem of scalable simulations.

Before considering the current parallelisation of FLAME it is worth considering the characteristics of a software system than make it worth taking the time and effort to parallelise. Some characteristics of when to parallelise:
\begin{itemize}
\item Code is practically incapable of running on one computer, memory requirements too great, run time too long
\item Code will be reused frequently - parallelisation is a large investment
\item Data structures are simple, calculations are local, easy to communicate and synchronize between processors
\end{itemize}
The converse should also be considered. When not to parallelise a code:
\begin{itemize}
\item Code will only be used once (or infrequently) - An efficient parallel code takes time to develop!
\item Current performance is acceptable and execution time is short
\item There will be frequent and significant code changes
\end{itemize}
It should also be remembered that some algorithms simply do not parallelise.

It is clear that developing a scalable agent-based framework will be difficult. As mentioned above there are few examples none of which attempt to utilise the power of high performance systems such the Cray XT4 or the IBM BlueGene or the multitude of Beowulf type systems being offered by vendors. Agent systems such as SAMAS \cite{SAMAS},JADE \cite {JADE}, SIMJADE \cite{SIMJADE}, MACE3J \cite{MACE3J} and SPADES \cite{SPADES} have been used to demonstrate scalable agent computing but these have been relatively small simulations.

The starting point of FLAME is different from these systems - high performance computing was thought to be essential and thus the implementation language is C.
