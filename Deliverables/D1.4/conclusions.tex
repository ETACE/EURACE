In this report we have described the parallel implementation of the FLAME framework and
demonstrated its use in a number of EURACE related simulations including one of the complete EURACE model. The new message board library abstracts away from the FLAME framework all interactions with the message boards and provide an appropriate API for the FLAME developer.

Implementing the message functions in this way will enable the developers to improve the  efficiency of the parallel implementation without effecting the interface to the FLAME framework.

The current implementation has been tested on a variety of models ranging form the simple \textsl{Circles} model to the current \textsl{complete} EURACE model. Some benchmarking of the FLAME application have been performed and the process of optimisation started.

Two areas that will require significant development have been noted:
\begin{description}
	\item [Message board optimisation:] We are aware that the are a variety of ways in which the current message board implementation can be improved. The first of these is a more considered approach to defining \textsl{Filters} and \textsl{Filter} variables in the FLAME models to represent and \textsl{locality} in the models and subsequently an optimisation of how these will be used within the FLAME framework.
 \item [Dynamic load balancing:] The nature of a model might well change during a simulation. The computational or communications load of an agent may change leading to an imbalance in process usage. Although a simulation may tolerate a certain level of imbalance there will be level at which the elapsed time of the simulation will deteriate. In these case dynamic re-balancing of the processor loads may help.
 \end{description}
Both of these activities are very much research areas but there is very little literature currently published.

Although at the end of EURACE we will not have achieve the \textsl{optimum} solution to these problems we hope to have least advanced the current state of the art.