In this report we described the parallel implementation of the FLAME Framework \cite{Coakley}. We will cover not only the reason for parallelisation but also a detailed description of the approach being adopted coupled with some initial results from a set of benchmarks which hope to capture the computation and communications loads inherent in the EURACE application.

Before considering the current parallelisation of FLAME it is worth considering the characteristics of a software system than make it worth taking the time and effort to parallelise. Some characteristics of when to parallelise:
\begin{itemize}
\item Code is practically incapable of running on one computer, memory requirements too great, run time too long
\item Code will be reused frequently - parallelisation is a large investment
\item Data structures are simple, calculations are local, easy to communicate and synchronize between processors
\end{itemize}
The converse should also be considered. When not to parallelise a code:
\begin{itemize}
\item Code will only be used once (or infrequently) - An efficient parallel code takes time to develop!
\item Current performance is acceptable and execution time is short
\item There will be frequent and significant code changes
\end{itemize}
It should also be remembered that some algorithms simply do not parallelise.

