\subsection{Message Board Library}

The Message Board Library is a very new addition to the project. It currently contains the core infrastructure and a very basic communication strategy that, while functional, requires additional tuning and optimisation. The next step in the development would therefore include further research and innovation geared towards a more efficient and scalable performance. 

Some of the activities that are in the pipeline are:
\begin{itemize}
\item Assist in the parsing of \texttt{<filter>} tags within the FLAME parser to ensure Filter Functions are used optimally.
\item Accelerate the adoption of \texttt{<filter>} tags in models so we can obtain benchmarks results that are more indicative of actual performance.
\item Analyse the  performance of the library, and optimise the communication routines.
\item Profile the memory usage of the library, and improve the memory management and utilisation.
\item Allow the tuning of performance to specific platforms by identifying and exposing relevant parameters.
\end{itemize} 

\subsection{Partitioning and Dynamic Load Balancing}

The present implementation includes a \textit{timer} that has already been used to time sections of the FLAME framework and agent functions. In addition to timing data load balancing requires knowledge of the communication patterns in a particular model and the next phase will be to design and implement a \textit{communication data collector}. With this in place and a working EURACE model (or at least working isolated market models) the performance of simulations can be assessed and a start can be made on designing strategies for balancing computation and communication with a resulting reduction in simulation time. The data collected will also help with partitioning the initial agent population when multiple compute nodes are used.

\subsection{Job Submission}

The next step for the job submission scripts is to integrate them with the GUI in discussion with TUBITEK and extend the range of machines and job schedulers on which the submission system has been tested. There may be refinements of the scripts and functionality as the whole EURACE system comes together. 
