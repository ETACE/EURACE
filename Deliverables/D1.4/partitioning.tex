As described above in general terms parallelisation in FLAME has been introduced through distributed message boards and distributed agents populations. Hence at the start of any simulation the agent population must be distributed over the available processors.

As achieving some form of load balance - each processor performing a similar work load - is important in reducing the elapsed time of a simulation, the initial distribution of the population should attempt to achieve this. However such an initial distribution can only be based on the information provided in the XMML models files and the associated user provided C code. In the current version of FLAME there is little useful information provided.

Although achieving a load balance over the processors is important in reducing elapsed time reducing inter-processor communication is equally if not more important in agent-based applications. Deriving information on the communications load of an agent population can only be achieved whilst the application is executing although some information can be derived from the XML and C code.

Two basic methods of static partitioning have been developed: partitioning based on a separator and \textsl{round robin} partitioning.