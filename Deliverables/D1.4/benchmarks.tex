\subsection{Approach to Benchmarking}
\subsection{The Circles Model}
The Circles agent is very simple. It has a position in two-dimensional space and a radius of influence. Each agent will react to its neighbours within its interaction radius repulsively. So given a sufficient simulation time the initial distribution of agents will tend to a field of uniformly spaced agents.

The description of the agent is given as a example of XMML in the sections above. Each agent has $x$, $y$, $fx$, $fy$ and $radius$ in its memory and has three states: outputdata, inputdata and move. The agents communicate via a single message board, $location$, which holds the agent $id$ and position.

The Circles problem is very simple but allows us an initial assessment of the performance of the parallelisation within FLAME. The simulation was started with a populations of $10^6$  agents and experiments performed using from 4 to 100 processors. The averaged results are shown in Table~\ref{tab:ExecutionTimesForCircles} and Figure~\ref{fig:Circles-graph}.
\subsection{The C@S Model}
The C@S model was the first economic model to be implemented in FLAME by the EURACE Project.  It is based on work detailed in Delli Gatti \textsl{et al.} \cite{Delli Gatti} where an economy is populated by a finite number of \textsl{firms}, \textsl{workers}/\textsl{consumers} and \textsl{banks}. The acronym C@S stands for \textsl{Complex Adaptive Trivial System}.

This provides an initial economic model for testing FLAME. The EURACE version of C@S contains models for consumption goods, labour services and credit services. The population is a mix of agents: \textsl{Malls}, \textsl{Firms} and \textsl{People}. Each of these has different states and communicates with other agents in the population through 9 message types.

As the agents in the C@S Model have some positional/location data and the communication is localised, the initial distribution of agents to processors, as in the Circles Model, can be based on location. This helps reduce cross-processor communication.

The initial population contained: 20000 firms, 100000 people and 4000 malls (124000 agents in total).