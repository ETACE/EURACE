\documentclass[a4paper,11pt]{article}
\usepackage[pdftex]{graphicx}
\usepackage{draftwatermark}
\SetWatermarkScale{5.0}


\vfuzz2pt % Don't report over-full v-boxes if over-edge is small
\hfuzz2pt % Don't report over-full h-boxes if over-edge is small

\setlength{\oddsidemargin}{0cm}
\setlength{\evensidemargin}{0cm}
\setlength{\textwidth}{16cm}
\setlength{\textheight}{24cm}
\setlength{\topmargin}{-1cm}

\begin{document}
\thispagestyle{empty}

% Common EURACE Title Page
% EURACE and FW6 logos
\vspace{\baselineskip}
\includegraphics[width=45mm]{EURACE-logo.png}		
\hfill
\includegraphics[width=45mm]{FW6-logo.png}

% Title
\begin{center}
Project no.\\
035086\\
Project acronym\\
{\bf EURACE}\\
Project title\\
{\bf An Agent-Based software platform for European economic policy design with heterogeneous interacting agents: new insights from a bottom up approach to economic modelling and simulation}\\
\end{center}

\vspace*{\baselineskip}\noindent
Instrument: STREP\\[\baselineskip]
Thematic Priority: IST FET PROACTIVE INITIATIVE ``SIMULATING EMERGENT PROPERTIES IN COMPLEX SYSTEMS\\

% Deliverable information: title & number
\vspace*{2\baselineskip}
\begin{center}
{\bf
Deliverable reference number and title\\
D1.4: Porting of agent models to parallel computers\\
}
Due date of deliverable:\\
31/08/2008\\
Actual submission date:\\
\end{center}


% Project Info
\vspace*{\baselineskip}\noindent
{Start date of project: September 1$^{st}$} 2006 \hfill {Duration: 36 months}\\

%Deliverable Info - partner
\vspace{2\baselineskip}\noindent
Organisation name of lead contractor for this deliverable\\
{\bf STFC Rutherford Appleton Laboratory - STFC}
\begin{flushright} 
Revision 1
\end{flushright}

\vspace{\baselineskip}
\begin{table}[hb]
\begin{tabular}{||c|l|l||} \hline\hline
\multicolumn{3}{||l||}{\small Project co-funded by the European Commission within the Sixth Framework Programme (2002-2006)}\\ \hline
\multicolumn{3}{||l||}{\bf Dissemination Level}\\ \hline
\bf PU &\small Public\hfill~& \bf X \\ \hline
\bf PP &\small Restricted to other programme participants (including the Commission Services)& \\ \hline 
\bf RE &\small Restricted to a group specified by the consortium (including the Commission Services)&  \\ \hline
\bf CO &\small Confidential, only for members of the consortium (including the Commission Services)&  \\ \hline \hline
\end{tabular}
\end{table}
\pagebreak
% End of EURACE Title Page

\pagenumbering{roman}
% Table of Contents
\tableofcontents
\pagebreak

% Abstract/SUmmary
\begin{abstract}\noindent
Making use of high performance computers in agent-based simulation is a complex and difficult task. This report describes the approach being explored within the EURACE project to exploit parallel computing technology in large-scale agent-based simulations involving millions of agents. The underlying software is the FLAME framework and the report will give an overview of the features and use of FLAME and discuss the implementation of techniques that attempt to exploit large parallel computing systems.  

Some early simulation results will be presented together with a discussion of the requirements for efficient parallel simulations and the performance of the current implementation. Although these results demonstrate that the parallel implementation works and is portable between a number of systems the efficiency is quite poor as the full implementation of message filtering is not yet complete and filtering is not yet fully exploited within the EURACE models.

Finally the report also gives an indication of the work planned for the final year of the project which is focused around improving the parallel performance of the basic implementation. The optimisation and effective use of the message filtering will be an important element of this work.
\end{abstract}
\pagebreak
\pagenumbering{arabic}

% Start of document
%
% Fix page number

%
% General structure
%   1. Introduction
%   2. Characterisatics of Agent Based Simulators
%   3. General Parallel Implementation of FLAME.
%   4. Detailed Desciption of Parallelisation
%   5. Detail on Dynamic Load Balancing
%   6. Testing: Functional and Portability
%   7. Benchmark Problems
%   8. Future Development and Optimisation
%   9. Conclusion\section{Introduction}
\section{Introduction}	In this deliverable we present the internal logical consistency of the fully integrated EURACE model.
Since the focus of WP8 is on the development, integration and validation of the EURACE model, we found it appropriate to restrict D8.5 to these topics. This deliverable therefore does not contain any scenarios or policy experiments, which are collected in WP9.

Chapter 1 describes the construction of a system of national accounts for EURACE, with all the interlinkages between the balance sheets of the agents. We provide a set of accounting rules that should be satisfied for the model to be stock-flow consistent. We have verified that all these rules a true for the fully integrated model, thereby validating the internal logical consistency of all monetary and physical flows.

Chapter 2 on robustness checks shows that the model is robust against changes in critical model parameters and to scaling of the population size. We also show the effects of synchronous versus asynchronous timing of the agents.

Finally, in Chapter 3 we show benchmark results for the Integrated Model that provide the background for the more elaborate policy experiments presented in D9.1-D9.3. 
\section{General Parallel Implementation of FLAME} The FLAME architecture has some inherently good characteristics that lend itself to parallelisation. Unfortunately, it also has a number of bad characteristics. Because FLAME is an application generator
it does not have a full understanding of the application it is generating. Agent-based applications could be characterised as a set of communicating tasks. Although the agent population and their interactions can be specified \textit{a priori} the computational load of each agent and the number of communications they perform are very difficult to determine without running the code. 

We will not address the dynamic re-configuring of an agent population at run time. Initial experiments have shown that this is a very complex problem where there are many trade offs to be considered. In this effort to have an automatically generated parallel implementation of a FLAME application we focus on the most basic characteristic of FLAME and its agents: that of communications - agent to agent. This communication between agents is implemented within FLAME as a set of \textit{message boards} on which agents post messages (information) and from which agents can read the messages (information). There is one message board per message type and FLAME manages all the users interactions with the message boards through a Message Board API.

In a fully connected and communicating agent population, interaction may not be local but long range leading to many-to-many, inter-node communication which can drastically impact the scalability and simulation time. However, in many applications of FLAME, we have seen that there is sufficient locality (that can be taken advantage of) to consider parallelisation taking into account the general population sizes. 

The use of simple read/write, single-type message boards allows the framework implementer to divide the agent population and their associated communications areas. This division could be based on any number of parameters or separators but the simplest to appreciate is position or locality. If, as in EURACE, agents are people or companies for example, they will have locality defined either as location or by some group topology. It is also reasonable to assume that the dominant communications in both scenarios will be with neighbouring agents.

As explained above, FLAME uses a collection of message boards to facilitate inter-agent communication. As the majority of large high performance computing systems currently use a distributed memory model a Single Program Multiple Data (SPMD) paradigm is considered most appropriate for the FLAME architecture. The parallelisation of FLAME utilises partitioned agent populations and distributed message boards linked through MPI communication. Figure~\ref{fig:Figure2} shows the difference between the serial and parallel implementation.

\begin{figure}[h]
	\centering
		\includegraphics[scale=0.25]{flame.jpg}
	\caption{Serial and Parallel Message Boards}
	\label{fig:Figure2}
\end{figure}

The most significant operation in the parallel implementation is providing the message information required by agents on one node of the processor array but stored on a remote node of the processor. The FLAME Message Board Library manages these data requests by using a set of predefined message filters to limit the message movement. This process could be considered a synchronisation of the local message boards within an iteration of the simulation. This synchronisation essentially ensures that local agents have the message information they need as the simulation progresses.

An additional advantage of implementing parallelism in FLAME through the Message Board Library is that development of the FLAME framework and the message board algorithms can continue independently to a great extend as the Message Board API defines the interface between the two elements of the code. This should enable the message board routines to be developed and optimised without major re-engineering of the framework.

The two main areas of algorithmic and technical development needed to achieve an effective parallel implementation are load balancing and communications strategy. 

Initial load balancing is not too difficult: we have a population of agents, of various complexities, to which we can assign relative weights and so in the most general case the agents can be distributed over the available processors using the weights. This may well give an initial load balance but makes no reference to the possible communication patterns of the agent population. As the simulation develops the numbers of agents in the population may change and adversely affect the load balance of the processors. It is a very interesting and difficult problem to gauge whether the additional work (computation and communication) involved in remedying a load imbalance  is worth the gain. Given that the goal of any dynamic re-organising of the agents is to reduce the elapsed time of the overall simulation, determining whether a process of dynamically re-balancing the population will contribute to this is very problematic. It may well be that a slight load in-balance will have no significant effect of the wall clock time of the simulation. These problems are under investigation.

The patterns and volumes of communication for the population will have a considerable impact on the performance and parallel efficiency of the simulation. In general, agents are rather light-weight in terms of computational load. Where all agents can and do communicate with all others the communications load within and across processors will be great. Fortunately communications within a processor are generally efficient. However across processors this communication can dominate the application. Within FLAME communication between agents is managed by the Message Board Library, which uses MPI to communicate between processors. The Message Board Library implementation attempts to minimise this communication overhead by overlapping the computational load of the agents with the communication. 

Where the agents have some form of locality the initial distribution of agents makes use of this information in placing agents on processing nodes. During the simulation agents can be dynamically re-distributed to maintain computational load balance. However given the light-weight computational nature of many agent types the effect of dynamically re-distributing agents on the grounds of their communications load may well turn out to be more important than considering computational load.

Within the EURACE Project a parallel version of FLAME has been developed using these ideas and the sections below discuss some of the results in performing parallel simulations.


\section{Detailed Description of Parallelisation} \subsection{Overview}

Within a FLAME simulation, every agent only interacts with its environment via the reading and writing of messages to a collection of Message Boards. This makes the Message Board component the ideal candidate for enabling parallelism. With a distributable Message Boards, agents can be farmed out across multiple processing nodes and simulated in parallel, while a coherent simulation can be maintained through the unified view of the distributed Boards.

\begin{figure}[h]
 \centering
  \includegraphics[scale=0.50]{mboard_flame.jpg}
 \caption{Parallelisation of FLAME using distributed Message Boards}
 \label{fig:mb_flame}
\end{figure}

In the recent code release, the Message Board was decoupled from the FLAME framework and implemented as a separate library. This will provides us with the flexibility to experiment with different parallelisation strategies while minimising the impact on current users of the FLAME framework.

The Message Board Library (\textit{libmboard}) was designed as a static library that can be linked to the simulation binaries and accessed via the libmboard Application Program Interface (API). 

\textit{libmboard} uses MPI to communicate between processors, and POSIX threads (pthreads) to fork a separate thread for handling data management and inter-process communication. The use of threads enables the memory intensive operations of managing the Message Boards to be performed concurrently with the agent simulations. 

Apart from potentially making better use of multi-core processors, delegating \textit{libmboard} operations to a separate thread also allows us to minimise the overheads by overlapping the Board synchronisation time with useful computation.
\section{Initial Data Partitioning} As described above in general terms parallelisation in FLAME has been introduced through distributed message boards and distributed agents populations. Hence at the start of any simulation the agent population must be distributed over the available processors.

As achieving some form of load balance - each processor performing a similar work load - is important in reducing the elapsed time of a simulation, the initial distribution of the population should attempt to achieve this. However such an initial distribution can only be based on the information provided in the XMML models files and the associated user provided C code. In the current version of FLAME there is little useful information provided.

Although achieving a load balance over the processors is important in reducing elapsed time reducing inter-processor communication is equally if not more important in agent-based applications. Deriving information on the communications load of an agent population can only be achieved whilst the application is executing although some information can be derived from the XML and C code.

Two basic methods of static partitioning have been developed: partitioning based on a separator and \textsl{round robin} partitioning.
\section{Detail on Dynamic Load Balancing} %load balancing
\subsection{Dynamic Load Balancing Overview}

The overall aim of parallelising the FLAME framework is to reduce the wall clock time for running a simulation. This relies on efficient parallelisation of communication and keeping the work load balanced between computing nodes. The message board library addresses the first of these and dynamic load balancing addresses the second.

To illustrate the problems in getting load balancing right, the diagram in Figure \ref{fig:load_balance_problem} shows agents on two nodes and their communication patterns. The top portion shows an unbalanced number of agents but the frequent communication is internal to each node with only occasional communication between the nodes. If agents are moved in an attempt to balance the load then frequent communication between nodes is introduced (lower portion of figure) which could mean a large increase in communication time and hence wall clock time. This example shows that measurement of communication between nodes must be part of the load balancing algorithm as well as elapsed time for various parts of the framework.

\begin{figure}[h]
 \centering
  \includegraphics[scale=0.50]{load_balance.jpg}
 \caption{Illustrating some problems of load balancing}
 \label{fig:load_balance_problem}
\end{figure}

The dynamic load balancing library will therefore have to track data from two sources:

\begin{itemize}
 \item elapsed time for various parts of a FLAME run,
 \item data communication pattern and volume between nodes.
\end{itemize}

The timing data will show problems with computational load balance, where some nodes are idle and others working; and the communication data will show imbalance in the message flow between nodes, where some nodes send a lot of data and others very little. Timing data will also show whether it is the computation by the agents or time for communication that dominates the simulation. Attention can then be focused on improving load balance or improving communication balance as required.

An iteration of the model in FLAME proceeds in stages as agents move from one state to the next, therefore it will be necessary to look at data from each stage as well as the overall data for one iteration. If there is always an imbalance in load or communication for the same subset of nodes for all stages then it is easy to say that redistributing agents is necessary. However if the imbalance shows up for different nodes at different stages then the strategy for load balancing is more difficult to decide upon. 

Agents within FLAME are allowed to have dynamic memory variables (i.e.\ their size is not determined when the model is compiled but during the simulation) and so writing code to pack an agent's memory ready for transfer to another node is very difficult and execution of that code will be a large overhead in the execution of the model. As a consequence it is not envisaged that agents will be moved at every stage of an iteration. The dynamic load balancing framework will monitor the imbalances and will only decide to move agents if an imbalance is ``too great''. Experiments with the EURACE models will help determine what is meant by ``too great'' and what strategies can be adopted when seeking to achieve the best distribution of agents.

Since agents do not communicate directly with other agents but via message boards, the communication data will be the volume of message data sent from one node to another. This does not allow for identification of individual agents that may be causing problems but, by analysing the messages that are sent by a certain type of agent at the time of the imbalance it may be possible to identify a set of agents of a certain type whose redistribution could help.

Implementation has started with a timer which allows developers to insert timing into any section of FLAME, from the framework itself to the user's model functions. The details of the implementation are described in the next section.

\subsection{The \textit{timer} Package}

Timers will be used to measure the elapsed CPU time for portions of the running code and this data will be used as input to the load balancing strategy used in the FLAME framework. The initial requirements against which the timer package was implemented are given below.

\begin{itemize}
\item Can have multiple timers running simultaneously
\item Timers can be identified individually
\item Functions to start/stop/reset a named timer
\item Function to get elapsed time from a named timer
\item Definition of a set of timers.
\item Functions to get statistics from a set of timer
\item Turn timing on/off during program execution 
\end{itemize}

We have implemented all the functionality for individual timers and a set of unit tests and an example program using the timers. Code is stored under Subversion source code control in the FLAME project on the CCPForge site. 

User documentation is supplied in \cite{TimerAPI}.

\subsection{Timing Results}

We have demonstrated the use of timers in the simple circles model by timing the work done by agents as the number of partitions over which the agents are distributed increases. The agents were not uniformly distributed in space and so, with geometric partitioning, some partitions will have more agents than others. The time taken by the agents on each node is plotted against the number of agents on the node in Figure \ref{fig:circle_timings} and we can see that there is a direct relation between the number of agents and the work done on each node. From this we can conclude that distributing the agents equally over the nodes will give a good load balance.

There is a cautionary note to be struck however. The timing data for circles has illustrated the problem identified in the overview, namely that naively giving equal numbers of agents to each node without taking into account communication can lead to worse performance. Comparing elapsed time for geometric and round robin for 5 partitions in Figure \ref{fig:timings_problems} shows that the elapsed time increased even though the work done by the agents decreased.

\begin{figure}[h]
 \centering
  \includegraphics[scale=0.75]{circles-timings.png}
 \caption{Timings results from Circles model}
 \label{fig:circle_timings}
\end{figure}

\begin{figure}[h]
  \hspace{-10mm}
  \includegraphics[scale=0.6]{timings-problems.png}
 \caption{Illustration of problems with load balancing with Circles model}
 \label{fig:timings_problems}
\end{figure}


\section{Remote Job Submission} %job submission
\subsection{Introduction}

We have designed and implemented a job submission system for FLAME jobs so that it is easy to get runs going on remote (parallel) machines. The sequence of steps for job submission was drawn up in discussion with TUBITEK and is as follows:

\begin{enumerate}
	\item \textbf{Check authentication}. Does the user provided information allow log in to the target machine? Return code for success/failure.
    \item \textbf{Check FLAME version}. Is the required version of FLAME available on the target machine? If not copy files onto target and install. Return code for success/failure of installation or success if installed. Could be some output text to say when FLAME has to be installed.
    \item \textbf{Create a project}. Send the model XMML file and C code. Parse and compile the model. Return code for parse failure/compilation failure/success. Success return code is the project id.
    \item \textbf{Submit job}. Send the 0.xml file(s) and project id. Submit the job according to data in $<$machine$>$.conf. Return code code for success/failure. Success return code is the job id.
    \item \textbf{Query job status}. Send project and job id. Return code pending/done/running/failed.
    \item \textbf{Query status of all jobs in project}. Send project id. Not sure about the return code. Return text could be {job id, status} for each job.
    \item \textbf{Query status of all jobs in all projects}. Again not sure about return code. Return text could be {job id, status} for each job.
    \item \textbf{Get results}. Send project and job id. Copy results back and gather if parallel. Return code for success/failure. 
\end{enumerate}

The details for each of these steps are given later.

Connection to remote machines will be via \texttt{ssh} a standard secure connection mechanism which encrypts data between machines or \texttt{gsissh} a grid-enabled version of ssh (part of Globus \verb+http://www.globus.org+) which requires the user to have a grid (X.509) certificate. The scripts will work best if the user arranges for login authentication without a password. For ssh this means generating a public/private key pair (see the Authentication section of \texttt{ssh} manual and the \texttt{ssh-keygen} manual for details). The public key should be copied to the remote machine and then, using \texttt{ssh-agent} as shown below the operations can be carried out without further authentication.

\begin{verbatim}
# Get the environment variables for ssh-agent
ssh-agent > file
# Set the variables
. ./file
# Add the private key to this session. Will require pass phrase for ssh key
ssh-add
# Run the job submission you want. As an example I have put in a simple ssh
ssh user@remote.machine.ac.uk
# Kill the ssh-agent session
ssh-agent -k	
\end{verbatim}


For gsissh the process is different. The local Grid computing community will have details on obtaining and using a Grid certificate and possibly be able to give advice on installing enough of Globus to use gsissh. It is beyond the scope of this document to go further

\subsection{Authentication}

This will check whether the user and remote machine data given in the configuration file allow a log in to the remote machine. Comparing the hostname of the remote machine with that in the configuration file will indicate whether the log in was successful or not.

\subsection{Check FLAME}

Check for the xparser executable on the remote machine, assuming that its presence means that necessary libraries (such as the message board library) are therefore present. First look in the PATH environment variable and if not found then look in a known directory where a previous check may have installed the parser. If the parser is found then check the version against that version required by the user. If the version is correct the script finishes. 

If the parser is not found or the version is incorrect then the script copies the source for the parser and associated libraries to the remote machine and builds and installs them in a known directory.

\subsection{Create Project}

A project comprises the XMML file and C code for a model and then jobs are added to projects by giving the initial data, number of iterations and number of partitions for a FLAME run. The project is created by giving a directory where the XMML and C code files can be found and they are copied to the remote machine. The xparser is run and the resulting C code compiled. Errors from the parsing or compilation are reported where necessary and when the project has been successfully created it is given a project id that is returned to the user. 

\subsection{Submit Job}

The initial data file is copied to the remote machine and the run initiated for the user defined number of iterations and number of partitions. The job should be assigned to a particular project so the system knows what model is to be run.  Typically large parallel machines use some form of job scheduler to ensure users get a fair share of the machine and details of how to submit jobs to the scheduler should be provided by the user. These details go in the configuration file. When the job is scheduled on the remote machine the script returns a job id for the user to use later in queries.

It is possible to run jobs interactively, that is the script starts the job and waits until it is complete before returning.



\subsection{Query Job Status}

\subsection{Get Results}



\section{Testing: Functional and Portability} \subsection{Unit testing of the message board API}
As mentioned above the message boards (are accessed by the FLAME framework via a Application Program Interface - the Message Board Library (the libmboard API). This provides the FLAME developer with a uniform interface to the functionality of the libmboard.
\subsection{Testing serial and parallel implementations}
It is important to ensure that application generated by the FLAME framework execute \textsl{correctly} in both their serial and parallel modes. Because of the stocastic nature of the agent-based approach to modelling it is unrealistic to expect complex simulations to following exactly the solution path although general trends should be similar. However for some simple applications we can expect to serial and parallel implementations to produce exactly the results throughout the simulation. Such example applications can be used to verify the correctness of both the serial and parallel implementations.

The \textsl{Circles Model} is one such application. The \textsl{Circles} agent is very simple. It has a position in two-dimensional space and a radius of influence. Each agent will react to its neighbours within its interaction radius repulsively. So given a sufficient simulation time the initial distribution of agents will tend to a field of uniformly spaced agents. Each agent has $x$, $y$, $fx$, $fy$ and $radius$ in its memory and has three states: outputdata, inputdata and move. The agents communicate via a single message board, $location$, which holds the agent $id$ and position. Given the simplicity of the agent it is possible to determine the final result of a number of ideal models.

A set of simple test models and problems have been developed based on the \textsl{Circles} agent. Each test has a \textsl{model.xmml} files and a set of initial data (\textsl{0.xml}).
\begin{description}
	\item [Test 1]: Model: single \textsl{Circles} agent type; Initial population of no agents. Expected result:
	\item [Test 2]: Model: single \textsl{Circles} agent type; Initial population of one agent at (0,0).
	\item [Test 3]: Model: Two \textsl{Circles} agent type; Initial population of agents at (-1,0) and (+,0).
	\item [Test 4]: Model: Four \textsl{Circles} agent type; Initial population of one agent at ($\pm$1,$\pm$1).
	\item [Test 5]: Model: Four \textsl{Circles} agent type; Initial population of one agent at (0,$\pm$1) and ($\pm$1,0).
	\item [Test 6]: Model: Four \textsl{Circles} agent type; Initial population of one agent at random positions.
	\end{description}
In each of these models the expected results can be specified and therefore they provide a very simple check of the implementation.

The \textsl{Circles} agent also provides a good mechanism to check the parallel implementation against the serial. Such is the nature of the model the positions on the agents at each iteration of the simulation is independent on the order of calculation. As the order of calculation can not be easily prescript in the parallel simulation we can use this characteristic to test the validity of the parallel implementation against the serial. We would expect to get the identical positions for each agent at very iteration of the simulation.


\section{Benchmark Problems} \begin{comment}
\documentclass{article}
\usepackage{epsfig,graphicx,verbatim, boxedminipage, url}
\begin{document}
\end{comment}

%\begin{comment}
%\subsubsection*{Growth rates}
\begin{figure}[H!]
\centering\leavevmode
\begin{minipage}{17cm}
\centering\leavevmode
\includegraphics[width=8cm]{./benchmark_plots/Eurostat-annual_growth_rates_monthly-gdp.png}
\includegraphics[width=8cm]{./benchmark_plots/Eurostat-annual_growth_rates_monthly-output.png}\\
\includegraphics[width=8cm]{./benchmark_plots/Eurostat-annual_growth_rates_monthly-unemployment_rate.png}
\includegraphics[width=8cm]{./benchmark_plots/Eurostat-annual_growth_rates_monthly-average_wage.png}
\end{minipage}
\caption{Annual growth rates (with respect to the same month the previous year) of GDP, total output, unemployment rate and average wage.}
\label{Figure: Eurostat macrodata growth rates}
\end{figure}
\clearpage
%\end{comment}

%\pagebreak
%\subsubsection*{Government}
\begin{figure}[H!]
\centering\leavevmode
\begin{minipage}{17cm}
\centering\leavevmode
\includegraphics[width=8cm]{./benchmark_plots/Government-monthly_tax_revenues.png}
\includegraphics[width=8cm]{./benchmark_plots/Government-monthly_benefit_payment.png}\\
%\includegraphics[width=8cm]{./benchmark_plots/Government-monthly_subsidy_payment.png}
%\includegraphics[width=8cm]{./benchmark_plots/Government-cumulated_deficit.png}
\includegraphics[width=8cm]{./benchmark_plots/Government-monthly_budget_balance.png}
\includegraphics[width=8cm]{./benchmark_plots/Government-total_bond_financing.png}
%\includegraphics[width=8cm]{./benchmark_plots/Government-total_money_financing.png}
\end{minipage}
\caption{Government finances.}
\label{Figure: Government}
\end{figure}
\clearpage

%\pagebreak
%\subsubsection*{Firms}
\begin{figure}[H!]
\centering\leavevmode
\begin{minipage}{17cm}
\centering\leavevmode
\includegraphics[width=8cm]{./benchmark_plots/Firm-output.png}
\includegraphics[width=8cm]{./benchmark_plots/Firm-cum_revenue.png}\\
\includegraphics[width=8cm]{./benchmark_plots/Firm-earnings.png}
\includegraphics[width=8cm]{./benchmark_plots/Firm-payment_account.png}\\
\includegraphics[width=8cm]{./benchmark_plots/Firm-no_employees.png}
\includegraphics[width=8cm]{./benchmark_plots/Firm-actual_cap_price.png}
%\includegraphics[width=8cm]{./benchmark_plots/Firm-bankruptcy_state.png}
\end{minipage}
\caption{Firm production data.}
\label{Figure: Firm Production}
\end{figure}

\begin{figure}[H!]
\centering\leavevmode
\begin{minipage}{17cm}
\centering\leavevmode
\includegraphics[width=8cm]{./benchmark_plots/Firm-total_units_capital_stock.png}
\includegraphics[width=8cm]{./benchmark_plots/Firm-capital_costs.png}
\end{minipage}
\caption{Firm capital stock. Left: units of capital stock, right: investments.}
\label{Figure: Firm capital stock}
\end{figure}

\begin{figure}[H!]
\centering\leavevmode
\begin{minipage}{17cm}
\centering\leavevmode
\includegraphics[width=8cm]{./benchmark_plots/Firm-cum_revenue-batches.png}
\includegraphics[width=8cm]{./benchmark_plots/Firm-earnings-batches.png}
\end{minipage}
\caption{Firm production data, all batch runs.}
\label{Figure: Firm Production batch}
\end{figure}
\clearpage


%\subsubsection*{Financial Data}
\begin{figure}[H!]
\centering\leavevmode
\begin{minipage}{17cm}
\centering\leavevmode
\includegraphics[width=8cm]{./benchmark_plots/Eurostat-total_assets.png}
\includegraphics[width=8cm]{./benchmark_plots/Eurostat-total_debt.png}\\
\includegraphics[width=8cm]{./benchmark_plots/Eurostat-total_equity.png}
\includegraphics[width=8cm]{./benchmark_plots/Eurostat-total_earnings.png}\\
\includegraphics[width=8cm]{./benchmark_plots/Eurostat-average_debt_earnings_ratio.png}
\includegraphics[width=8cm]{./benchmark_plots/Eurostat-average_debt_equity_ratio.png}
\end{minipage}
\caption{Firm financial data.}
\label{Figure: Firm Financial Data}
\end{figure}


%\pagebreak
%\subsubsection*{Labour Market}
\begin{figure}[H!]
\centering\leavevmode
\begin{minipage}{17cm}
\centering\leavevmode
\includegraphics[width=8cm]{./benchmark_plots/Eurostat-unemployment_rate.png}
\includegraphics[width=8cm]{./benchmark_plots/Eurostat-average_wage.png}\\
\includegraphics[width=8cm]{./benchmark_plots/Eurostat-unemployment_rate_skill_1.png}
\includegraphics[width=8cm]{./benchmark_plots/Eurostat-average_wage_skill_1.png}\\
\includegraphics[width=8cm]{./benchmark_plots/Eurostat-unemployment_rate_skill_5.png}
\includegraphics[width=8cm]{./benchmark_plots/Eurostat-average_wage_skill_5.png}
\end{minipage}
\caption{Labour market data.}
\label{Figure: Labour Market}
\end{figure}


\begin{figure}[H!]
\centering\leavevmode
\begin{minipage}{17cm}
\centering\leavevmode
\includegraphics[width=8cm]{./benchmark_plots/Eurostat-no_vacancies.png}
\includegraphics[width=8cm]{./benchmark_plots/Eurostat-labour_share_ratio.png}
%\includegraphics[width=8cm]{./benchmark_plots/Eurostat-average_s_skill.png}
\end{minipage}
\caption{Labour market data (cont).}
\label{Figure: Labour Market 2}
\end{figure}

\clearpage


%\subsubsection*{Consumption Market}
\begin{figure}[H!]
\centering\leavevmode
\begin{minipage}{17cm}
\centering\leavevmode
\includegraphics[width=8cm]{./benchmark_plots/Eurostat-monthly_output.png}
\includegraphics[width=8cm]{./benchmark_plots/Eurostat-monthly_planned_output.png}\\
\includegraphics[width=8cm]{./benchmark_plots/Eurostat-monthly_sold_quantity.png}
\includegraphics[width=8cm]{./benchmark_plots/Eurostat-monthly_revenue.png}\\
\includegraphics[width=8cm]{./benchmark_plots/Eurostat-firm_average_productivity.png}
\includegraphics[width=8cm]{./benchmark_plots/Eurostat-firm_average_productivity_progress.png}
\end{minipage}
\caption{Consumption goods market.}
\label{Figure: Consumption Market}
\end{figure}
\clearpage


%\pagebreak
%\subsubsection*{Credit market}
\begin{figure}[H!]
\centering\leavevmode
\begin{minipage}{17cm}
\centering\leavevmode
\includegraphics[width=8cm]{./benchmark_plots/Bank-cash.png}
\includegraphics[width=8cm]{./benchmark_plots/Bank-deposits.png}\\
\includegraphics[width=8cm]{./benchmark_plots/Bank-total_credit.png}
\includegraphics[width=8cm]{./benchmark_plots/Bank-equity.png}\\
\includegraphics[width=8cm]{./benchmark_plots/Bank-ecb_debt.png}
\includegraphics[width=8cm]{./benchmark_plots/Bank-total_dividends.png}
\end{minipage}
\caption{Credit market data.}
\label{Figure: Credit Market}
\end{figure}
%\clearpage

%\end{document}

\section{Future Development and Optimisation} 
\subsection{Partitioning and Dynamic Load Balancing}

The present implementation includes a \textit{timer} that has already been used to time sections of the FLAME framework and agent functions. Load balancing requires knowledge of the communication patterns in a particular model and the next phase will be to design and implement a \textit{communication data collector}. With this in place and a working EURACE model (or at least working isolaated market models) the performance of simulations can be assessed and a start can be made on designing strategies for balancing load and communication and hopefully a reduction in simulation time. The data collected will also help with partitioning the intitial data when multiple compute nodes are used.

\subsection{Job Submission}

The next step for the job submission scripts is to integrate them with the GUI in discussion with TUBITEK and extend the range of machines and job schedulers on which the submission system has been tested. There may be refinements of the scripts and functionality as the whole EURACE system comes together. 

\section{Conclusion} In this report we have described the parallel implementation of the FLAME framework and its assessment together with some benchmarking results using the EURACE Model. We have also demonstrated FLAMEs use in a number of EURACE related simulations in addition to complete EURACE model on populations ranging from a few hundereds of agents, through tens of thousands to, in one case, a million agents.  In some of these simulations the parallel implementation of FLAME has shown reasonable scalability and parallel efficiency but in other the results have been disappointing.

An important goal of the project has been to perform, in parallel, a large simulation using the EURACE Model. The project has achieve this to a degree: the model has been defined, important parameters have been values, a method of generating agent populations implemented and a parallel implementation of the EURACE model can be generated by FLAME. Using these steps serial and parallel simulations of the EURACE Model have been perform. In this process a detail assessment of the FLAME generated code, the serial and parallel implementations and the EURACE Model have been performed. Message counts, function times and sychronisation times are a few of the measures that have been used together with a detail static analysis of the model to identify the performance defficiencies in both the FLAME framework and the EURACE model.

All this analysis has lead to improvements in FLAME and the EURACE Model which in general have improved its computational performance. However the presence of substanial serial components in any model has resulted in very poor parallel scalability. It is well known that parallel speedup is limited by the serial faction of a code - this is Amdah's Law. The analyses performed on the EURACE Model have shown that the singleton agents - in particularly the Clearing House - have a significant impact of the parallel performance of the model.
These types of potential problem were understood - fine grained tasks - at the start of the project and the modeller took steps to avoid them. The Clearing House was thought necessary to the architecture of the EURACE Model and although different strategies were tested to reduce its effect there was little that could be achieved. The Clearing House and any other serial bottleneck will compromise the parallel performance of the application.

Although at the end of EURACE we have not achieved the \textsl{optimum} solution to these problems we have at least advanced the current state of the art in the parallel implementation of agent-based simulations in the context of the FLAME Framework.


\begin{thebibliography}{99}
\bibitem{Tesfatsion} Tesfatison (2006) ''Agent-based computational economics: a constructive approach to economic theory'' in Handbook for Computational Economics, Vol 2, Noth-Holland
\bibitem{Finin} Finin et al (1994) ''KQML as an Agent Communication Language'', The Proceedings of  the Third International Conference on Information and Knowledge Management
\bibitem{Gregory} Gregory et al (2001)  ''Computing Microbial Interactions and Communications in Real Life'', 4th International Conference on Information Processing in Cells and Tissues
\bibitem{Noble} Noble (2002)  ''Modeling the heart-from genes to cells to the whole organ'', Science
\bibitem{Coakley} Coakley  (2005) ''Formal Software Architecture for Agent-Based Modelling in Biology'', PhD Thesis, University of Sheffield
\bibitem{Walker-a} Walker et al (2004) ''Agent-based computational modeling of wounded epithelial cell monolayers'', IEEE Transactions in NanoBioscience
\bibitem{Walker-b} Walker et al (2004) ''The Epitheliome: Agent-Based Modelling Of The Social Behaviour Of Cells'', Biosystems
\bibitem{Pogson} Pogson  et al (2006) ''Formal Agent-Based Modelling of Intracellular Chemical Reactions'', to appear in Biosystems
\bibitem{Qwarnstrom} Qwarnstrom et al (2006) ''Predictive agent-based NFkB modelling - involvement of the actin cytoskeleton in pathway control'', Submitted
\bibitem{Jackson} Jackson et al (2004) ''Trial geometry gives polarity to ant foraging networks'', Nature
\bibitem{EURACE} EURACE (2006)  ''Agent-based software platform for European economic policy design with heterogeneous interacting agents'', EU IST Sixth Framework Programme.
\bibitem{Holcombe} Holcombe (1998) ''X-machines a basis for dynamic system specification'', Software Engineering Journal
\bibitem{Kefalas-a} Kefalas et al (2003) ''Communicating X-machines: From Theory to Practice'', Lecture Notes in Computer Science
\bibitem{Kefalas-b} Kefalas et al (2003) ''Simulation and verification of P systems through communicating X-machines'', Biosystems
\bibitem{Eleftherakis} Eleftherakis et al (2003) ''An agile formal development methodology'', Proceedings of the First South-East European Workshop on Formal Methods
\bibitem{Delli Gatti} Delli Gatti et al (2006), ''Emergent Macroeconomics An Agent-based Approach to Business Fluctuations'', submitted.
\end{thebibliography}
% End of document
\end{document} 
