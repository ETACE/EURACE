\subsection{Unit testing of the message board API}
As mentioned above the message boards are accessed by the FLAME framework via a Application Program Interface - the Message Board Library (the libmboard API). This provides the FLAME developer with a uniform interface to the functionality of the libmboard.

\subsection{Unit testing of Timer Package}

During development of the timer package component of the dynamic load balancing library the functionality of the package has been tested using unit tests. CUnit, a unit testing framework for C code has been used for this task as recommended by the UNICA unit. The tests implemented are:

\begin{itemize}
	\item Single timer creation. Test that the first timer has correct handle and elapsed time.
	\item Multiple timer creation. Create 3 timers and check the handles and elapsed time are correct.
	\item Multiple timer start. Check that starting a timer while still running has no effect.
	\item Timer stop. Check that stopping a timer really does stop it.
	\item Timer reset. Check that resetting a timer sets its elapsed time to zero.
	\item Multiple timer stop. Check that stopping an already stopped time has no effect.
\end{itemize}

An example program to illustrate the use of a timer is also provided.

\subsection{Testing serial and parallel implementations}
It is important to ensure that application generated by the FLAME framework execute \textsl{correctly} in both their serial and parallel modes. Because of the stocastic nature of the agent-based approach to modelling it is unrealistic to expect complex simulations to following exactly the solution path although general trends should be similar. However for some simple applications we can expect to serial and parallel implementations to produce exactly the results throughout the simulation. Such example applications can be used to verify the correctness of both the serial and parallel implementations.

The \textsl{Circles Model} is one such application. The \textsl{Circles} agent is very simple. It has a position in two-dimensional space and a radius of influence. Each agent will react to its neighbours within its interaction radius repulsively. So given a sufficient simulation time the initial distribution of agents will tend to a field of uniformly spaced agents. Each agent has $x$, $y$, $fx$, $fy$ and $radius$ in its memory and has three states: outputdata, inputdata and move. The agents communicate via a single message board, $location$, which holds the agent $id$ and position. Given the simplicity of the agent it is possible to determine the final result of a number of ideal models.

A set of simple test models and problems have been developed based on the \textsl{Circles} agent. Each test has a \textsl{model.xmml} files and a set of initial data (\textsl{0.xml}).
\begin{description}
	\item [Test 1]: Model: single \textsl{Circles} agent type; Initial population of no agents. Expected result:
	\item [Test 2]: Model: single \textsl{Circles} agent type; Initial population of one agent at (0,0).
	\item [Test 3]: Model: Two \textsl{Circles} agent type; Initial population of agents at (-1,0) and (+,0).
	\item [Test 4]: Model: Four \textsl{Circles} agent type; Initial population of one agent at ($\pm$1,$\pm$1).
	\item [Test 5]: Model: Four \textsl{Circles} agent type; Initial population of one agent at (0,$\pm$1) and ($\pm$1,0).
	\item [Test 6]: Model: Four \textsl{Circles} agent type; Initial population of one agent at random positions.
	\end{description}
In each of these models the expected results can be specified and therefore they provide a very simple check of the implementation.

The \textsl{Circles} agent also provides a good mechanism to check the parallel implementation against the serial. Such is the nature of the model the positions on the agents at each iteration of the simulation is independent on the order of calculation. As the order of calculation can not be easily prescript in the parallel simulation we can use this characteristic to test the validity of the parallel implementation against the serial. We would expect to get the identical positions for each agent at very iteration of the simulation.

