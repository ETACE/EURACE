\section{Execution}

Agents have a number of functions to perform. The order that these functions are run is defined by the states associated with each function.

Figure \ref{fig:functionstates} depicts an example of how the states
can link functions together. In the first case, the agent performs
two functions during the iteration step. The current state and the
next state determine the order of the functions. \emph{Function A}
is followed by \emph{Function B} by simply assigning the current and
next states to link the function chain together. Case 2, presents
another scenario, where the \emph{Function A} is run twice during a
simulation step. The same function can be run twice by linking if to
different current and next states.


\begin{figure}[!htb]
\begin{center}
  \includegraphics*[scale=0.4]{functionstates2.eps}
  \caption{Using states to form a function chain during one iteration step.}
  \label{fig:functionstates}
  \end{center}
\end{figure}

These order of states also determines the internal dependency
between the functions. This is only true if only one agent is being
discussed. But if there is a dependency between more than one agent,
communication dependencies are generated. These are denoted by the
messages being sent and read by other functions.



\begin{figure}[!htb]
\begin{center}
  \includegraphics*[scale=0.4]{commdependency.eps}
  \caption{Function D of Agent 2 depends on all Agent 1s to finish their Function A.}
  \label{fig:commdependency}
  \end{center}
\end{figure}

The communication dependency sets up a synchronisation point as
shown in Figure \ref{fig:commdependency}. This means that all agent
As have to finish running their \emph{Function A} before they can
start running the \emph{Function D} for agent B.

Using the X-machine methodology, the agents traverse through the
states to run the defined functions. These functions are also the
transition functions which are defined in the model XML file with
the,

\begin{itemize}
\item current state: the current state of the agent
\item input: the inputs the function is expecting
\item m$_{pre}$: the conditions on memory of executing the function
\item name: the name of the function
\item m$_{post}$: the changes in the memory (i.e. the function code)
\item output: possible outputs of the function
\item next state: the next state to move the agent to
\end{itemize}

By producing an order of function execution, this also provides a way to manage the processing of agents. By providing a link to an
agent list for each possible agent state, agents can be moved between these agent
state lists until they reach an end state.
