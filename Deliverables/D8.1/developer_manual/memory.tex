\section{Memory Allocation and its Problems}

Memory allocation for the agents and the messages is done as a continuous block size of memory.
The command \enph{sizeof} is used to return a byte size of the agent memory in use.
This is an important facet for parallelisation when using MPI. Sending data from one node to the other requires the program to know how many bytes have to be sent across to package it up in small packets. Thus it becomes important to determine its size.

\subsection{Dynamic Arrays}
FLAME also allows the use of dynamic arrays which causes a hindrance to this area of parallelisation. It is strongly discouraged for dynamic arrays to be used as part of the agent memory, if the agents have to be moved around in parallel. Dynamic arrays also prevents the associated memory to be allocated as blocks of continuous memory. Messages are another factor which discourages the use of dynamic arrays within the messages. The size of the message becomes difficult to be determined and sent to and fro for this reason.

\subsection{Data types}

User-defined data types are allocated as pointers in agent memory but
this has been modified in a new version to be released. This means that instead
of user using an arrow '-\textgreater' to dereference variables, a dot '.' is
used to access the data structure.

Dynamic array data structures are also not allocated as a pointer (but the actual
dynamic array is) which means functions to interact with a dynamic array data
structure need to pass a pointer. This means the use of the ampersand '\&' to
reference the data structure.

\subsection{Agent Memory Management}

Each agent has an associated memory data structure. Since the early versions of the
framework all agents have been managed in one list. This was so that the list
could be randomised and therefore remove any chances of agents having priority
over other agents by always being executed first. In essence, the same effect can be achieved by randomising the messages output and therefore the message inputs into agents.
The current framework has a generic agent memory stucture that can point to any
specific agent type.

With the introduction of the new message board library the action of randomising
(or now also sorting and filtering) messages the need to randomise the agent list
is redundant. Also redundant is the need to have a single list of all the agents.
The generic agent memory structure is therefore not needed and each agent type
can have it's only seperate list.
