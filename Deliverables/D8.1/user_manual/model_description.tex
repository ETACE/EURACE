\section{Model Description}
\label{model_description}

Models descriptions are formatted in XML (Extensible Markup Language) tag
structures to allow easy human and computer readability. This also allows easier collaborations between the
developers writing the application functions that interact with model definitions in the XML.

The DTD (Document Type Definition) of the XML document is currently located
at:

\begin{mylisting}
\begin{verbatim}
http://eurace.cs.bilgi.edu.tr/XMML.dtd
\end{verbatim}
\end{mylisting}

For users who are familiar with the HTML structure, a XML document is structured in a similar way as a nested tree structure, where tags contain data or other tags within them. This structure can be condensed into one level or a number of levels within the parent levels. In FLAME, the start and the end of a model file looks like as follows:

\begin{mylisting}
\begin{verbatim}
<?xml version="1.0" encoding="ISO-8859-1"?>
<!DOCTYPE xmodel SYSTEM "http://eurace.cs.bilgi.edu.tr/XMML.dtd">
<xmodel version="2">
<name>Model_name</name>
<version>the version</version>
<description>a description</description>
...
</xmodel>
\end{verbatim}
\end{mylisting}

The complete model is contained within the tag level of '\emph{xmodel}' . The \emph{name} of the model is the name of the model being modelled, \emph{version} denotes the version number of the model. The \emph{description} tags allows the model description to be contained in it for modellers to make notes. 

\subsection{Tags within the XModel}
Defining the xmodel is the parent level in the xml file being read by FLAME. This xmodel can be condensed into a number of different tag trees which contain further details about the model. These tags can contain information about:

\begin{itemize}
\item \textbf{Other models} - Other models can be enabled or disabled when being plugged into a model. This is to allow modeller to test more than one model at a time as well as mix a number of models together.
\item \textbf{Environment} - The environment contains the global variables of the model in which the agents exist in. Sometimes modellers make the environment act as an agent too with functions and memory states. But this requires another agent to be listed. Here the environment can act as global with constant values for all agents.
\begin{itemize}
\item Constant variables - Global variables.
\item Location of function files - Location where the functions or C files of the agents are located. 
\item Time units - Enables the programming of calenders, which can be assigned to each function to enable it to be active only at specific times during the simulation.
% \begin{itemize}
% \item name
% \item *** unit
% \item *** period
% \end{itemize}
\item Data types - Agent memories can use data structures for some of the variables instead of the traditional C variable types like int, char or double. These data types can be defined by the modeller to contain more than one type or array within it.
% \begin{itemize}
% \item name
% \item description
% \item variables
% \end{itemize}
\end{itemize}
\item \textbf{Agent types} - The agents involved in the system. For instance, in the swarm model, there was only one type of agent the bird agents. In an alternate model of the predator prey model there are two agent types, the fix and the rabbit. These depend on the model being modelled and the modeller's perspectives.
\begin{itemize}
\item Name - Name of the agent type
\item Description - Textual description of the agent.
\item Memory - A list of the memory variables for each type of agent.
% *** variables
\item Functions - A list of functions the agent can perform. These functions are encapsulated with states like the current and the next state to move to after this function has been executed. The functions would also contain the names of the messages being read in or output from the functions.
% *** name
% *** description
% *** current state
% *** next state
% *** condition
% *** inputs
% **** filter
% *** outputs
\end{itemize}
\item \textbf{Message types} - These are a list of all the messages being used in the model.
\begin{itemize}
\item Name - Name of the message.
\item Description - Textual description of the message.
\item Variables - Variables encapsulated with in the message.
\end{itemize}
\end{itemize}

Refer to the Appendix to see how these tags are brought together in one model xml file.

\subsection{Model in Multiple Files}

It is possible to define a model in a collection of multiple files. FLAME reads a model from
multiple files as if the model was defined in one file. This capability allows
different parts of a model to be enabled or disabled easily. For example if a
model includes different versions of a sub-model, these can be exchanged, or a
subsystem of a model can be disabled to see how it affects the model.
Alternatively this capability could be used as a hierarchy, for example a `body'
model could include a model of the `cardiovascular system' that includes a
model of the `heart'. The following tags show the inclusion of two models, one is
enabled and one disabled:

\begin{mylisting}
\begin{verbatim}
<models>
  <model><file>sub_model_1.xml</file><enabled>true</enabled></model>
  <model><file>sub_model_2.xml</file><enabled>false</enabled></model>
</models>
\end{verbatim}
\end{mylisting}


\subsection{Environment}

The environment of a model holds information that maybe required by a model but
is not part of an agent or a message. This includes:

\begin{itemize}
\item Constant variables - for setting up different simulations easily.
\item Location of function files - the path to the implementations of agent
functions in C files.
\item Time units - for easily activating agent functions dependent on time
periods.
\item Data types - user defined data types used by agent memory or
message variables other that typical C data types.
\end{itemize}

This notion of environment does not correspond to an environment that would be
a part of a model where agents would interact with the environment. Anything
that can change in a model must be represented by an agent, therefore if a
model includes a changeable environment that agents can interact with, this in
itself must be represented by an agent.

\subsubsection{Constant Variables}

Constant variables can be set up as part of a simulation for the runs. These are defined as follows:

\begin{mylisting}
\begin{verbatim}
<constants>
  <variable>
   <type>int</type><name>my_constant</name>
   <description>value read in initial simulation settings</description>
  </variable>
</constants>
\end{verbatim}
\end{mylisting}


Constant Variables refers to the global values used in the model. These can also be
defined in a separate header H file which can then be included in one of the
functions C file.
The header file should contain the global variable as:

 \begin{mylisting}
 \begin{verbatim}
#define <varname> <value>
 \end{verbatim}
 \end{mylisting}

 This file has to be saved as `my\_header.h' file, include this file into one of
 the function files so that the compiler knows about these arguments.

\subsubsection{Function Files}

Function files hold the source code for the implementation of the
agent functions. These are programmed in C language.
They are included in the compilation script (Makefile) of the produced model:

\begin{mylisting}
\begin{verbatim}
 <functionFiles>
 <file>function_source_code_1.c</file>
 <file>function_source_code_2.c</file>
 </functionFiles>
\end{verbatim}
\end{mylisting}

\subsubsection{Time Units}
\label{timeunit}

% Time units allow the possibility of restricting the functions to
% only execute during particular iterations.
% Time rules can be applied to function conditions instead of a
% condition rule and are defined by a time period and a phase. A time
% phase is the offset from the start of a period.

Time units are used to define time periods that agent functions act within. For
example a model that uses a calendar based time system could take a day to be
the smallest time step, i.e. one iteration. Other time units can then use this
definition to define other time units, for example weeks, months, and years.

A time unit contains:

\begin{itemize}
\item Name - name of the time unit.
\item Unit - can contain `iteration' or other defined time units.
\item Period - the length of the time unit using the above units.
\end{itemize}

An example of a calendar based time unit set up is given below:

\begin{mylisting}
\begin{verbatim}
<timeUnits>
  <timeUnit>
    <name>daily</name>
    <unit>iteration</unit>
    <period>1</period>
  </timeUnit>

  <timeUnit>
    <name>weekly</name>
    <unit>daily</unit>
    <period>5</period>
  </timeUnit>

  <timeUnit>
    <name>monthly</name>
    <unit>weekly</unit>
    <period>4</period>
  </timeUnit>

  <timeUnit>
    <name>quarterly</name>
    <unit>monthly</unit>
    <period>3</period>
  </timeUnit>

  <timeUnit>
    <name>yearly</name>
    <unit>monthly</unit>
    <period>12</period>
  </timeUnit>

</timeUnits>
\end{verbatim}
\end{mylisting}

These time units can be added to the functions, when they are listed as part of the agent. These time units act as conditions on the functions. This has been discussed in Section \ref{functioncond}.

\subsubsection{Data Types}

Data types are user defined data types that can be used in a model. They are a
structure for holding variables. Variables can be a:

\begin{itemize}
  \item Single C fundamental data types - int, float, double, char.
  \item Static array - of size ten: variable\_name[10].
  \item Dynamic array - available by placing `\_array' after
  the data type name: variable\_name\_array.
  \item User defined data type - defined before the current data type.
\end{itemize}

The example below contains a variable
of data structure \emph{position} which contains the x, y and z position in one structure. The position data structure can then be a data type in the \emph{line} data structure.

\begin{mylisting}
\begin{verbatim}
<dataTypes>

 <dataType>
  <name>position/name>
  <description>position in 3D using doubles</description>
  <variables>
   <variable><type>double</type><name>x</name>
    <description>position on x-axis</description>
   </variable>
   <variable><type>double</type><name>y</name>
    <description>position on y-axis</description>
   </variable>
   <variable><type>double</type><name>z</name>
    <description>position on z-axis</description>
   </variable>
  </variables>
 </dataType>

 <dataType>
  <name>line</name>
  <description>a line defined by two points</description>
  <variables>
   <variable><type>position</type><name>start</name>
    <description>start position of the line</description>
   </variable>
   <variable><type>position</type><name>end</name>
    <description>end position of the line</description>
   </variable>
  </variables>
 </dataType>

</dataTypes>
\end{verbatim}
\end{mylisting}

\subsection{Agents}

A model has to constitute agents. These agents are defined as their type in the model xml file. An agent type contains a name, a description, memory, and functions:

\begin{mylisting}
\begin{verbatim}
<agents>

  <xagent>
    <name>Agent_Name</name>
    <description></description>
    <memory>
     ...
    </memory>
    <functions>
      ...
    </functions>
  </xagent>
\end{verbatim}
\end{mylisting}
%
%   <xagent>
%     <name>Household</name>
%     <description></description>
%     <memory>
%       <variable><type>int</type><name>id</name>
%        <description></description>
%       </variable>
%       <variable><type>int</type><name>region_id</name>
%        <description></description>
%       </variable>
%       <variable><type>int_array</type><name>neighboring_region_ids</name>
%        <description></description>
%       </variable>
%       <variable><type>int</type><name>gov_id</name>
%        <description></description>
%       </variable>
%       <variable><type>int</type><name>day_of_month_to_act</name>
%        <description></description>
%       </variable>
%       <variable><type>double</type><name>payment_account</name>
%        <description></description>
%       </variable>
%     </memory>
%     <functions>
%      <function>
%        <name>Household_read_firing_messages</name>
%         <description>Check for being fired or not</description>
%         <currentState>EXIT_FINANCIAL_MARKET</currentState>
%         <nextState>01d</nextState>
%         <condition>
%          <lhs><value>a.employee_firm_id</value></lhs>
%          <op>NEQ</op>
%          <rhs><value>-1</value></rhs>
%         </condition>
%         <inputs>
%          <input><messageName>firing</messageName></input>
%         </inputs>
%       </function>
%     </functions>
%   </xagent>
% </agents>


\subsubsection{Agent Memory}

Agent memory defines variables, where variables are defined by their type, C
data types or user defined data types from the environment, a name, and a
description:

\begin{mylisting}
\begin{verbatim}
<memory>
 <variable><type>int</type><name>id</name>
  <description>identity number</description>
 </variable>
 <variable><type>double</type><name>x</name>
  <description>position in x-axis</description>
 </variable>
  <variable><type>position</type><name>xyz</name>
  <description>position in x-axis, y-axis, z-axis</description>
 </variable>
</memory>
\end{verbatim}
\end{mylisting}

\subsubsection{Agent Functions}

The model xml file requires the agent functions to be listed as well to tell FLAME when the functions will be called in from the C files. An agent function contains:

\begin{itemize}
\item Name - the function name which must correspond to an implemented function
name
\item Description
\item Current state - the current state the agent has to be in for this function to execute.
\item Next state - the next state the agent will transition to after the function.
\item Condition - a possible condition of the function transition.
\item Inputs - the possible input messages.
\item Outputs - the possible output messages.
\end{itemize}

And as tags, the xml file will contain:

\begin{mylisting}
\begin{verbatim}
<function>
 <name>function_name</name>
 <description>function description</description>
 <currentState>current_state</currentState>
 <nextState>next_state</nextState>
 <condition>
 ...
 </condition>
 <inputs>
 ...
 </inputs>
 <outputs>
 ...
 </outputs>
</function>
\end{verbatim}
\end{mylisting}

The current state and next state tags hold the names of states. This is the
only place where states are defined. State names must coordinate with other
functions states to produce a transitional graph from the start state to end
states.

\subsubsection{Function Condition}
\label{functioncond}

A function can have a condition on its transition. This condition can include
conditions on the agent memory and also on any time units defined in the
environment. Each transition will take the agent from a starting state to an end state at the end of the simulation.

Each possible transition
must be mutually exclusive. This means that if a certain condition is true on one part of the branch of functions, there should be an alternate branch which would be the opposite of this condition. This will ensure the model does not halt in the middle during simulation if the condition fails. A function named `idle' is available to be used for
functions that do not require an implementation and a reverse of the conditions.

Conditions (that are not just time unit based) take the form:

\begin{itemize}
  \item lhs - left hand side of comparison.
  \item op - the comparison operator.
  \item rhs - the right hand side of the comparison.
\end{itemize}

Or in tags form:

\begin{mylisting}
\begin{verbatim}
<lhs></lhs><op></op><rhs></rhs>
\end{verbatim}
\end{mylisting}

Sides to compare (lhs or rhs) can be either a value, denoted within value tags or
a formula. 
Values and formulas can include agent variables, which are preceded by `a', or message variables, which are preceded by `m.'.

 \begin{mylisting}
 \begin{verbatim}
 a.agent_var
 m.message_var
 \end{verbatim}
 \end{mylisting}

The comparison operator, op, can be one of the following comparison functions:

\begin{itemize}
\item EQ - equal to.
\item NEQ - not equal to.
\item LEQ - less than or equal to.
\item GEQ - greater than or equal to.
\item LT - less then.
\item GT - greater than.
\item IN - an integer (in lhs) is a member of an array of integers (in rhs).
\end{itemize}

Or one of the following logic operators can be used as well:

\begin{itemize}
\item AND
\item OR
\end{itemize}

The operator `NOT' can be used by placing `not' tags around a comparison rule.
For example the following tagged rule describes the condition being true when
the `z' variable of the agent is greater than zero and less than ten:

\begin{mylisting}
\begin{verbatim}
<condition>
 <lhs>
  <lhs><value>a.z</value></lhs>
  <op>GT</op>
  <rhs><value>0.0</value></rhs>
 </lhs>
 <op>AND</op>
 <rhs>
  <not>
  <lhs><value>a.z</value></lhs>
  <op>LT</op>
  <rhs><value>10.0</value></rhs>
  </not>
 </rhs>
</condition>
\end{verbatim}
\end{mylisting}

\subsubsection{Time conditions}

A condition can also depend on any time units described in the environment. For
example the following condition is true when the agent variable
`day\_of\_month\_to\_act' is equal to the number of iterations since of the
start, the phase, of the `monthly' period, i.e. twenty iterations as defined in
the time unit:

\begin{mylisting}
\begin{verbatim}
 <condition>
     <time>
     <period>monthly</period>
     <phase>a.day_of_month_to_act</phase>
     </time>
 </condition>
\end{verbatim}
\end{mylisting}

 The condition allows the function to run \emph{monthly} at the phase
 of \emph{day\_of\_month\_to\_act}. The
 \emph{day\_of\_month\_to\_act} is a variable extracted from the
 agent memory and is thus defined as
 \emph{a.day\_of\_month\_to\_act}.

%
% Refer to section \ref{functioncond} for more details on function
% condition definitions.
%
% These rules are then parsed into rule functions and placed in a file
% called rules.c
\subsubsection{Messages in and out of Functions}
Functions can have input and output message types. For example, the following
example the function takes message types `a' and `b' as inputs and outputs
message type `c':

\begin{mylisting}
\begin{verbatim}
<inputs>
 <input><messageName>a</messageName></input>
 <input><messageName>b</messageName></input>
</inputs>
<outputs>
 <output><messageName>c</messageName></output>
</outputs>
\end{verbatim}
\end{mylisting}

%\paragraph{Message Filter}
\subsubsection{Filters}\label{sect:msgfilter}
Message filters can be applied to message inputs to allow the messages to be
filtered. Filters are defined similar to function conditions but include
message variables which are prefixed by an `m'. The following example filter only
allows messages where the agent variable `id' is equal to the message variable
`worker\_id':

\begin{mylisting}
\begin{verbatim}
<input>
 <messageName>firing</messageName>
 <filter>
  <lhs><value>a.id</value></lhs>
  <op>EQ</op>
  <rhs><value>m.worker_id</value></rhs>
 </filter>
 <random>false<random>
</input>
\end{verbatim}
\end{mylisting}

The previous example also includes the use of a random tag, set to false, to show
that the input does not need to be randomised, as randomising input messages can be
computationally expensive. By default all message inputs are randomised.

Using filters in the model description enables FLAME to make message
communication more efficient by pre-sorting messages and using other techniques.

% Thus in the above example messages will be filtered according to the
% message variable \emph{worker\_id} (defined as m.<varname>) to be EQ
% (equal) to the agent \emph{id} (defined as a.<varname>).

\subsection{Messages}

Messages defined in a model must have a type which is defined by a name and the
variables that are included in the message. The following example is a message
called `signal' that holds a position in 3D.

\begin{mylisting}
\begin{verbatim}
<messages>

 <message>
  <name>signal</name>
  <description>Holds the position of the sending agent</description>
  <variables>
    <variable><type>double</type><name>x</name>
     <description>The x-axis position</description>
    </variable>
    <variable><type>double</type><name>y</name>
     <description>The y-axis position</description>
    </variable>
    <variable><type>double</type><name>z</name>
     <description>The z-axis position</description>
    </variable>
  </variables>
 </message>

</messages>
\end{verbatim}
\end{mylisting}
