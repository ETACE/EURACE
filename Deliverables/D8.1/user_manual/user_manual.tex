\documentclass[12pt,a4paper]{article}
\usepackage{t1enc}
\usepackage[latin1]{inputenc}
\usepackage[english]{babel}
\usepackage{url,graphics,lscape}
\usepackage{a4wide}
\usepackage{graphicx}
\usepackage{color}
%\usepackage{fancyheadings}
\usepackage{verbatim}

\vfuzz2pt % Don't report over-full v-boxes if over-edge is small
\hfuzz2pt % Don't report over-full h-boxes if over-edge is small

\newenvironment{mylisting}
{\begin{list}{}{\setlength{\leftmargin}{1em}}\item\small\bfseries}
{\end{list}}

\parskip 6pt         % sets spacing between paragraphs
\parindent 0pt       % sets leading space for paragraphs
%\pagestyle{fancy}
\begin{document}



\title{\textbf{FLAME}
\\Flexible Large-scale Agent-based Modelling Environment\\
 \emph{User Manual}}
\author{Simon Coakley\\Mariam Kiran
\\
\\ University of Sheffield
\\ Unit - USFD}
%\date{17 May 2006}

\maketitle

\pagebreak

\begin{abstract}

FLAME (Flexible Large-scale Agent-based Modelling Environment) is a
tool which allows modellers from all disciplines, economics, biology
or social sciences to easily write their own agent-based models. The
environment is a first of its kind which allows simulations of large
concentrations of agents to be run on parallel computers without any
hindrance to the modellers themselves.

This document present a comprehensive guide to the keywords and
functions available in the FLAME environment for the modellers to
write their own agent models to facilitate research. This user
manual describes how to create model description and write
implementation code for the agents.

Installation guides to the tools required with FLAME have been
provided separately in the document titled `\emph{Getting started
with FLAME}'. this documentation also contains details on executing
example FLAME code and running your own models.

\end{abstract}

\pagebreak
\tableofcontents
\pagebreak

\section{Introduction}

The FLAME framework is an enabling tool to create agent-based models
that can be run on high performance computers (HPCs). The framework
is based on the the logical communicating extended finite state
machine theory (X-machine) which gives the agents more power to
enable writing of complex models for large complex systems.

The agents are modelled as communicating X-machines allowing them to
communicate through messages being sent to each other as per
required by the modeller. This information is automatically read by
the FLAME framework and generate a simulation program which enables
these models to be parallelised efficiently over parallel computers.

The simulation program for FLAME is called the \textbf{Xparser}. The
Xparser is a series of compilation files which can be compiled with
the modeller's files to produce the simulation package for running
the simulations. Various tools have to be installed with the Xparser
to allow the simulation program o be produced. These have been
explained in the accompanying document, `\emph{Getting started with
FLAME}'.

Various parallel platforms like SCARF, (add more) have been used in
the development process to test the efficiency of the FLAME
framework. This work was done in conjunction with STFC and more
details of the results obtained can be found in `\emph{Deliverable
1.4: Porting of agent models to parallel computers}'.



\begin{figure}[!htb]
\begin{center}
  \includegraphics*[scale=0.35]{xparserdiag.eps}
  \caption{Block diagram of the Xparser, the FLAME simulation program. Blocks in blue are the files automatically generated. The green blocks are modeller files.}
  \label{fig:xparserdiag}
  \end{center}
\end{figure}

This document is a comprehensive guide to modellers of all
disciplines to write their own agent models with ease. The key
advantage of FLAME lies in the fact that modellers from all
disciplines, regardless of belonging to the computer science
background can write their own models with ease.

This document has been divided in the following order, Section
\ref{chap:design} describes the X-machine methodology to explain the
structure of the agents. Modellers are required to write only two
files for their model, model.xml and the functions of the agents.
Section \ref{model_description} describes how the model is written
focussing on writing the `model.xml' file which contains the
complete description of the model, with the agents involved, their
structures and the environment they exist in. Section
\ref{model_implementation} focuses on the second modeller file which
is writing the agent functions and the routines provided by FLAME.
Section \ref{model_execution} gives a brief description of how the
model can be executed with a summary of the facilities provided by
FLAME for data analysis for the output files generated. Appendices
provides a listing of the tags and the data flow documentation used
by FLAME.

\section{Model Design}\label{chap:design}
\label{model_design}

Traditionally specifying software behaviour has used finite state
machines to express its working. Extended finite state machines (X-machines) are more
powerful than the simple finite state machine as it gives the model more flexibility than a traditional finite state machine. 

FLAME uses X-machines to represent all agents acting in the system. Each would thus possess the following characteristics:

\begin{itemize}
\item A finite set of internal states of the agent.
\item Set of transitions functions that operate between the states.
\item An internal memory set of the agent.
\item A language for sending and receiving messages among agents.
\end{itemize}

Figure \ref{fig:commxm} shows the structure of how two X-machines will communicate. The machines communicate through a common message board, to which they post and read from their messages.
Using conventional state machines to describe the state-dependent behaviour
of a system by outlining the inputs to the system, but this failed to include the
effect of messages being read and the changes in the memory values of the machine. X-Machines are an extension to conventional state
machines that include the manipulation of memory as part of the
system behaviour, and thus are a suitable way to specify agents.
Describing a system in FLAME includes the following stages:

\begin{itemize}
\item Identifying the agents and their functions.
\item Identify the states which impose some order of function
execution with in the agent.
\item Identify the input messages and output messages of each function
(including possible filters on inputs which will be explained in Section \ref{sect:msgfilter}).
\item Identify the memory as the set of variables that are accessed by
functions (including possible conditions on variables for the
functions to occur).
\end{itemize}



\begin{figure}[!htb]
\begin{center}
  \includegraphics*[scale=0.45]{commxm.eps}
  \caption{How two agent x-machines communicate. The agents send and read messages from the message board which maintains a database of all the messages sent by the agents.}
  \label{fig:commxm}
  \end{center}
\end{figure}



\subsection{Swarm Example}

A swarm model in a model which presents the behaviour of birds flocking together. The individual birds follow simple rules, but collectively they produce complex behaviour of the group, as observed in nature.
This simple flocking
model involves birds to sense where other birds are and then respond accordingly. The activities or functions they perform are:

\begin{itemize}
\item Observe if there is a bird nearby.
\item Adjust bird position, direction and velocity accordingly.
\end{itemize}


Converting this model into an agent-based model requires visualising the model as a collection of agents. As the only individuals involved in the model are birds, agents will be representing birds. The functions these bird agents would perform will be:

\begin{itemize}
\item Signal. The agent would send information of its current
position.
\item Observe. The agent would read in the positions from other agents and possibly change
velocity.
\item Respond. The agent would update position via the current
velocity.
\end{itemize}

The functions would occur in an order as shown in Figure
\ref{fig:swarm_1}. The complete figure represents the functions the agents would be performing during one iteration\footnote{FLAME prevents the agents to loop back due to
parallelisation constraints.}.



\begin{figure}[ht]
\begin{center}
\includegraphics*[scale=0.65]{swarm_1.ps}
\caption{Swarm model including states}
\label{fig:swarm_1}
\end{center}
\end{figure}

Figure \ref{fig:swarm_2} depicts a situation where their would be conditions added to the functions of the agents. For instance, in the swarm model, there could be a condition added to the z-axis value to determine which response function to perform for the agent. If z is more than zero, the agent would be flying, else if z is zero, then the agent is stationary.


\begin{figure}[ht]
\begin{center}
\includegraphics*[scale=0.65]{swarm_2.ps}
\caption{Swarm model including function conditions}
\label{fig:swarm_2}
\end{center}
\end{figure}

The message being used for communication between the agents, in the model, is a signal message, which is the output from `signal' function and the input to the `observe' function (Figure
\ref{fig:swarm_3}). This message includes the position of the agent that
sent it with the x, y and z coordinates (Table \ref{tab:signal_message}). 


\begin{figure}[ht]
\begin{center}
\includegraphics*[scale=0.65]{swarm_3.ps}
\caption{Swarm model including messages}
\label{fig:swarm_3}
\end{center}
\end{figure}



\begin{table}[ht]
\centering
\begin{tabular}{|l||c||l|}
\hline
Type&Name&Description\\
\hline \hline
double&px&x-axis position\\
\hline
double&py&y-axis position\\
\hline
double&pz&z-axis position\\
\hline
\end{tabular}
\caption{Signal Message}
\label{tab:signal_message}
\end{table}


An important factor to note here is that FLAME carries the features of a filter which can be added to the messages. This filter can ensure that only the messages in the agents viewing distance are being read, preventing each agent to traverse through all the messages on the message board. The filter will be a formula involving the position contained in the
message (the position of the sending agent) and the receiving agent position.


\subsection{Transition Functions}
 Transition functions allow agents to change the state in
 which they are in, modifying their behaviour. Transition functions take as input the current state $s_{1}$ of the agent, current memory value
 $m_{1}$ and the possible arrival of a message that the agent reads  $t_{1}$. Depending on these three values the agent changes to another state $s_{2}$, updates the memory to $m_{2}$ and
 optionally sends a message $t_{2}$. 
 
 
There could be situations where some of the transition functions do not depend on the incoming
 messages. Agent transition functions may also be expressed in terms of
 stochastic rules, which allows the multi-agent systems to be called stochastic systems.

 \subsection{Memory and States}
 The difference between the internal set of states and the internal
 memory set allows the added flexibility when modelling systems.
 There can be agents with one internal state and all the complexity
 defined in the memory or equivalently, there could be agents with
 a trivial memory, with the complexity then bound up in a large state
 space. It depends on the modeller's perspective on how he/she write the model and where the complexity is added.

In FLAME, one iteration is taken as a
standalone run of a simulation. Once all the functions in that iteration have taken place, the message board is emptied, deleting all the messages. This means that messages cannot be sent between iterations, thus models have to be written in a way which considers this.

Table \ref{tab:swarm_memory} describes the memory variables being used by the bird agents in the swarm model. 


\begin{table}[ht]
\centering
\begin{tabular}{|l||c||l|}
\hline
Type&Name&Description\\
\hline \hline
double&px&position in x-axis\\
\hline
double&py&position in y-axis\\
\hline
double&pz&position in z-axis\\
\hline
double&vx&velocity in x-axis\\
\hline
double&vy&velocity in y-axis\\
\hline
double&vz&velocity in z-axis\\
\hline
\end{tabular}
\caption{Swarm Agent Memory}
\label{tab:swarm_memory}
\end{table}

Modellers can add more variables to the agent memory as they see required. Table \ref{tab:swarmtransition} represents a transition table presentation of the swarm model. The terms in the table have been defined below:

\begin{itemize}
  \item Current State - is the state the agent is currently in.
  \item Input - is any inputs into the transition function.
  \item $M_{pre}$ - are any preconditions of the memory on the transition.
  \item Function - is the function name.
  \item $M_{post}$ - is any change in the agent memory.
  \item Output - is any outputs from the transition.
  \item Next State - is the next state that is entered by the agent.
\end{itemize}

%\begin{landscape}
\begin{table}[ht]
\centering
\begin{tabular}{|c|c|c||c||c|c|c|}
\hline
Current State&Input&$M_{pre}$&Function&$M_{post}$&Output&Next State\\
\hline
\hline
start&&&signal&&signal&1\\
\hline
1&signal&&observe&(velocity updated)&&2\\
\hline
2&&$x > 0$&flying&(position updated)&&end\\
\hline
2&&$x == 0$&resting&(position updated)&&end\\
\hline
\end{tabular}
\caption{Swarm Agent Transition Table}
\label{tab:swarmtransition}
\end{table}
%\end{landscape}

The next Section \ref{model_description} describes how a model can be written up in the xml format that FLAME can understand. Section
\ref{model_implementation} discusses how to implement
the individual agent functions, i.e. $M_{post}$ from the transition table.
Section \ref{model_execution} on model execution describes how to use the tools
in FLAME to generate a simulation program and execute the simulations.

% \begin{equation}\label{streamxmachine}
%     X = (\Sigma, \Gamma, Q, M, \Phi, F, q_{0}, m_{0})
% \end{equation}
% where,
% \begin{itemize}
% \item $\Sigma$ are the set of input alphabets
% \item $\Gamma$ are the set of output alphabets
% \item $Q$ denotes the set of states
% \item $M$ denotes the variables in the memory.
% \item $\Phi$ denotes the set of partial functions $\phi$ that map
% and input and memory variable to an output and a change on the
% memory variable. The set $\phi$: $\Sigma \times M\ \longrightarrow\
% \Gamma\times M$
% \item $F$ in the next state transition function, $F : Q \times\phi\longrightarrow
% Q$
% \item $q_{0}$ is the initial state and $m_{0}$ is the initial memory
% of the machine.
% \end{itemize}
%
% \subsection{Transition Function}
% The transition functions allow the agents to change the state in
% which they are in, modifying their behaviour accordingly. These would
% require as inputs their current state $s_{1}$, current memory value
% $m_{1}$, and the possible arrival of a message that the agent is able to
% read, $t_{1}$. Depending on these three values the agent can then
% change to another state $s_{2}$, updates the memory to $m_{2}$ and
% optionally sends a message, $t_{2}$. Figure
% \ref{fig:trans} depicts how the transition function
% works within the agent.
%
% % \begin{figure}
% % \begin{center}
% % \includegraphics*[scale=0.5]{transfn.eps}
% % \caption{Transition function} \label{fig:trans}
% % \end{center}
% % \end{figure}
%
%
% Extended finite state machines or X-Machines are used to define agents within a
% model.
% The basic definition of an
% agent would thus, in accordance to the computational model, contain
% the following components:
% \begin{enumerate}
%  \item A finite set of internal states.
%  \item A set of transition functions that operate between states.
%  \item An internal memory set. In practice, the memory would be a finite set and can be structured in any way required.
%  \item A language for sending and receiving messages between other agents.
% \end{enumerate}
%
%
% Some of the transition functions may not depend on the incoming
% message. Thus the message would then be represented as:
% \begin{equation}\label{msg}
%     Message = \{ \emptyset, <data> \}
% \end{equation}
%
% These agent transition functions may be expressed in terms of
% stochastic rules, thus allowing the multi-agent systems to be termed
% as stochastic systems.
%
% \subsubsection{Memory and States}
% The difference between the internal set of states and the internal
% memory set allows for added flexibility when modelling systems.
% There can be agents with one internal state and all the complexity
% defined in the memory or equivalently, there could be agents with
% a trivial memory with the complexity then bound up in a large state
% space. There are good examples of choosing an appropriate balance
% between these two as this enables the complexity of the models to be
% better managed.

% \begin{figure}
% \begin{center}
% \includegraphics*[width = 4in]{X-Machine_agent.eps}
% \caption{X-Machine agent} \label{fig:xmachine}
% \end{center}
% \end{figure}

\section{Model Description}
\label{model_description}

Models descriptions are formatted in XML (Extensible Markup Language) tag
structures to allow easy human and computer readability. This also allows easier collaborations between the
developers writing the application functions that interact with model definitions in the XML.

The DTD (Document Type Definition) of the XML document is currently located
at:

\begin{mylisting}
\begin{verbatim}
http://eurace.cs.bilgi.edu.tr/XMML.dtd
\end{verbatim}
\end{mylisting}

For users who are familiar with the HTML structure, a XML document is structured in
a similar way as a nested tree structure, where tags contain data or other tags within them.
This structure can be condensed into one level or a number of levels within the parent levels.
In FLAME, the start and the end of a model file looks like as follows:

\begin{mylisting}
\begin{verbatim}
<?xml version="1.0" encoding="ISO-8859-1"?>
<!DOCTYPE xmodel SYSTEM "http://eurace.cs.bilgi.edu.tr/XMML.dtd">
<xmodel version="2">
<name>Model_name</name>
<version>the version</version>
<description>a description</description>
...
</xmodel>
\end{verbatim}
\end{mylisting}

The complete model is contained within the tag level of
`\emph{xmodel}' . The \emph{name} of the model is the name of the
model being modelled, \emph{version} denotes the version number of
the model. The \emph{description} tags allows the model description
to be contained in it for modellers to make notes.

\subsection{Tags within the XModel}
Defining the xmodel is the parent level in the xml file being read by FLAME. This xmodel can be condensed into a number of different tag trees which contain further details about the model. These tags can contain information about:

\begin{itemize}
\item \textbf{Other models} - Other models can be enabled or disabled when being plugged into a model. This is to allow modeller to test more than one model at a time as well as mix a number of models together.
\item \textbf{Environment} - The environment contains the global variables of the model in which the agents exist in. Sometimes modellers make the environment act as an agent too with functions and memory states. But this requires another agent to be listed. Here the environment can act as global with constant values for all
agents. The environment can contain the following information,
\begin{itemize}
\item Constant variables - Global variables.
\item Location of function files - Location where the functions or C files of the agents are located.
\item Time units - Enables the programming of calenders, which can be assigned to each function to enable it to be active only at specific times during the simulation.
% \begin{itemize}
% \item name
% \item *** unit
% \item *** period
% \end{itemize}
\item Data types - Agent memories can use data structures for some of the variables instead of the traditional C variable types like int, char or double. These data types can be defined by the modeller to contain more than one type or array within it.
% \begin{itemize}
% \item name
% \item description
% \item variables
% \end{itemize}
\end{itemize}
\item \textbf{Agent types} - The agents involved in the system. For instance, in the swarm model, there was only one type of agent the bird agents. In an alternate model of the predator prey model there are two agent types, the fix and the rabbit. These depend on the model being modelled and the modeller's
perspectives. The agents are defined by the `\emph{xagent}' tag and
can contain the following information,
\begin{itemize}
\item Name - Name of the agent type
\item Description - Textual description of the agent.
\item Memory - A list of the memory variables for each type of agent.
% *** variables
\item Functions - A list of functions the agent can perform. These functions are encapsulated with states like the current and the next state to move to after this function has been executed. The functions would also contain the names of the messages being read in or output from the functions.
% *** name
% *** description
% *** current state
% *** next state
% *** condition
% *** inputs
% **** filter
% *** outputs
\end{itemize}
\item \textbf{Message types} - These are a list of all the messages being used in the
model. The details with in the message are,
\begin{itemize}
\item Name - Name of the message.
\item Description - Textual description of the message.
\item Variables - Variables encapsulated with in the message.
\end{itemize}
\end{itemize}

Refer to the Appendix to see how these tags are brought together in
one model XML file.

\subsection{Model in Multiple Files}

It is possible to define a model in a collection of multiple files. FLAME reads a model from
multiple files as if the model was defined in one file. This capability allows
different parts of a model to be enabled or disabled easily. For example if a
model includes different versions of a sub-model, these can be exchanged, or a
subsystem of a model can be disabled to see how it affects the model.
Alternatively this capability could be used as a hierarchy, for example a `body'
model could include a model of the `cardiovascular system' that includes a
model of the `heart'. The following tags show the inclusion of two models, one is
enabled and one disabled:

\begin{mylisting}
\begin{verbatim}
<models>
  <model><file>sub_model_1.xml</file><enabled>true</enabled></model>
  <model><file>sub_model_2.xml</file><enabled>false</enabled></model>
</models>
\end{verbatim}
\end{mylisting}


\subsection{Environment}

The environment of a model holds information that maybe required by a model but
is not part of an agent or a message. This includes:

\begin{itemize}
\item Constant variables - for setting up different simulations easily.
\item Location of function files - the path to the implementations of agent
functions in C files.
\item Time units - for easily activating agent functions dependent on time
periods.
\item Data types - user defined data types used by agent memory or
message variables other that typical C data types.
\end{itemize}

This notion of environment does not correspond to an environment that would be
a part of a model where agents would interact with the environment. Anything
that can change in a model must be represented by an agent, therefore if a
model includes a changeable environment that agents can interact with, this in
itself must be represented by an agent.

\subsubsection{Constant Variables}

Constant variables can be set up as part of a simulation for the runs. These are defined as follows:

\begin{mylisting}
\begin{verbatim}
<constants>
  <variable>
   <type>int</type><name>my_constant</name>
   <description>value read in initial simulation settings</description>
  </variable>
</constants>
\end{verbatim}
\end{mylisting}


Constant Variables refers to the global values used in the model. These can also be
defined in a separate header H file which can then be included in one of the
functions C file.
The header file should contain the global variable as:

 \begin{mylisting}
 \begin{verbatim}
#define <varname> <value>
 \end{verbatim}
 \end{mylisting}

 This file has to be saved as `my\_header.h' file, include this file into one of
 the function files so that the compiler knows about these arguments.

\subsubsection{Function Files}

Function files hold the source code for the implementation of the
agent functions. These are programmed in C language.
They are included in the compilation script (Makefile) of the produced model:

\begin{mylisting}
\begin{verbatim}
 <functionFiles>
 <file>function_source_code_1.c</file>
 <file>function_source_code_2.c</file>
 </functionFiles>
\end{verbatim}
\end{mylisting}

\subsubsection{Time Units}
\label{timeunit}

% Time units allow the possibility of restricting the functions to
% only execute during particular iterations.
% Time rules can be applied to function conditions instead of a
% condition rule and are defined by a time period and a phase. A time
% phase is the offset from the start of a period.

Time units are used to define time periods that agent functions act within. For
example a model that uses a calendar based time system could take a day to be
the smallest time step, i.e. one iteration. Other time units can then use this
definition to define other time units, for example weeks, months, and years.

A time unit contains:

\begin{itemize}
\item Name - name of the time unit.
\item Unit - can contain `iteration' or other defined time units.
\item Period - the length of the time unit using the above units.
\end{itemize}

An example of a calendar based time unit set up is given below:

\begin{mylisting}
\begin{verbatim}
<timeUnits>
  <timeUnit>
    <name>daily</name>
    <unit>iteration</unit>
    <period>1</period>
  </timeUnit>

  <timeUnit>
    <name>weekly</name>
    <unit>daily</unit>
    <period>5</period>
  </timeUnit>

  <timeUnit>
    <name>monthly</name>
    <unit>weekly</unit>
    <period>4</period>
  </timeUnit>

  <timeUnit>
    <name>quarterly</name>
    <unit>monthly</unit>
    <period>3</period>
  </timeUnit>

  <timeUnit>
    <name>yearly</name>
    <unit>monthly</unit>
    <period>12</period>
  </timeUnit>

</timeUnits>
\end{verbatim}
\end{mylisting}

These time units can be added to the functions, when they are listed as part of the agent. These time units act as conditions on the functions. This has been discussed in Section \ref{functioncond}.

\subsubsection{Data Types}

Data types are user defined data types that can be used in a model. They are a
structure for holding variables. Variables can be a:

\begin{itemize}
  \item Single C fundamental data types - int, float, double, char.
  \item Static array - of any size for example ten is written as `variable\_name[10]'.
  \item Dynamic array - available by placing `\_array' after
  the data type name: variable\_name\_array.
  \item User defined data type - defined before the current data type.
\end{itemize}

The example below contains a variable
of data structure \emph{position} which contains the x, y and z position in one structure. The position data structure can then be a data type in the \emph{line} data structure.

\begin{mylisting}
\begin{verbatim}
<dataTypes>

 <dataType>
  <name>position/name>
  <description>position in 3D using doubles</description>
  <variables>
   <variable><type>double</type><name>x</name>
    <description>position on x-axis</description>
   </variable>
   <variable><type>double</type><name>y</name>
    <description>position on y-axis</description>
   </variable>
   <variable><type>double</type><name>z</name>
    <description>position on z-axis</description>
   </variable>
  </variables>
 </dataType>

 <dataType>
  <name>line</name>
  <description>a line defined by two points</description>
  <variables>
   <variable><type>position</type><name>start</name>
    <description>start position of the line</description>
   </variable>
   <variable><type>position</type><name>end</name>
    <description>end position of the line</description>
   </variable>
  </variables>
 </dataType>

</dataTypes>
\end{verbatim}
\end{mylisting}

\subsection{Agents}

A model has to constitute agents. These agents are defined as their type in the model xml file. An agent type contains a name, a description, memory, and functions:

\begin{mylisting}
\begin{verbatim}
<agents>

  <xagent>
    <name>Agent_Name</name>
    <description></description>
    <memory>
     ...
    </memory>
    <functions>
      ...
    </functions>
  </xagent>
\end{verbatim}
\end{mylisting}
%
%   <xagent>
%     <name>Household</name>
%     <description></description>
%     <memory>
%       <variable><type>int</type><name>id</name>
%        <description></description>
%       </variable>
%       <variable><type>int</type><name>region_id</name>
%        <description></description>
%       </variable>
%       <variable><type>int_array</type><name>neighboring_region_ids</name>
%        <description></description>
%       </variable>
%       <variable><type>int</type><name>gov_id</name>
%        <description></description>
%       </variable>
%       <variable><type>int</type><name>day_of_month_to_act</name>
%        <description></description>
%       </variable>
%       <variable><type>double</type><name>payment_account</name>
%        <description></description>
%       </variable>
%     </memory>
%     <functions>
%      <function>
%        <name>Household_read_firing_messages</name>
%         <description>Check for being fired or not</description>
%         <currentState>EXIT_FINANCIAL_MARKET</currentState>
%         <nextState>01d</nextState>
%         <condition>
%          <lhs><value>a.employee_firm_id</value></lhs>
%          <op>NEQ</op>
%          <rhs><value>-1</value></rhs>
%         </condition>
%         <inputs>
%          <input><messageName>firing</messageName></input>
%         </inputs>
%       </function>
%     </functions>
%   </xagent>
% </agents>


\subsubsection{Agent Memory}

Agent memory defines variables, where variables are defined by their type, C
data types or user defined data types from the environment, a name, and a
description:

\begin{mylisting}
\begin{verbatim}
<memory>
 <variable><type>int</type><name>id</name>
  <description>identity number</description>
 </variable>
 <variable><type>double</type><name>x</name>
  <description>position in x-axis</description>
 </variable>
  <variable><type>position</type><name>xyz</name>
  <description>position in x-axis, y-axis, z-axis</description>
 </variable>
</memory>
\end{verbatim}
\end{mylisting}

Agent memory variables can be defined as being constant by using the
<constant> tag and defining it to be true. This will stop the
variable being allowed to be changed. This helps message
communication in parallel when input filters are dependent upon
constant agent memory variables.
\begin{mylisting}
\begin{verbatim}
<variable>
<type>int</type><name>id</name><constant>true</constant><description></description>
</variable>
\end{verbatim}
\end{mylisting}


\subsubsection{Agent Functions}

The model XML file requires the agent functions to be listed as well
to tell FLAME when the functions will be called in from the C files.
An agent function contains:

\begin{itemize}
\item Name - the function name which must correspond to an implemented function
name
\item Description
\item Current state - the current state the agent has to be in for this function to execute.
\item Next state - the next state the agent will transition to after the function.
\item Condition - a possible condition of the function transition.
\item Inputs - the possible input messages.
\item Outputs - the possible output messages.
\end{itemize}

And as tags, the xml file will contain:

\begin{mylisting}
\begin{verbatim}
<function>
 <name>function_name</name>
 <description>function description</description>
 <currentState>current_state</currentState>
 <nextState>next_state</nextState>
 <condition>
 ...
 </condition>
 <inputs>
 ...
 </inputs>
 <outputs>
 ...
 </outputs>
</function>
\end{verbatim}
\end{mylisting}

The current state and next state tags hold the names of states. This is the
only place where states are defined. State names must coordinate with other
functions states to produce a transitional graph from the start state to end
states.

\subsubsection{Function Condition}
\label{functioncond}

A function can have a condition on its transition. This condition can include
conditions on the agent memory and also on any time units defined in the
environment. Each transition will take the agent from a starting state to an end state at the end of the simulation.

Each possible transition
must be mutually exclusive. This means that if a certain condition is true on one part of the branch of functions, there should be an alternate branch which would be the opposite of this condition. This will ensure the model does not halt in the middle during simulation if the condition fails. A function named `idle' is available to be used for
functions that do not require an implementation and a reverse of the conditions.

Conditions (that are not just time unit based) take the form:

\begin{itemize}
  \item lhs - left hand side of comparison.
  \item op - the comparison operator.
  \item rhs - the right hand side of the comparison.
\end{itemize}

Or in tags form:

\begin{mylisting}
\begin{verbatim}
<lhs></lhs><op></op><rhs></rhs>
\end{verbatim}
\end{mylisting}

Sides to compare (lhs or rhs) can be either a value, denoted within value tags or
a formula.
Values and formulas can include agent variables, which are preceded by `a', or message variables, which are preceded by `m.'.

 \begin{mylisting}
 \begin{verbatim}
 a.agent_var
 m.message_var
 \end{verbatim}
 \end{mylisting}

The comparison operator, op, can be one of the following comparison functions:

\begin{itemize}
\item EQ - equal to.
\item NEQ - not equal to.
\item LEQ - less than or equal to.
\item GEQ - greater than or equal to.
\item LT - less then.
\item GT - greater than.
\item IN - an integer (in lhs) is a member of an array of integers (in rhs).
\end{itemize}

Or one of the following logic operators can be used as well:

\begin{itemize}
\item AND
\item OR
\end{itemize}

The operator `NOT' can be used by placing `not' tags around a comparison rule.
For example the following tagged rule describes the condition being true when
the `z' variable of the agent is greater than zero and less than ten:

\begin{mylisting}
\begin{verbatim}
<condition>
 <lhs>
  <lhs><value>a.z</value></lhs>
  <op>GT</op>
  <rhs><value>0.0</value></rhs>
 </lhs>
 <op>AND</op>
 <rhs>
  <not>
  <lhs><value>a.z</value></lhs>
  <op>LT</op>
  <rhs><value>10.0</value></rhs>
  </not>
 </rhs>
</condition>
\end{verbatim}
\end{mylisting}

\subsubsection{Time conditions}

A condition can also depend on any time units described in the environment. For
example the following condition is true when the agent variable
`day\_of\_month\_to\_act' is equal to the number of iterations since of the
start, the phase, of the `monthly' period, i.e. twenty iterations as defined in
the time unit:

\begin{mylisting}
\begin{verbatim}
 <condition>
     <time>
     <period>monthly</period>
     <phase>a.day_of_month_to_act</phase>
     </time>
 </condition>
\end{verbatim}
\end{mylisting}

 The condition allows the function to run \emph{monthly} at the phase
 of \emph{day\_of\_month\_to\_act}. The
 \emph{day\_of\_month\_to\_act} is a variable extracted from the
 agent memory and is thus defined as
 \emph{a.day\_of\_month\_to\_act}.

%
% Refer to section \ref{functioncond} for more details on function
% condition definitions.
%
% These rules are then parsed into rule functions and placed in a file
% called rules.c
\subsubsection{Messages in and out of Functions}
Functions can have input and output message types. For example, the following
example the function takes message types `a' and `b' as inputs and outputs
message type `c':

\begin{mylisting}
\begin{verbatim}
<inputs>
 <input><messageName>a</messageName></input>
 <input><messageName>b</messageName></input>
</inputs>
<outputs>
 <output><messageName>c</messageName></output>
</outputs>
\end{verbatim}
\end{mylisting}

%\paragraph{Message Filter}
\subsubsection{Message Filters}\label{sect:msgfilter}
Message filters can be applied to message inputs to allow the messages to be
filtered. Filters are defined similar to function conditions but include
message variables which are prefixed by an `m'.

The various tags associated with message filters are as follows:
\begin{itemize}
\item Conditions on the value of a variable within the message. This
is denoted by the lhs, op and rhs operators.

The following example filter only allows messages where the agent
variable `id' is equal to the message variable `worker\_id',

\begin{mylisting}
\begin{verbatim}
<input>
 <messageName>firing</messageName>
 <filter>
  <lhs><value>a.id</value></lhs>
  <op>EQ</op>
  <rhs><value>m.worker_id</value></rhs>
 </filter>
 <random>true<random>
</input>
\end{verbatim}
\end{mylisting}

The previous example also includes the use of a random tag, set to
false, to show that the input does not need to be randomised, as
randomising input messages can be computationally expensive. By
default all message inputs are not being randomised.

\item IN tag. Message input filters can now accept the `IN' operator.
The IN operator accepts a single integer in the <lhs> tag and an integer array
(static or dynamic) in the <rhs> tag. The filter returns true for any single integer
that is a member of the integer array. For example:
\begin{mylisting}
\begin{verbatim}
<filter>
  <lhs><value>m.id</value></lhs>
  <op>IN</op>
  <rhs><value>a.id_array</value></rhs>
</filter>
\end{verbatim}
\end{mylisting}
\item The random tag. The random tag defines if the input needs to be randomised or not, either `true' or
`false'. By default inputs are NOT randomised.
\begin{mylisting}
\begin{verbatim}
<random>true</random>
\end{verbatim}
\end{mylisting}
\item The sort tag. A sort can be defined for a message input by defining the message variable to be sorted, the `key',
and the order of the sort, either `ascend' or `descend'. The following example orders the messages with the highest values of the variable `wage' first.
By defining random to be true similar values will be randomly
sorted.
\begin{mylisting}
\begin{verbatim}
<sort><key>wage</key><order>descend</order></sort>
\end{verbatim}
\end{mylisting}


\end{itemize}




Using filters in the model description enables FLAME to make message
communication more efficient by pre-sorting messages and using other techniques.

% Thus in the above example messages will be filtered according to the
% message variable \emph{worker\_id} (defined as m.<varname>) to be EQ
% (equal) to the agent \emph{id} (defined as a.<varname>).

\subsection{Messages}

Messages defined in a model must have a type which is defined by a name and the
variables that are included in the message. The following example is a message
called `signal' that holds a position in 3D.

\begin{mylisting}
\begin{verbatim}
<messages>

 <message>
  <name>signal</name>
  <description>Holds the position of the sending agent</description>
  <variables>
    <variable><type>double</type><name>x</name>
     <description>The x-axis position</description>
    </variable>
    <variable><type>double</type><name>y</name>
     <description>The y-axis position</description>
    </variable>
    <variable><type>double</type><name>z</name>
     <description>The z-axis position</description>
    </variable>
  </variables>
 </message>

</messages>
\end{verbatim}
\end{mylisting}

\include{model_implementation}
\section{Model Execution}
\label{model_execution}

FLAME contains a parser program called `xparser' that parses a model
XML definition into simulation program source code. This can be compiled
together with model implementation source code for the simulations. The xparser includes
template files which are used to generate the simulation program source code.

The xparser takes as parameters the location of the model file and an option
for serial or parallel (MPI) version, serial being the default if the option is
not specified.

\subsection{Xparser Generated Files}

The xparser  generates simulation source code files in the same directory
as the model file. These files are:

\begin{itemize}
  \item Doxyfile - a configuration file for generating documentation using
 the program `doxygen'.
  \item header.h - a C header file for global variables and function
  declarations between source code files.
  \item low\_primes.h - holds data used for partitioning agents.
  \item main.c - the source code file containing the main program loop.
  \item Makefile - the compilation script used by the program `make'.
  \item memory.c - the source code file that handles the memory requirements
  of the simulation.
  \item xml.c - the source code file that handles inputs and outputs of the
  simulation.
  \item <agent\_name>\_agent\_header.h - the header file containing macros for
  accessing agent memory variables.
  \item rules.c - the source code file containing the generated rules for
  function conditions and message input filters.
  \item messageboards.c - deprecated?
  \item partitioning.c - still used?
\end{itemize}

For running in parallel, additional files are generated:

\begin{itemize}
  \item propagate\_messages.c - deprecated?
  \item propagate\_agents.c - still used?
\end{itemize}

The simulation source code files then require compilation, which can be easily
achieved using the included compilation script `Makefile' using the `make'
build automation tool. The program `make' invokes the `gcc' C compiler, which
are both free and available on various operating systems. If the parallel
version of the simulation was specified the compiler invoked by `make' is
`mpicc' which is a script usually available on parallel systems.

The compiled program is called `main'. The parameters required to run a
simulation include the number of iterations to run for and the initial start
states (memory) of the agents, currently a formatted XML file.

\subsection{Start States Files 0.xml}

The format of the initial start states XML is given by the following example:

\begin{mylisting}
\begin{verbatim}
<states>
<itno>0</itno>

<environment>
<my_constant>6</my_constant>
</environment>

<xagent>
<name>agent_name</name>
<var_name>0</var_name>
...
</xagent>

...

</states>
\end{verbatim}
\end{mylisting}

The root tag is called `states' and the `itno' tag holds the iteration number
that these states refer to. If there are any environment constants these are
placed within the `environment' tags. Any agents that exist are defined within
`xagent' tags and require the name of the agent within `name' tags. Any agent
memory variable (or environment constant) value is defined within tags with
the name of the variable. Arrays and data types are defined within curly
brackets with commas between each element.

When a simulation is running after every iteration, a states file is produced
in the same directory and in the same format as the start states file with the
values of each agent's memory.

\subsection{Running a Simulation}

After writing the model xml file and C functions files of the agent, the xparser has to be used to compile the simulation program. This is done by going into where the xparser is placed and writing the following commands:

\begin{mylisting}
\begin{verbatim}
FLAME\_xparser> xparser.exe ../model/model.xml
\end{verbatim}
\end{mylisting}

This creates all files which contain details of running the program. Extra files are created in `.dot' format which can be opened using Graphviz. The dot files represent graph structures of the agents which show a description of how the model will work.

By default, the xparser will generate files for running the model in a serial format. If parallel version of the model was required then just an extra tag has to be added,

\begin{mylisting}
\begin{verbatim}
FLAME\_xparser> xparser.exe ../model/model.xml -p
\end{verbatim}
\end{mylisting}

The parallel version, by default, produces code for geometric partitioning of the agents depending on the locations. 

After creating these files, users have to go into the folder where the model was located and compile the files.

\begin{mylisting}
\begin{verbatim}
Model>make 
\end{verbatim}
\end{mylisting}

This creates a main program which is the main simulation program. The main.exe file can then be linked with the initial start states and the number of iterations wanted to be written out.
\begin{mylisting}
\begin{verbatim}
Model>main.exe 10 its/0.xml
\end{verbatim}
\end{mylisting}

Main.exe is the simulation program, 10 is the number of iterations to produce ad its/0.xml is the initial start states of the model which the modeller defined.

\begin{mylisting}
\begin{verbatim}
Model>mpirun -np 2 main.exe 10 its/0.xml -r
\end{verbatim}
\end{mylisting}

If the model is being executed in parallel, the mpirun is called to use MPI (Message passing Interface) for running the model. 2 denotes the number of nodes the model is being divided over and the `-r' flag denotes a round robin distribution of the agents over the modes. This flag is optionary.



\appendix

\section{XML DTD}                       % B
\label{cha_xmldtd}

\small{{\tt \verbatiminput{xmml.dtd}}}


%\bibliographystyle{alpha}
%\bibliography{EURACE_refs}

\end{document}
