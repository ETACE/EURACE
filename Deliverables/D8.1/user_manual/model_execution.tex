\section{Model Execution}
\label{model_execution}

FLAME contains a parser program called `xparser' that parses a model
XML definition into simulation program source code. This can be compiled
together with model implementation source code for the simulations. The xparser includes
template files which are used to generate the simulation program source code.

The xparser takes as parameters the location of the model file and an option
for serial or parallel (MPI) version, serial being the default if the option is
not specified.

\subsection{Xparser Generated Files}

The xparser  generates simulation source code files in the same directory
as the model file. These files are:

\begin{itemize}
  \item Doxyfile - a configuration file for generating documentation using
 the program `doxygen'.
  \item header.h - a C header file for global variables and function
  declarations between source code files.
  \item low\_primes.h - holds data used for partitioning agents.
  \item main.c - the source code file containing the main program loop.
  \item Makefile - the compilation script used by the program `make'.
  \item memory.c - the source code file that handles the memory requirements
  of the simulation.
  \item xml.c - the source code file that handles inputs and outputs of the
  simulation.
  \item <agent\_name>\_agent\_header.h - the header file containing macros for
  accessing agent memory variables.
  \item rules.c - the source code file containing the generated rules for
  function conditions and message input filters.
  \item messageboards.c - deprecated?
  \item partitioning.c - still used?
\end{itemize}

For running in parallel, additional files are generated:

\begin{itemize}
  \item propagate\_messages.c - deprecated?
  \item propagate\_agents.c - still used?
\end{itemize}

The simulation source code files then require compilation, which can be easily
achieved using the included compilation script `Makefile' using the `make'
build automation tool. The program `make' invokes the `gcc' C compiler, which
are both free and available on various operating systems. If the parallel
version of the simulation was specified the compiler invoked by `make' is
`mpicc' which is a script usually available on parallel systems.

The compiled program is called `main'. The parameters required to run a
simulation include the number of iterations to run for and the initial start
states (memory) of the agents, currently a formatted XML file.

\subsection{Start States Files 0.xml}

The format of the initial start states XML is given by the following example:

\begin{mylisting}
\begin{verbatim}
<states>
<itno>0</itno>

<environment>
<my_constant>6</my_constant>
</environment>

<xagent>
<name>agent_name</name>
<var_name>0</var_name>
...
</xagent>

...

</states>
\end{verbatim}
\end{mylisting}

The root tag is called `states' and the `itno' tag holds the iteration number
that these states refer to. If there are any environment constants these are
placed within the `environment' tags. Any agents that exist are defined within
`xagent' tags and require the name of the agent within `name' tags. Any agent
memory variable (or environment constant) value is defined within tags with
the name of the variable. Arrays and data types are defined within curly
brackets with commas between each element.

When a simulation is running after every iteration, a states file is produced
in the same directory and in the same format as the start states file with the
values of each agent's memory.

\subsection{Running a Simulation}

After writing the model xml file and C functions files of the agent, the xparser has to be used to compile the simulation program. This is done by going into where the xparser is placed and writing the following commands:

\begin{mylisting}
\begin{verbatim}
FLAME\_xparser> xparser.exe ../model/model.xml
\end{verbatim}
\end{mylisting}

This creates all files which contain details of running the program. Extra files are created in `.dot' format which can be opened using Graphviz. The dot files represent graph structures of the agents which show a description of how the model will work.

By default, the xparser will generate files for running the model in a serial format. If parallel version of the model was required then just an extra tag has to be added,

\begin{mylisting}
\begin{verbatim}
FLAME\_xparser> xparser.exe ../model/model.xml -p
\end{verbatim}
\end{mylisting}

The parallel version, by default, produces code for geometric partitioning of the agents depending on the locations. 

After creating these files, users have to go into the folder where the model was located and compile the files.

\begin{mylisting}
\begin{verbatim}
Model>make 
\end{verbatim}
\end{mylisting}

This creates a main program which is the main simulation program. The main.exe file can then be linked with the initial start states and the number of iterations wanted to be written out.
\begin{mylisting}
\begin{verbatim}
Model>main.exe 10 its/0.xml
\end{verbatim}
\end{mylisting}

Main.exe is the simulation program, 10 is the number of iterations to produce ad its/0.xml is the initial start states of the model which the modeller defined.

\begin{mylisting}
\begin{verbatim}
Model>mpirun -np 2 main.exe 10 its/0.xml -r
\end{verbatim}
\end{mylisting}

If the model is being executed in parallel, the mpirun is called to use MPI (Message passing Interface) for running the model. 2 denotes the number of nodes the model is being divided over and the `-r' flag denotes a round robin distribution of the agents over the modes. This flag is optionary.

