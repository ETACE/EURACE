\section{Model Design}\label{chap:design}
\label{model_design}

Traditionally specifying software behaviour has used finite state
machines. Extended finite state machines (X-machines) are more
powerful than the simple finite state machine and are used to
represent the agents. Using these machines, the following
characteristics of the agents are identified:

\begin{itemize}
\item A finite set of internal states
\item Set of transitions functions that operate between the states.
\item An internal memory set of the agent.
\item A language for sending and receiving messages among agents.
\end{itemize}


Conventional state machines describe the state-dependent behaviour
of a system in terms of its inputs, but this fails to include the
effect of data. X-Machines are an extension to conventional state
machines that include the manipulation of memory as part of the
system behaviour, and thus are a suitable way to specify agents.
Describing a system would thus include the following individual
stages for creating a model:

\begin{itemize}
\item Identifying the agents and their functions.
\item Identify the states which impose some order of function
execution.
\item Identify the input messages and output messages of each function
(including possible filters on inputs).
\item Identify the memory as the set of variables that are accessed by
functions (including possible conditions on variables for the
functions to occur).
\end{itemize}



\begin{figure}[!htb]
\begin{center}
  \includegraphics*[scale=0.45]{commxm.eps}
  \caption{How two agent x-machines communicate. The agents send and read messages from the message board which maintains a database of all the messages sent by the agents.}
  \label{fig:commxm}
  \end{center}
\end{figure}

Agent functions can accept an input stream of messages This is
handled by the implementation of each function written in the source
code.

\subsection{Swarm Example}

A swarm model is presented here as an example of how the agents
behave. Swarm model describes the swarming behaviour in birds where
they produce various patterns during flight. This simple flocking
model would include agents which have to sense where other agents
are and then respond accordingly. The functions they would possess
would be:

\begin{itemize}
\item Signal. The agent would send information of its current
position.
\item Observe. The agent would read in the positions from other agents and possibly change
velocity.
\item Respond. The agent would update position via the current
velocity.
\end{itemize}

The functions would occur in an order as seen in Figure
\ref{fig:swarm_1}. The agents would traverse through the states
during one iteration. FLAME prevents the agents to loop back due to
parallelisation constraints.



\begin{figure}[ht]
\begin{center}
\includegraphics*[scale=0.65]{swarm_1.ps}
\caption{Swarm model including states}
\label{fig:swarm_1}
\end{center}
\end{figure}

Functions can also have conditions in the model. For instance, in
the swarm model, there can be a response function for flying and an
alternate for resting on the ground. The condition on the flying
response function would be that the z-axis position of the agent be
greater than zero while the resting response function condition
would be when the z-axis position was zero, see Figure
\ref{fig:swarm_2}.

\begin{figure}[ht]
\begin{center}
\includegraphics*[scale=0.65]{swarm_2.ps}
\caption{Swarm model including function conditions}
\label{fig:swarm_2}
\end{center}
\end{figure}


\subsection{Transition Function}
 The transition functions allow the agents to change the state in
 which they are in, modifying their behaviour accordingly. These would
 require as inputs their current state $s_{1}$, current memory value
 $m_{1}$, and the possible arrival of a message that the agent is able to
 read, $t_{1}$. Depending on these three values the agent can then
 change to another state $s_{2}$, updates the memory to $m_{2}$ and
 optionally sends a message, $t_{2}$. Figure
 \ref{fig:trans} depicts how the transition function
 works within the agent.

The messages required for communication between agents are a signal message,
which is output from `signal' and input to `observe', see Figure
\ref{fig:swarm_3}. This message would include the position of the agent that
sent it, see Table \ref{tab:signal_message}. A feature of swarm models and most
agent-based models is that there is generally a limit on incoming communication.
In the swarm case this is the perceived distance of sight that an agent can view the location of other
agents. This feature can be added to the model as a filter on inputs to a
function, where the filter is a formula involving the position contained in the
message (the position of the sending agent) and the receiving agent position.


 Some of the transition functions may not depend on the incoming
 message. Thus the message would then be represented as:
 \begin{equation}\label{msg}
     Message = \{ \emptyset, <data> \}
 \end{equation}

 These agent transition functions may be expressed in terms of
 stochastic rules, thus allowing the multi-agent systems to be termed
 as stochastic systems.

 \subsection{Memory and States}
 The difference between the internal set of states and the internal
 memory set allows for added flexibility when modelling systems.
 There can be agents with one internal state and all the complexity
 defined in the memory or equivalently, there could be agents with
 a trivial memory with the complexity then bound up in a large state
 space. There are good examples of choosing an appropriate balance
 between these two as this enables the complexity of the models to be
 better managed.

\begin{table}[ht]
\centering
\begin{tabular}{|l||c||l|}
\hline
Type&Name&Description\\
\hline \hline
double&px&x-axis position\\
\hline
double&py&y-axis position\\
\hline
double&pz&z-axis position\\
\hline
\end{tabular}
\caption{Signal Message}
\label{tab:signal_message}
\end{table}

\begin{figure}[ht]
\begin{center}
\includegraphics*[scale=0.65]{swarm_3.ps}
\caption{Swarm model including messages}
\label{fig:swarm_3}
\end{center}
\end{figure}

Functions that take a message type as input are only executed once all functions
that output the same message type have finished. One iteration is taken as a
standalone run of a simulation, so once all the functions that have a message
type as an input have been executed, the messages are deleted as they are no
longer required. Messages cannot be sent between iterations.

Finally the memory required by the agent functions include the position of the
agent, and its velocity, as shown in Table \ref{tab:swarm_memory}.

\begin{table}[ht]
\centering
\begin{tabular}{|l||c||l|}
\hline
Type&Name&Description\\
\hline \hline
double&px&position in x-axis\\
\hline
double&py&position in y-axis\\
\hline
double&pz&position in z-axis\\
\hline
double&vx&velocity in x-axis\\
\hline
double&vy&velocity in y-axis\\
\hline
double&vz&velocity in z-axis\\
\hline
\end{tabular}
\caption{Swarm Agent Memory}
\label{tab:swarm_memory}
\end{table}

The swarm model can also be represented as a transition table, see Table
\ref{tab:swarmtransition}, where:

\begin{itemize}
  \item Current State -- is the state the agent is currently in.
  \item Input -- is any inputs into the transition function.
  \item $M_{pre}$ -- are any preconditions of the memory on the transition.
  \item Function -- is the function name.
  \item $M_{post}$ -- is any change in the agent memory.
  \item Output -- is any outputs from the transition.
  \item Next State -- is the next state that is entered by the agent.
\end{itemize}

%\begin{landscape}
\begin{table}[ht]
\centering
\begin{tabular}{|c|c|c||c||c|c|c|}
\hline
Current State&Input&$M_{pre}$&Function&$M_{post}$&Output&Next State\\
\hline
\hline
start&&&signal&&signal&1\\
\hline
1&signal&&observe&(velocity updated)&&2\\
\hline
2&&$x > 0$&flying&(position updated)&&end\\
\hline
2&&$x == 0$&resting&(position updated)&&end\\
\hline
\end{tabular}
\caption{Swarm Agent Transition Table}
\label{tab:swarmtransition}
\end{table}
%\end{landscape}

Section \ref{model_description} on model description describes how to write a
model description into an XML file that FLAME can understand. Section
\ref{model_implementation} on model implementation describes how to implement
the individual agent functions, i.e. $M_{post}$ from the transition table.
Section \ref{model_execution} on model execution describes how to use the tools
in FLAME to generate a simulation program, compile it, and run it.

% \begin{equation}\label{streamxmachine}
%     X = (\Sigma, \Gamma, Q, M, \Phi, F, q_{0}, m_{0})
% \end{equation}
% where,
% \begin{itemize}
% \item $\Sigma$ are the set of input alphabets
% \item $\Gamma$ are the set of output alphabets
% \item $Q$ denotes the set of states
% \item $M$ denotes the variables in the memory.
% \item $\Phi$ denotes the set of partial functions $\phi$ that map
% and input and memory variable to an output and a change on the
% memory variable. The set $\phi$: $\Sigma \times M\ \longrightarrow\
% \Gamma\times M$
% \item $F$ in the next state transition function, $F : Q \times\phi\longrightarrow
% Q$
% \item $q_{0}$ is the initial state and $m_{0}$ is the initial memory
% of the machine.
% \end{itemize}
%
% \subsection{Transition Function}
% The transition functions allow the agents to change the state in
% which they are in, modifying their behaviour accordingly. These would
% require as inputs their current state $s_{1}$, current memory value
% $m_{1}$, and the possible arrival of a message that the agent is able to
% read, $t_{1}$. Depending on these three values the agent can then
% change to another state $s_{2}$, updates the memory to $m_{2}$ and
% optionally sends a message, $t_{2}$. Figure
% \ref{fig:trans} depicts how the transition function
% works within the agent.
%
% % \begin{figure}
% % \begin{center}
% % \includegraphics*[scale=0.5]{transfn.eps}
% % \caption{Transition function} \label{fig:trans}
% % \end{center}
% % \end{figure}
%
%
% Extended finite state machines or X-Machines are used to define agents within a
% model.
% The basic definition of an
% agent would thus, in accordance to the computational model, contain
% the following components:
% \begin{enumerate}
%  \item A finite set of internal states.
%  \item A set of transition functions that operate between states.
%  \item An internal memory set. In practice, the memory would be a finite set and can be structured in any way required.
%  \item A language for sending and receiving messages between other agents.
% \end{enumerate}
%
%
% Some of the transition functions may not depend on the incoming
% message. Thus the message would then be represented as:
% \begin{equation}\label{msg}
%     Message = \{ \emptyset, <data> \}
% \end{equation}
%
% These agent transition functions may be expressed in terms of
% stochastic rules, thus allowing the multi-agent systems to be termed
% as stochastic systems.
%
% \subsubsection{Memory and States}
% The difference between the internal set of states and the internal
% memory set allows for added flexibility when modelling systems.
% There can be agents with one internal state and all the complexity
% defined in the memory or equivalently, there could be agents with
% a trivial memory with the complexity then bound up in a large state
% space. There are good examples of choosing an appropriate balance
% between these two as this enables the complexity of the models to be
% better managed.

% \begin{figure}
% \begin{center}
% \includegraphics*[width = 4in]{X-Machine_agent.eps}
% \caption{X-Machine agent} \label{fig:xmachine}
% \end{center}
% \end{figure}
