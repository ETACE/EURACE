\section{Testing}



\subsection{Unit Testing}

The testing of individual modules of a piece of software, in the case of models
these are the individual agent functions. Each function has an accompanying
unit test function that sets the initial agent memory and any input, calls the
function, and assets that the new agent memory and outputs are what is expected.

FLAME provides procedures to help with unit testing:

\begin{itemize}
  \item initialise\_unit\_testing()
  \item unittest\_init\_agentname\_agent()
  \item \ldots update agent memory \ldots
  \item \ldots create input messages \ldots
  \item \ldots call function to test \ldots
  \item \ldots call test assertions \ldots
  \item unittest\_free\_agentname\_agent()
  \item free\_messages()
  \item clean\_up()
\end{itemize}

\subsection{Integration Testing}

The testing of combinations of program modules up to the level of the whole
system.

Need coverage of testing combinations, a test set.

The W-method proposed by T. Chow \cite{CHOW:1978} provides a complete test set
of sequences through a state machine.

Used an implementation called statechum (http://statechum.sourceforge.net/)
that accepts graphml as input. Converted each agent into a separate state
machine as graphml. 

One way to analyse the output of runs is to use an invarient detector. This
program reports any likely invariants, properties that hold at certain points.
This could then be analysed by the modeller and possibly used as assertions in
future test runs.
