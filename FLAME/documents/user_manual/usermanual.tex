\documentclass[12pt,a4paper]{report}
\usepackage{t1enc}
\usepackage[latin1]{inputenc}
\usepackage[english]{babel}
\usepackage{url,graphics,lscape}
\usepackage{a4wide}
\usepackage{graphicx}
\usepackage{color}
%\usepackage{fancyheadings}
\usepackage{verbatim}

\newenvironment{mylisting}
{\begin{list}{}{\setlength{\leftmargin}{1em}}\item\small\bfseries}
{\end{list}}

\parskip 6pt         % sets spacing between paragraphs
\parindent 0pt       % sets leading space for paragraphs
%\pagestyle{fancy}
\begin{document}



\title{FLAME
\\User Manual}
\author{Simon Coakley\\Mariam Kiran
\\
\\ Unit - USFD}
%\date{17 May 2006}

\maketitle



\begin{abstract}
The report presents a manual for the FLAME framework. How to design a model,
create a description of the model, and write the implementation of the model.
Details are also included about how to execute a model.
\end{abstract}

\tableofcontents
\pagebreak

\section{Introduction}

The FLAME framework is an enabling tool to create agent-based models that can
be run on high performance computers (HPCs). Models are created based upon
extended finite state machines that include message inputs and outputs. This information
is used by the framework to automatically generate a simulation program that
can run models efficiently on HPCs.

\section{Model Design}

Extended finite state machines or X-Machines are used to define agents within a
model. 
The basic definition of an
agent would thus, in accordance to the computational model, contain
the following components:
\begin{enumerate}
 \item A finite set of internal states.
 \item A set of transition functions that operate between states.
 \item An internal memory set. In practice, the memory would be a finite set and can be structured in any way required.
 \item A language for sending and receiving messages between other agents.
\end{enumerate}

\begin{equation}\label{streamxmachine}
    X = (\Sigma, \Gamma, Q, M, \Phi, F, q_{0}, m_{0})
\end{equation}
where,
\begin{itemize}
\item $\Sigma$ are the set of input alphabets
\item $\Gamma$ are the set of output alphabets
\item $Q$ denotes the set of states
\item $M$ denotes the variables in the memory.
\item $\Phi$ denotes the set of partial functions $\phi$ that map
and input and memory variable to an output and a change on the
memory variable. The set $\phi$: $\Sigma \times M\ \longrightarrow\
\Gamma\times M$
\item $F$ in the next state transition function, $F : Q \times\phi\longrightarrow
Q$
\item $q_{0}$ is the initial state and $m_{0}$ is the initial memory
of the machine.
\end{itemize}

\subsection{Transition Function}
The transition functions allow the agents to change the state in
which they are in, modifying their behaviour accordingly. These would
require as inputs their current state $s_{1}$, current memory value
$m_{1}$, and the possible arrival of a message that the agent is able to
read, $t_{1}$. Depending on these three values the agent can then
change to another state $s_{2}$, updates the memory to $m_{2}$ and
optionally sends a message, $t_{2}$. Figure
\ref{fig:trans} depicts how the transition function
works within the agent.

% \begin{figure}
% \begin{center}
% \includegraphics*[scale=0.5]{transfn.eps}
% \caption{Transition function} \label{fig:trans}
% \end{center}
% \end{figure}

Some of the transition functions may not depend on the incoming
message. Thus the message would then be represented as:
\begin{equation}\label{msg}
    Message = \{ \emptyset, <data> \}
\end{equation}

These agent transition functions may be expressed in terms of
stochastic rules, thus allowing the multi-agent systems to be termed
as stochastic systems.

\subsubsection{Memory and States}
The difference between the internal set of states and the internal
memory set allows for added flexibility when modelling systems.
There can be agents with one internal state and all the complexity
defined in the memory or equivalently, there could be agents with
a trivial memory with the complexity then bound up in a large state
space. There are good examples of choosing an appropriate balance
between these two as this enables the complexity of the models to be
better managed.

% \begin{figure}
% \begin{center}
% \includegraphics*[width = 4in]{X-Machine_agent.eps}
% \caption{X-Machine agent} \label{fig:xmachine}
% \end{center}
% \end{figure}

Specifying software behaviour have traditionally involved finite state
machines which allow modelling a system in terms of its inputs and outputs.
More abstract system descriptions include UML which has already been proposed as a way to design agent-based models \cite{BAUER:2000,BAUER:2001,HUGET:2002,WEISBUCH:2000} but these techniques lack precise descriptions needed for generating simulation code and for testing.
Testing a system specified as a finite state machine makes it easier for the behaviour to be expressed as a graph
and allow traversals of all possible and impossible executions of the system \footnote{This is similar to branch traversal testing.}. Conventional state machines describe the state-dependent behaviour of a system in terms of its inputs, but this fails to include the effect of data.
X-Machines are an extension to conventional state machines that
include the manipulation of memory as part of the system behaviour,
and thus are a suitable way to specify agents. The advantages of this
approach have been highlighted in Section \ref{xmachine}. Describing a system would thus include the following individual
stages for creating a model:

\begin{itemize}
\item Identifying the system functions
\item Identify the states which impose some order of function execution
\item Identify the input messages and output messages
\item For each state identify the memory as the set of variables that are accessed by outgoing and incoming transition functions
\end{itemize}

\section{Model Description}
\label{model_description}

Models descriptions are formatted in XML (Extensible Markup Language) tag
structures to allow easy human and computer readability, and allow easier collaborations between
developers writing applications that interact with model definitions.

The DTD (Document Type Definition) of the XML document is currently located
at:

\begin{mylisting}
\begin{verbatim}
http://eurace.cs.bilgi.edu.tr/XMML.dtd
\end{verbatim}
\end{mylisting}

The start and end of a model file should be formatted as follows:

\begin{mylisting}
\begin{verbatim}
<?xml version="1.0" encoding="ISO-8859-1"?>
<!DOCTYPE xmodel SYSTEM "http://eurace.cs.bilgi.edu.tr/XMML.dtd">
<xmodel version="2">
<name>Model_name</name>
<version>the version</version>
<description>a description</description>
...
</xmodel>
\end{verbatim}
\end{mylisting}

Where name is the name of the model, version is the version, and description
allows the description of the model. Models can also contain:

\begin{itemize}
\item \textbf{Other models} (enabled or disabled)
\item \textbf{Environment}
\begin{itemize}
\item constant variables
\item location of function files
\item time units
% \begin{itemize}
% \item name
% \item *** unit
% \item *** period
% \end{itemize}
\item data types
% \begin{itemize}
% \item name
% \item description
% \item variables
% \end{itemize}
\end{itemize}
\item \textbf{Agent types}
\begin{itemize}
\item name
\item description
\item memory
% *** variables
\item functions
% *** name
% *** description
% *** current state
% *** next state
% *** condition
% *** inputs
% **** filter
% *** outputs
\end{itemize}
\item \textbf{Message types}
\begin{itemize}
\item name
\item description
\item variables
\end{itemize}
\end{itemize}

\subsection{Model in Multiple Files}

It is possible to define a model in multiple files. FLAME reads a model from
multiple files as if the model was defined in one file. This capability allows
different parts of a model to be enabled or disabled easily. For example if a
model includes different versions of a sub-model that can be exchanged, or a
subsystem of a model can be disabled to see how it affects the model.
Alternatively this capability could be used as a hierarchy, for example a `body'
model could include a model of the `cardiovascular system' that includes a
model of the `heart'. The following tags show the inclusion of two models, one
enabled and one disabled:

\begin{mylisting}
\begin{verbatim}
<models>
  <model><file>sub_model_1.xml</file><enabled>true</enabled></model>
  <model><file>sub_model_2.xml</file><enabled>false</enabled></model>
</models>
\end{verbatim}
\end{mylisting}

\subsection{Environment}

The environment of a model holds information that maybe required by a model but
is not part of an agent or a message. This includes:

\begin{itemize}
\item constant variables -- for setting up different simulations easily
\item location of function files -- the path to the implementations of agent
functions
\item time units -- for easily activating agent functions dependent on time
periods
\item data types -- user defined data types used by agent memory or
message variables
\end{itemize}

This notion of environment does not correspond to an environment that would be
a part of a model where agents would interact with the environment. Anything
that can change in a model must be represented by an agent, therefore if a
model includes a changeable environment that agents can interact with, this in
itself must be represented by an agent.

\subsubsection{Constant Variables}

These are constant variables that can be set as part of a simulation runs
initial starting values, and can be defined as follows:

\begin{mylisting}
\begin{verbatim}
<constants>
  <variable>
   <type>int</type><name>my_constant</name>
   <description>value read in initial simulation settings</description>
  </variable>
</constants>
\end{verbatim}
\end{mylisting}

% Constant Variables refers to the global values used in the model. These can me
% defined in a separate header file which can then be included in one of the
% functions file.
%
% The header file would look as follows:
%
% \begin{mylisting}
% \begin{verbatim}
% #define <varname> <value>
% \end{verbatim}
% \end{mylisting}
%
% If this file was saved as a `my\_header.h' file, include this file into one of
% the function files so that the compiler knows about these arguments.

\subsubsection{Function Files}

Function files hold the source code for the implementation of the
agent functions.
They are included in the compilation script (Makefile) of the produced model:

\begin{mylisting}
\begin{verbatim}
 <functionFiles>
 <file>function_source_code_1.c</file>
 <file>function_source_code_2.c</file>
 </functionFiles>
\end{verbatim}
\end{mylisting}

\subsubsection{Time Units}
\label{timeunit}

% Time units allow the possibility of restricting the functions to
% only execute during particular iterations.
% Time rules can be applied to function conditions instead of a
% condition rule and are defined by a time period and a phase. A time
% phase is the offset from the start of a period.

Time units are used to define time periods that agent functions act within. For
example a model that uses a calendar based time system could take a day to be
the smallest time step, i.e. one iteration. Other time units can then use this
definition to define other time units, for example weeks, months, and years.

A time unit contains:

\begin{itemize}
\item name -- name of the time unit.
\item unit -- can contain `iteration' or other defined time units.
\item period -- the length of the time unit using the above units.
\end{itemize}

An example of a calendar based time unit set up is given below:

\begin{mylisting}
\begin{verbatim}
<timeUnits>
  <timeUnit>
    <name>daily</name>
    <unit>iteration</unit>
    <period>1</period>
  </timeUnit>

  <timeUnit>
    <name>weekly</name>
    <unit>daily</unit>
    <period>5</period>
  </timeUnit>

  <timeUnit>
    <name>monthly</name>
    <unit>weekly</unit>
    <period>4</period>
  </timeUnit>

  <timeUnit>
    <name>quarterly</name>
    <unit>monthly</unit>
    <period>3</period>
  </timeUnit>

  <timeUnit>
    <name>yearly</name>
    <unit>monthly</unit>
    <period>12</period>
  </timeUnit>

</timeUnits>
\end{verbatim}
\end{mylisting}

\subsubsection{Data Types}

Data types are user defined data types that can be used in a model. They are a
structure for holding variables. Variables can be:

\begin{itemize}
  \item single C fundamental data types -- int, float, double, char.
  \item static array -- of size ten: variable\_name[10].
  \item dynamic array -- available by placing `\_array' after
  the data type name: variable\_name\_array.
  \item user defined data type -- defined before the current data type.
\end{itemize}

The example below the data type \emph{line} contains a variable
of data type \emph{position} which is defined above it:

\begin{mylisting}
\begin{verbatim}
<dataTypes>

 <dataType>
  <name>position/name>
  <description>position in 3D using doubles</description>
  <variables>
   <variable><type>double</type><name>x</name>
    <description>position on x-axis</description>
   </variable>
   <variable><type>double</type><name>y</name>
    <description>position on y-axis</description>
   </variable>
   <variable><type>double</type><name>z</name>
    <description>position on z-axis</description>
   </variable>
  </variables>
 </dataType>

 <dataType>
  <name>line</name>
  <description>a line defined by two points</description>
  <variables>
   <variable><type>position</type><name>start</name>
    <description>start position of the line</description>
   </variable>
   <variable><type>position</type><name>end</name>
    <description>end position of the line</description>
   </variable>
  </variables>
 </dataType>

</dataTypes>
\end{verbatim}
\end{mylisting}

\subsection{Agents}

An agent type contains a name, a description, memory, and functions:

\begin{mylisting}
\begin{verbatim}
<agents>

  <xagent>
    <name>Agent_Name</name>
    <description></description>
    <memory>
     ...
    </memory>
    <functions>
      ...
    </functions>
  </xagent>
\end{verbatim}
\end{mylisting}
%
%   <xagent>
%     <name>Household</name>
%     <description></description>
%     <memory>
%       <variable><type>int</type><name>id</name>
%        <description></description>
%       </variable>
%       <variable><type>int</type><name>region_id</name>
%        <description></description>
%       </variable>
%       <variable><type>int_array</type><name>neighboring_region_ids</name>
%        <description></description>
%       </variable>
%       <variable><type>int</type><name>gov_id</name>
%        <description></description>
%       </variable>
%       <variable><type>int</type><name>day_of_month_to_act</name>
%        <description></description>
%       </variable>
%       <variable><type>double</type><name>payment_account</name>
%        <description></description>
%       </variable>
%     </memory>
%     <functions>
%      <function>
%        <name>Household_read_firing_messages</name>
%         <description>Check for being fired or not</description>
%         <currentState>EXIT_FINANCIAL_MARKET</currentState>
%         <nextState>01d</nextState>
%         <condition>
%          <lhs><value>a.employee_firm_id</value></lhs>
%          <op>NEQ</op>
%          <rhs><value>-1</value></rhs>
%         </condition>
%         <inputs>
%          <input><messageName>firing</messageName></input>
%         </inputs>
%       </function>
%     </functions>
%   </xagent>
% </agents>


\subsubsection{Agent Memory}

Agent memory defines variables, where variables are defined by their type, C
data types or user defined data types from the environment, a name, and a
description:

\begin{mylisting}
\begin{verbatim}
<memory>
 <variable><type>int</type><name>id</name>
  <description>identity number</description>
 </variable>
 <variable><type>double</type><name>x</name>
  <description>position in x-axis</description>
 </variable>
</memory>
\end{verbatim}
\end{mylisting}

\subsubsection{Agent Functions}

An agent function contains:

\begin{itemize}
\item name - the function name which must correspond to an implemented function
name
\item description
\item current state - the current state the agent has to be in.
\item next state - the next state the agent will transition to.
\item condition - a possible condition of the function transition.
\item inputs - the possible input messages.
\item outputs - the possible output messages.
\end{itemize}

And as tags:

\begin{mylisting}
\begin{verbatim}
<function>
 <name>function_name</name>
 <description>function description</description>
 <currentState>current_state</currentState>
 <nextState>next_state</nextState>
 <condition>
 ...
 </condition>
 <inputs>
 ...
 </inputs>
 <outputs>
 ...
 </outputs>
</function>
\end{verbatim}
\end{mylisting}

The current state and next state tags hold the names of states. This is the
only place where states are defined. State names must coordinate with other
functions states to produce a transitional graph from the start state to end
states.

%\paragraph{Function Condition}
\label{functioncond}

A function can have a condition on its transition. This condition can include
conditions on the agent memory and also on any time units defined in the
environment. At any state with outgoing transitions with conditions it must be
possible for a transition to happen, i.e. it must be possible for every agent
to transition from the start state to an end state. Each possible transition
must be mutually exclusive, i.e. the order that the function conditions are
tested is not defined. A function named `idle' is available to be used for
functions that do not require an implementation.

Conditions (that are not just time unit based) take the form:

\begin{itemize}
  \item lhs -- left hand side of comparison
  \item op -- the comparison operator
  \item rhs -- the right hand side of the comparison
\end{itemize}

Or in tags:

\begin{mylisting}
\begin{verbatim}
<lhs></lhs><op></op><rhs></rhs>
\end{verbatim}
\end{mylisting}

Sides to compare (lhs or rhs) can be either a value, denoted by value tags,
a formula, currently also in value tags, or another comparison rule.
Values and formula can include agent variables which are preceded by `a'.

% \begin{mylisting}
% \begin{verbatim}
% a.agent_var
% m.message_var
% \end{verbatim}
% \end{mylisting}

The comparison operator can be one of the following comparison functions:

\begin{itemize}
\item EQ -- equal to
\item NEQ -- not equal to
\item LEQ -- less than or equal to
\item GEQ -- greater than or equal to
\item LT -- less then
\item GT -- greater than
\item IN -- an integer (in lhs) is a member of an array of integers (in rhs)
\end{itemize}

or can be one of the following logic operators as well:

\begin{itemize}
\item AND
\item OR
\end{itemize}

The operator `NOT' can be used by placing `not' tags around a comparison rule.
For example the following tagged rule describes the condition being true when
the `z' variable of the agent is greater than zero and less than ten:

\begin{mylisting}
\begin{verbatim}
<condition>
 <lhs>
  <lhs><value>a.z</value></lhs>
  <op>GT</op>
  <rhs><value>0.0</value></rhs>
 </lhs>
 <op>AND</op>
 <rhs>
  <not>
  <lhs><value>a.z</value></lhs>
  <op>LT</op>
  <rhs><value>10.0</value></rhs>
  </not>
 </rhs>
</condition>
\end{verbatim}
\end{mylisting}

A condition can also depend on any time units described in the environment. For
example the following condition is true when the agent variable
`day\_of\_month\_to\_act' is equal to the number of iterations since of the
start, the phase, of the `monthly' period, i.e. twenty iterations as defined in
the time unit:

\begin{mylisting}
\begin{verbatim}
 <condition>
     <time>
     <period>monthly</period>
     <phase>a.day_of_month_to_act</phase>
     </time>
 </condition>
\end{verbatim}
\end{mylisting}

% The condition allows the function to run \emph{monthly} at the phase
% of \emph{day\_of\_month\_to\_act}. The
% \emph{day\_of\_month\_to\_act} is a variable extracted from the
% agent memory and is thus defined as
% \emph{a.day\_of\_month\_to\_act}.
%
% Refer to section \ref{functioncond} for more details on function
% condition definitions.
%
% These rules are then parsed into rule functions and placed in a file
% called rules.c

Functions can have input and output message types. For example the following
example the function takes message types `a' and `b' as inputs and outputs
message type `c':

\begin{mylisting}
\begin{verbatim}
<inputs>
 <input><messageName>a</messageName></input>
 <input><messageName>b</messageName></input>
</inputs>
<outputs>
 <output><messageName>c</messageName></output>
</outputs>
\end{verbatim}
\end{mylisting}

%\paragraph{Message Filter}

Message filters can be applied to message inputs and allow the messages to be
filtered. Filters are defined similar to function conditions but include
message variables which are prefixed by an `m'. The following filter only
allows messages where the agent variable `id' is equal to the message variable
`worker\_id':

\begin{mylisting}
\begin{verbatim}
<input>
 <messageName>firing</messageName>
 <filter>
  <lhs><value>a.id</value></lhs>
  <op>EQ</op>
  <rhs><value>m.worker_id</value></rhs>
 </filter>
 <random>false<random>
</input>
\end{verbatim}
\end{mylisting}

The previous example also includes the use of a random tag, set to false, to show
that the input does not need to be randomised, as randomising input messages can be
computationally expensive. By default all message inputs are randomised.

Using filters in the model description enables FLAME to make message
communication more efficient by pre-sorting messages and using other techniques.

% Thus in the above example messages will be filtered according to the
% message variable \emph{worker\_id} (defined as m.<varname>) to be EQ
% (equal) to the agent \emph{id} (defined as a.<varname>).

\subsection{Messages}

Messages defined in a model must have a type which is defined by a name and the
variables that are included in the message. The following example is a message
called `signal' that holds a position in 3D.

\begin{mylisting}
\begin{verbatim}
<messages>

 <message>
  <name>signal</name>
  <description>Holds the position of the sending agent</description>
  <variables>
    <variable><type>double</type><name>x</name>
     <description>The x-axis position</description>
    </variable>
    <variable><type>double</type><name>y</name>
     <description>The y-axis position</description>
    </variable>
    <variable><type>double</type><name>z</name>
     <description>The z-axis position</description>
    </variable>
  </variables>
 </message>

</messages>
\end{verbatim}
\end{mylisting}

\section{Model Implementation}


\section{Model Execution}
\label{model_execution}

FLAME contains a parser program called `xparser' that parses a model
XML definition into simulation program source code. This can be compiled
together with model implementation source code for the simulations. The xparser includes
template files which are used to generate the simulation program source code.

The xparser takes as parameters the location of the model file and an option
for serial or parallel (MPI) version, serial being the default if the option is
not specified.

\subsection{Xparser Generated Files}

The xparser  generates simulation source code files in the same directory
as the model file. These files are:

\begin{itemize}
  \item Doxyfile - a configuration file for generating documentation using
 the program `doxygen'.
  \item header.h - a C header file for global variables and function
  declarations between source code files.
  \item low\_primes.h - holds data used for partitioning agents.
  \item main.c - the source code file containing the main program loop.
  \item Makefile - the compilation script used by the program `make'.
  \item memory.c - the source code file that handles the memory requirements
  of the simulation.
  \item xml.c - the source code file that handles inputs and outputs of the
  simulation.
  \item <agent\_name>\_agent\_header.h - the header file containing macros for
  accessing agent memory variables.
  \item rules.c - the source code file containing the generated rules for
  function conditions and message input filters.
  \item messageboards.c - deprecated?
  \item partitioning.c - still used?
\end{itemize}

For running in parallel, additional files are generated:

\begin{itemize}
  \item propagate\_messages.c - deprecated?
  \item propagate\_agents.c - still used?
\end{itemize

The simulation source code files then require compilation, which can be easily
achieved using the included compilation script `Makefile' using the `make'
build automation tool. The program `make' invokes the `gcc' C compiler, which
are both free and available on various operating systems. If the parallel
version of the simulation was specified the compiler invoked by `make' is
`mpicc' which is a script usually available on parallel systems.

The compiled program is called `main'. The parameters required to run a
simulation include the number of iterations to run for and the initial start
states (memory) of the agents, currently a formatted XML file.

\subsection{Start States Files 0.xml}

The format of the initial start states XML is given by the following example:

\begin{mylisting}
\begin{verbatim}
<states>
<itno>0</itno>

<environment>
<my_constant>6</my_constant>
</environment>

<xagent>
<name>agent_name</name>
<var_name>0</var_name>
...
</xagent>

...

</states>
\end{verbatim}
\end{mylisting}

The root tag is called `states' and the `itno' tag holds the iteration number
that these states refer to. If there are any environment constants these are
placed within the `environment' tags. Any agents that exist are defined within
`xagent' tags and require the name of the agent within `name' tags. Any agent
memory variable (or environment constant) value is defined within tags with
the name of the variable. Arrays and data types are defined within curly
brackets with commas between each element.

When a simulation is running after every iteration, a states file is produced
in the same directory and in the same format as the start states file with the
values of each agent's memory.

\subsection{Running a Simulation}

After writing the model xml file and C functions files of the agent, the xparser has to be used to compile the simulation program. This is done by going into where the xparser is placed and writing the following commands:

\begin{mylisting}
\begin{verbatim}
FLAME\_xparser> xparser.exe ../model/model.xml
\end{verbatim}
\end{mylisting}

This creates all files which contain details of running the program. Extra files are created in `.dot' format which can be opened using Graphviz. The dot files represent graph structures of the agents which show a description of how the model will work.

After creating these files, users have to go into the folder where the model was located and compile the files.

\begin{mylisting}
\begin{verbatim}
Model>make 
\end{verbatim}
\end{mylisting}

This creates a main program which is the main simulation program. The main.exe file can then be linked with the initial start states and the number of iterations wanted to be written out.
\begin{mylisting}
\begin{verbatim}
Model>main.exe 10 its/0.xml
\end{verbatim}
\end{mylisting}

Main.exe is the simulation program, 10 is the number of iterations to produce ad its/0.xml is the initial start states of the model which the modeller defined.




\appendix

\section{XML DTD}                       % B
\label{cha_xmldtd}

\small{{\tt \verbatiminput{xmml.dtd}}}


\bibliographystyle{alpha}
\bibliography{eurace_refs}

\end{document}
